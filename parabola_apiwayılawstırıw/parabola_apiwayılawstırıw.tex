689.1) Установить, что каждое из следующих уравнений является параболическим; каждое из них привести к простейшему виду; установить, какие геометрические образы они определяют; для каждого случая изобразить на чертеже оси первоначальной координатной системы, оси других координатных систем, которые вводятся по ходу решения, и геометрический образ, определяемый данным уравнением: $9 x^2-24 x y+16 y^2-20 x+110 y-50=0$;
689.2) Установить, что каждое из следующих уравнений является параболическим; каждое из них привести к простейшему виду; установить, какие геометрические образы они определяют; для каждого случая изобразить на чертеже оси первоначальной координатной системы, оси других координатных систем, которые вводятся по ходу решения, и геометрический образ, определяемый данным уравнением: $9 x^2+12 x y+4 y^2-24 x-16 y+3=0$;
689.3) Установить, что каждое из следующих уравнений является параболическим; каждое из них привести к простейшему виду; установить, какие геометрические образы они определяют; для каждого случая изобразить на чертеже оси первоначальной координатной системы, оси других координатных систем, которые вводятся по ходу решения, и геометрический образ, определяемый данным уравнением: $16 x^2-24 x y+9 y^2-160 x+120 y+425=0$.
690.1) То же задание, что и в предыдушей задаче, выполнить для уравнений: $9 x^2+24 x y+16 y^2-18 x+226 y+209=0$;
690.2) То же задание, что и в предыдушей задаче, выполнить для уравнений: $x^2-2 x y+y^2-12 x+12 y-14=0$
690.3) То же задание, что и в предыдушей задаче, выполнить для уравнений: $4 x^2+12 x y+9 y^2-4 x-6 y+1=0$.
691. Для любого параболического уравнения доказать, что коэффициенты $A$ и $C$ не могут быть числами разных знаков и что они одновременно не могут обрашаться в нуль.
692. Доказать, что любое параболическое уравнение может быть написано в виде: $(\alpha x+\beta y)^2+2 D x+2 E y+F=0$. Доказать также, что эллиптические и гиперболические уравнения в таком виде не могут быть написаны.
693.1) Установить, что следующие уравнения являются параболическими, и записать каждое из них в виде $(\alpha x+\beta y)^2+2 D x+2 E y+F=0$: $x^2+4 x y+4 y^2+4 x+y-15=0 ;$
693.2) Установить, что следующие уравнения являются параболическими, и записать каждое из них в виде $(\alpha x+\beta y)^2+2 D x+2 E y+F=0$: $9 x^2-6 x y+y^2-x+2 y-14=0$;
693.3) Установить, что следующие уравнения являются параболическими, и записать каждое из них в виде $(\alpha x+\beta y)^2+2 D x+2 E y+F=0$: $25 x^2-20 x y+4 y^2+3 x-y+11=0$;
693.4) Установить, что следующие уравнения являются параболическими, и записать каждое из них в виде $(\alpha x+\beta y)^2+2 D x+2 E y+F=0$: $16 x^2+16 x y+4 y^2-5 x+7 y=0$;
693.5) Установить, что следующие уравнения являются параболическими, и записать каждое из них в виде $(\alpha x+\beta y)^2+2 D x+2 E y+F=0$: $9 x^2-42 x y+49 y^2+3 x-2 y-24=0$.
694. Доказать, что если уравнение второй степени является параболическим и написано в виде $(\alpha x+\beta y)^2+2 D x+2 E y+F=0$ то дискриминант его левой части определяется формулой $\Delta=-(D \beta-E \alpha)^2$.
695. Доказать, что параболическое уравнение $(\alpha x+\beta y)^2+2 D x+2 E y+F=0$ при помощи преобразования $\begin{aligned}& x=x^{\prime} \cos \theta-y^{\prime} \sin \theta, \\& y=x^{\prime} \sin \theta+y^{\prime} \cos \theta,\end{aligned} \quad \operatorname{tg} \theta=-\frac{\alpha}{\beta}$ приводится к виду $C^{\prime} y^{\prime 2}+2 D^{\prime} x^{\prime}+2 E^{\prime} y^{\prime}+F^{\prime}=0$ где $C^{\prime}=\alpha^2+\beta^2, \quad D^{\prime}= \pm \sqrt{\frac{-\Delta}{a^2+\beta^2}}$ а $\Delta$ - дискриминант левой части данного уравнения.
696. Доказать, что параболическое уравнение определяет параболу в том и только в том случае, когда $\Delta \neq 0$. Доказать, что в этом случае параметр параболы определяется формулой $p=\sqrt{\frac{-\Delta}{(A+C)^3}}$.
697.1) Не проводя преобразования координат, установить, что каждое из следующих уравнений определяет параболу, и найти параметр этой параболы: $9 x^2+24 x y+16 y^2-120 x+90 y=0$;
697.2) Не проводя преобразования координат, установить, что каждое из следующих уравнений определяет параболу, и найти параметр этой параболы: $9 x^2-24 x y+16 y^2-54 x-178 y+181=0$;
697.3) Не проводя преобразования координат, установить, что каждое из следующих уравнений определяет параболу, и найти параметр этой параболы: $x^2-2 x y+y^2+6 x-14 y+29=0$;
697.4) Не проводя преобразования координат, установить, что каждое из следующих уравнений определяет параболу, и найти параметр этой параболы: $9 x^2-6 x y+y^2-50 x+50 y-275=0$.
698. Доказать, что уравнение второй степени является уравнением вырожденной линии в том и только в том случае, когда $\Delta=0$.