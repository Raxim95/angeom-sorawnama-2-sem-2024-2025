673.1) Определить тип каждого из следующих уравнений каждое из них путем параллельного переноса осей координат привести к простейшему виду; установить, какие геометрические образы они определяют, и изобразить на чертеже расположение этих образов относительно старых и новых осей координат: $4 x^2+9 y^2-40 x+36 y+100=0$;
673.2) Определить тип каждого из следующих уравнений каждое из них путем параллельного переноса осей координат привести к простейшему виду; установить, какие геометрические образы они определяют, и изобразить на чертеже расположение этих образов относительно старых и новых осей координат: $9 x^2-16 y^2-54 x-64 y-127=0$;
673.3) Определить тип каждого из следующих уравнений каждое из них путем параллельного переноса осей координат привести к простейшему виду; установить, какие геометрические образы они определяют, и изобразить на чертеже расположение этих образов относительно старых и новых осей координат: $9 x^2+4 y^2+18 x-8 y+49=0$;
673.4) Определить тип каждого из следующих уравнений каждое из них путем параллельного переноса осей координат привести к простейшему виду; установить, какие геометрические образы они определяют, и изобразить на чертеже расположение этих образов относительно старых и новых осей координат: $4 x^2-y^2+8 x-2 y+3=0$;
673.5) Определить тип каждого из следующих уравнений каждое из них путем параллельного переноса осей координат привести к простейшему виду; установить, какие геометрические образы они определяют, и изобразить на чертеже расположение этих образов относительно старых и новых осей координат: $2 x^2+3 y^2+8 x-6 y+11=0$.
674.1) Каждое из следующих уравнений привести к простейшему виду; определить тип каждого из них; установить, какие геометрические образы они определяют, и изобразить на чертеже расположение этих образов относительно старых и новых осей координат: $32 x^2+52 x y-7 y^2+180=0$;
674.2) Каждое из следующих уравнений привести к простейшему виду; определить тип каждого из них; установить, какие геометрические образы они определяют, и изобразить на чертеже расположение этих образов относительно старых и новых осей координат: $5 x^2-6 x y+5 y^2-32=0$;
674.3) Каждое из следующих уравнений привести к простейшему виду; определить тип каждого из них; установить, какие геометрические образы они определяют, и изобразить на чертеже расположение этих образов относительно старых и новых осей координат: $17 x^2-12 x y+8 y^2=0$;
674.4) Каждое из следующих уравнений привести к простейшему виду; определить тип каждого из них; установить, какие геометрические образы они определяют, и изобразить на чертеже расположение этих образов относительно старых и новых осей координат: $5 x^2+24 x y-5 y^2=0$;
674.5) Каждое из следующих уравнений привести к простейшему виду; определить тип каждого из них; установить, какие геометрические образы они определяют, и изобразить на чертеже расположение этих образов относительно старых и новых осей координат: $5 x^2-6 x y+5 y^2+8=0$.
675.1) Определить тип каждого из следующих уравнений при помощи вычисления дискриминанта старших членов: $2 x^2+10 x y+12 y^2-7 x+18 y-15=0$;
675.2) Определить тип каждого из следующих уравнений при помощи вычисления дискриминанта старших членов: $3 x^2-8 x y+7 y^2+8 x-15 y+20=0$;
675.3) Определить тип каждого из следующих уравнений при помощи вычисления дискриминанта старших членов: $25 x^2-20 x y+4 y^2-12 x+20 y-17=0$;
675.4) Определить тип каждого из следующих уравнений при помощи вычисления дискриминанта старших членов: $5 x^2+14 x y+11 y^2+12 x-7 y+19=0$;
675.5) Определить тип каждого из следующих уравнений при помощи вычисления дискриминанта старших членов: $x^2-4 x y+4 y^2+7 x-12=0$;
675.6) Определить тип каждого из следующих уравнений при помощи вычисления дискриминанта старших членов: $3 x^2-2 x y-3 y^2+12 y-15=0$.
676.1) Каждое из следующих уравнений привести к каноническому виду; определить тип каждого из них; установить, какие геометрические образы они определяют; для каждого случая изобразить на чертеже оси первоначальной координатной системы, оси других координатных систем, которые вводятся по ходу решения, и геометрический образ, определяемый данным уравнением: $3 x^2+10 x y+3 y^2-2 x-14 y-13=0$;
676.2) Каждое из следующих уравнений привести к каноническому виду; определить тип каждого из них; установить, какие геометрические образы они определяют; для каждого случая изобразить на чертеже оси первоначальной координатной системы, оси других координатных систем, которые вводятся по ходу решения, и геометрический образ, определяемый данным уравнением: $25 x^2-14 x y+25 y^2+64 x-64 y-224=0$;
676.3) Каждое из следующих уравнений привести к каноническому виду; определить тип каждого из них; установить, какие геометрические образы они определяют; для каждого случая изобразить на чертеже оси первоначальной координатной системы, оси других координатных систем, которые вводятся по ходу решения, и геометрический образ, определяемый данным уравнением: $4 x y+3 y^2+16 x+12 y-36=0$;
676.4) Каждое из следующих уравнений привести к каноническому виду; определить тип каждого из них; установить, какие геометрические образы они определяют; для каждого случая изобразить на чертеже оси первоначальной координатной системы, оси других координатных систем, которые вводятся по ходу решения, и геометрический образ, определяемый данным уравнением: $7 x^2+6 x y-y^2+28 x+12 y+28=0$;
676.5) Каждое из следующих уравнений привести к каноническому виду; определить тип каждого из них; установить, какие геометрические образы они определяют; для каждого случая изобразить на чертеже оси первоначальной координатной системы, оси других координатных систем, которые вводятся по ходу решения, и геометрический образ, определяемый данным уравнением: $19 x^2+6 x y+11 y^2+38 x+6 y+29=0$;
676.6) Каждое из следующих уравнений привести к каноническому виду; определить тип каждого из них; установить, какие геометрические образы они определяют; для каждого случая изобразить на чертеже оси первоначальной координатной системы, оси других координатных систем, которые вводятся по ходу решения, и геометрический образ, определяемый данным уравнением: $5 x^2-2 x y+5 y^2-4 x+20 y+20=0$.
677.1) То же задание, что и в предыдущей задаче, выполнить для уравнений: $14 x^2+24 x y+21 y^2-4 x+18 y-139=0$;
677.2) То же задание, что и в предыдущей задаче, выполнить для уравнений: $11 x^2-20 x y-4 y^2-20 x-8 y+1=0$;
677.3) То же задание, что и в предыдущей задаче, выполнить для уравнений: $7 x^2+60 x y+32 y^2-14 x-60 y+7=0$;
677.4) То же задание, что и в предыдущей задаче, выполнить для уравнений: $50 x^2-8 x y+35 y^2+100 x-8 y+67=0$;
677.5) То же задание, что и в предыдущей задаче, выполнить для уравнений: $41 x^2+24 x y+34 y^2+34 x-112 y+129=0$;
677.6) То же задание, что и в предыдущей задаче, выполнить для уравнений: $29 x^2-24 x y+36 y^2+82 x-96 y-91=0$;
677.7) То же задание, что и в предыдущей задаче, выполнить для уравнений: $4 x^2+24 x y+11 y^2+64 x+42 y+51=0$;
677.8) То же задание, что и в предыдущей задаче, выполнить для уравнений: $41 x^2+24 x y+9 y^2+24 x+18 y-36=0$.
678.1) He проводя преобразования координат, установить, что каждое из следующих уравнений определяет эллипс, и найти величины его полуосей: $41 x^2+24 x y+9 y^2+24 x+18 y-36=0$;
678.2) He проводя преобразования координат, установить, что каждое из следующих уравнений определяет эллипс, и найти величины его полуосей: $8 x^2+4 x y+5 y^2+16 x+4 y-28=0$;
678.3) He проводя преобразования координат, установить, что каждое из следующих уравнений определяет эллипс, и найти величины его полуосей: $13 x^2+18 x y+37 y^2-26 x-18 y+3=0$;
678.4) He проводя преобразования координат, установить, что каждое из следующих уравнений определяет эллипс, и найти величины его полуосей: $13 x^2+10 x y+13 y^2+46 x+62 y+13=0$.
679.1) Не проводя преобразования координат, установить, что каждое из следующих уравнений определяет единственную точку (вырожденный эллипс), и найти ее координаты: $5 x^2-6 x y+2 y^2-2 x+2=0$;
679.2) Не проводя преобразования координат, установить, что каждое из следующих уравнений определяет единственную точку (вырожденный эллипс), и найти ее координаты: $x^2+2 x y+2 y^2+6 y+9=0$;
679.3) Не проводя преобразования координат, установить, что каждое из следующих уравнений определяет единственную точку (вырожденный эллипс), и найти ее координаты: $5 x^2+4 x y+y^2-6 x-2 y+2=0$;
679.4) Не проводя преобразования координат, установить, что каждое из следующих уравнений определяет единственную точку (вырожденный эллипс), и найти ее координаты: $x^2-6 x y+10 y^2+10 x-32 y+26=0$.
680.1) Не проводя преобразования координат, установить, что каждое из следующих уравнений определяет гиперболу, и найти величины ее полуосей: $4 x^2+24 x y+11 y^2+64 x+42 y+51=0$;
680.2) Не проводя преобразования координат, установить, что каждое из следующих уравнений определяет гиперболу, и найти величины ее полуосей: $12 x^2+26 x y+12 y^2-52 x-48 y+73=0$
680.3) Не проводя преобразования координат, установить, что каждое из следующих уравнений определяет гиперболу, и найти величины ее полуосей: $3 x^2+4 x y-12 x+16=0$;
680.4) Не проводя преобразования координат, установить, что каждое из следующих уравнений определяет гиперболу, и найти величины ее полуосей: $x^2-6 x y-7 y^2+10 x-30 y+23=0$.
681.1) Не проводя преобразования координат, установить, что каждое из следующих уравнений определяет пару пересекающихся прямых (вырожденную гиперболу), и найти их уравнения: $3 x^2+4 x y+y^2-2 x-1=0$;
681.2) Не проводя преобразования координат, установить, что каждое из следующих уравнений определяет пару пересекающихся прямых (вырожденную гиперболу), и найти их уравнения: $x^2-6 x y+8 y^2-4 y-4=0$;
681.3) Не проводя преобразования координат, установить, что каждое из следующих уравнений определяет пару пересекающихся прямых (вырожденную гиперболу), и найти их уравнения: $x^2-4 x y+3 y^2=0$;
681.4) Не проводя преобразования координат, установить, что каждое из следующих уравнений определяет пару пересекающихся прямых (вырожденную гиперболу), и найти их уравнения: $x^2+4 x y+3 y^2-6 x-12 y+9=0$.
682.1) Не проводя преобразования координат, установить, какие геометрические образы определяются следующими уравнениями: $8 x^2-12 x y+17 y^2+16 x-12 y+3=0$;
682.2) Не проводя преобразования координат, установить, какие геометрические образы определяются следующими уравнениями: $17 x^2-18 x y-7 y^2+34 x-18 y+7=0$;
682.3) Не проводя преобразования координат, установить, какие геометрические образы определяются следующими уравнениями: $2 x^2+3 x y-2 y^2+5 x+10 y=0$;
682.4) Не проводя преобразования координат, установить, какие геометрические образы определяются следующими уравнениями: $6 x^2-6 x y+9 y^2-4 x+18 y+14=0$;
683. Для любого эллиптического уравнения доказать, что ни один из коэффициентов $A$ и $C$ не может обрашаться в нуль и что они суть числа одного знака.
684. Доказать, что эллиптическое уравнение второй степени ( $\delta>0$ ) определяет эллипс в том и только в том случае, когда $A$ и $\Delta$ суть числа разных знаков.
685. Доказать, что эллиптичсское уравнение второй степсни ( $\delta>0$ ) является уравпением мнимого эллипса в том и только в том случае, когда $A$ и $\Delta$ суть числа одинаковых знаков.
686. Доказать, что эллиптическое уравнение второй степени ( $\delta>0$ ) определяет вырожденный эллипс (точку) в том и только в том случае, когда $\Delta=0$.
