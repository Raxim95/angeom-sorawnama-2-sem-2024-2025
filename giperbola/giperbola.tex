515.1)Составить уравнение гиперболы, фокусы когорой расположены на оси абсцисс симметрично относительно начала координат, зная, кроме того, что: ее оси $2 a=10$ и $2 b=8$;
515.2)Составить уравнение гиперболы, фокусы когорой расположены на оси абсцисс симметрично относительно начала координат, зная, кроме того, что: расстсяние между фокусами $2 c=10$ и ось $2 b=8$;
515.3)Составить уравнение гиперболы, фокусы когорой расположены на оси абсцисс симметрично относительно начала координат, зная, кроме того, что: расстояние между фокусами $2 c=6$ и эксцентриситет $\varepsilon=\frac{3}{2}$;
515.4)Составить уравнение гиперболы, фокусы когорой расположены на оси абсцисс симметрично относительно начала координат, зная, кроме того, что: ось $2 a=16$ и эксцентриситет $\varepsilon=\frac{5}{4}$;
515.5)Составить уравнение гиперболы, фокусы когорой расположены на оси абсцисс симметрично относительно начала координат, зная, кроме того, что: уравнения асимптот $y= \pm \frac{4}{3} x$ и расстояние между фокусами $2 c=20$;
515.6)Составить уравнение гиперболы, фокусы когорой расположены на оси абсцисс симметрично относительно начала координат, зная, кроме того, что: расстояние между директрисами равно $22 \frac{2}{13}$ и расстояние между фокусами $2 c=26$;
515.7)Составить уравнение гиперболы, фокусы когорой расположены на оси абсцисс симметрично относительно начала координат, зная, кроме того, что: расстояние между директрисами равно $\frac{32}{5}$ и ось $2 b=6$;
515.8)Составить уравнение гиперболы, фокусы когорой расположены на оси абсцисс симметрично относительно начала координат, зная, кроме того, что: расстояние между директрисами равно $\frac{8}{3}$ и эксцентриситет $\varepsilon=\frac{3}{2}$;
515.9)Составить уравнение гиперболы, фокусы когорой расположены на оси абсцисс симметрично относительно начала координат, зная, кроме того, что: уравнения асимптот $y= \pm \frac{3}{4} x$ и расстояние между директрисами равно $12 \frac{4}{5}$.
516.1)Составить уравнение гиперболы, фокусы которой расположены на оси ординат симметрично относительно начала координат, зная, кроме того, что: ее полуоси $a=6, b=18$ (буквой $a$ мы обозначаем полуось гинерболы, расположенную на оси абсцисс) ;
516.2)Составить уравнение гиперболы, фокусы которой расположены на оси ординат симметрично относительно начала координат, зная, кроме того, что: расстояние между фокусами $2 c=10$ и эксцентриситет $\varepsilon=\frac{5}{3}$;
516.3)Составить уравнение гиперболы, фокусы которой расположены на оси ординат симметрично относительно начала координат, зная, кроме того, что: уравнения асимптот $y= \pm \frac{12}{5} x$ и расстояние между вершинами равно 48;
516.4)Составить уравнение гиперболы, фокусы которой расположены на оси ординат симметрично относительно начала координат, зная, кроме того, что: расстояние между директрисами равно $7 \frac{1}{7}$ и эксцентриситет $\varepsilon=\frac{7}{5}$;
516.5)Составить уравнение гиперболы, фокусы которой расположены на оси ординат симметрично относительно начала координат, зная, кроме того, что: уравнения асимптот $y= \pm \frac{4}{3} x$ и расстояние между директрисами равно $6 \frac{2}{5}$.
517.1) Определить полуоси $a$ и $b$ каждой из следующих гипербол: $\frac{x^2}{9}-\frac{y^2}{4}=1$;
517.2) Определить полуоси $a$ и $b$ каждой из следующих гипербол: $\frac{x^2}{16}-y^2=1$;
517.3) Определить полуоси $a$ и $b$ каждой из следующих гипербол: $x^2-4 y^2=16$;
517.4) Определить полуоси $a$ и $b$ каждой из следующих гипербол: $x^2-y^2=1$;
517.5) Определить полуоси $a$ и $b$ каждой из следующих гипербол: $4 x^2-9 y^2=25$;
517.6) Определить полуоси $a$ и $b$ каждой из следующих гипербол: $25 x^2-16 y^2=1$;
517.7) Определить полуоси $a$ и $b$ каждой из следующих гипербол: $9 x^2-64 y^2=1$.
518. Дана гипербола $16 x^2-9 y^2=144$. Найти: 1) полуоси $a$ и $b ; 2$ ) фокусы; 3) эксцентриситет; 4) уравнения асимптот; 5) уравнения директрис.
519. Дана гипербола $16 x^2-9 y^2=-144$. Найти: 1) полуоси $a$ и $b ; 2$ ) фокусы; 3) эксцентриситет; 4) уравнения асимптот; 5) уравнения директрис.
520. Вычислить площадь треугольника, образованного асимптотами гиперболы $\frac{x^2}{4}-\frac{y^2}{9}=1$ и прямой $9 x+$ $+2 y-24=0$
521.1) Установить, какие линии определяются следующими уравнениями: $y=+\frac{2}{3} \sqrt{x^2-9}$
521.2) Установить, какие линии определяются следующими уравнениями: $y=-3 \sqrt{x^2+1}$;
521.3) Установить, какие линии определяются следующими уравнениями: $x=-\frac{4}{3} \sqrt{y^2+9} ;$
521.4) Установить, какие линии определяются следующими уравнениями: $y=+\frac{2}{5} \sqrt{x^2+25}$
522. Дана точка $M_1(10 ;-\sqrt{5})$ на гиперболе $\frac{x^2}{80}-$ $-\frac{y^2}{20}=1$. Составить уравнения прямых, на которых лежат фокальные радиусы точки $M_1$.
523. Убедившись, что точка $M_1\left(-5 ; \frac{9}{4}\right)$ лежит на гиперболе $\frac{x^2}{16}-\frac{y^2}{9}=1$, определить фокальные радиусы точки $M_1$.
524. Эксцентриситет гиперболы $\varepsilon=2$, фокальный радиус ее точки $M$, проведенный из некоторого фокуса, равен 16. Вычислить расстояние от точки $M$ до односто* ронней с этим фокусом директрисы.
525. Эксцентриситет гиперболы $\varepsilon=3$, расстояние от точки. $M$ гиперболы до директрисы равно 4 . Вычислить расстояние от точки $M$ до фокуса, одностороннего с этой директрисой.
526. Эксцентриситет гиперболы $\varepsilon=2$, центр ее лежит в начале координат, один из фокусов $F(12 ; 0)$. Вычислить расстояние от точки $M_1$ гиперболы с абсциссой, равной 13 , до директрисы, соответствующей заданному фокусу.
527. Эксцентриситет гиперболы $\varepsilon=\frac{3}{2}$, центр ее лежит в начале координат, одна из директрис дана уравнением $x=-8$. Вычислить расстояние от точки $M_1$ гиперболы с абсциссой, равной 10 , до фокуса, соответствующего заданной директрисе.
528. Определить точки гиперболы $\frac{x^2}{64}-\frac{y^2}{36}=1$, расстояние которых до правого фокуса равно 4,5 .
529. Определить точки гиперболы $\frac{x^2}{9}-\frac{y^2}{16}=1$, расстояние которых до левого фокуса равно 7.
530. Через левый фокус гиперболы $\frac{x^2}{144}-\frac{y^2}{25}=1$ проведен перпендикуляр к ее оси, содержащей вершины. Определить расстояния от фокусов до точек пересечения этого перпендикуляра с гиперболой.
531. Пользуясь одним циркулем, построить фокусы гиперболы $\frac{x^2}{16}-\frac{y^2}{25}=1$ (считая, что оси координат изображены и масштабная единица задана).
532.1) Составить уравнение гиперболы, фокусы которой лежат на оси абсцисс симметрично относительно начала координат, если даны: точки $M_1(6 ;-1)$ и $M_2(-8 ; 2 \sqrt{2})$ гиперболы;
532.2) Составить уравнение гиперболы, фокусы которой лежат на оси абсцисс симметрично относительно начала координат, если даны: точка $M_1(-5 ; 3)$ гиперболы и эксцентриситет $\varepsilon=\sqrt{2}$;
532.3) Составить уравнение гиперболы, фокусы которой лежат на оси абсцисс симметрично относительно начала координат, если даны: точка $M_1\left(\frac{9}{2} ;-1\right)$ гиперболы и уравнения асимптот $y= \pm \frac{2}{3} x$;
532.4) Составить уравнение гиперболы, фокусы которой лежат на оси абсцисс симметрично относительно начала координат, если даны: точка $M_1\left(-3 ; \frac{5}{2}\right)$ гиперболы и уравнения директрис $x= \pm \frac{4}{3}$;
532.5) Составить уравнение гиперболы, фокусы которой лежат на оси абсцисс симметрично относительно начала координат, если даны: уравнения асимптот $y= \pm \frac{3}{4} x$ и уравнения директрис $x= \pm \frac{16}{5}$.
533. Определить эксцентриситет равносторонней гиперболы.
534. Определить эксцентриситет гиперболы, если отрезок между ее вершинами виден из фокусов сопряженной гиперболы под углом в $60^{\circ}$.
535. Фокусы гиперболы совпадают с фокусами эллипса $\frac{x^2}{25}+\frac{y^2}{9}=1$. Составить уравнение гиперболы, если ее эксцентриситет $\varepsilon=2$.
536. Составить уравнение гиперболы, фокусы которой лежат в вершинах эллинса $\frac{x^2}{100}+\frac{y^2}{64}=1$, а директрисы проходят через фокусы этого эллипса.
537. Доказать, что расстояние от фокуса гиперболы $\frac{x^2}{a^2}-\frac{y^2}{b^2}=1$ до ее асимптоты равно $b$.
538. Доказать что произведение расстояний от любой точки гиперболы $\frac{x^2}{a^2}-\frac{y^2}{b^2}=1$ до двух ее асимптот есть величина постоянная, равная $\frac{a^2 b^2}{a^2+b^2}$.
539. Доказать, что площадь параллелограмма, ограниченного асимптотами гиперболы $\frac{x^2}{a^2}-\frac{y^2}{b^2}=1$ и прямыми, проведенными через любую ее точку параллельно асимптотам, есть величина постоянная, равная $\frac{a b}{2}$.
540. Составить уравнение гиперболы, если известны ее полуоси $a$ и $b$, центр $C\left(x_0 ; y_0\right)$ и фокусы расположены на прямой: 1) параллельной оси $O x$; 2) параллельной оси $O y$.
541.1) Установить, что каждое из следующих уравнений определяет гиперболу, и найти координаты еє центра $C$, полуоси, эксцентриситет, уравнения асимпь тот и уравнения директрис: $16 x^2-9 y^2-64 x-54 y-161=0$;
541.2) Установить, что каждое из следующих уравнений определяет гиперболу, и найти координаты еє центра $C$, полуоси, эксцентриситет, уравнения асимпь тот и уравнения директрис: $9 x^2-16 y^2+90 x+32 y-367=0 ;$
541.3) Установить, что каждое из следующих уравнений определяет гиперболу, и найти координаты еє центра $C$, полуоси, эксцентриситет, уравнения асимпь тот и уравнения директрис: $16 x^2-9 y^2-64 x-18 y+199=0$
542.1) Установить, какие линии определяются следующими уравнениями: $y=-1+\frac{2}{3} \sqrt{x^2-4 x-5}$;
542.2) Установить, какие линии определяются следующими уравнениями: $y=7-\frac{3}{2} \sqrt{x^2-6 x+13}$;
542.3) Установить, какие линии определяются следующими уравнениями: $x=9-2 \sqrt{y^2+4 y+8}$;
542.4) Установить, какие линии определяются следующими уравнениями: $x=5-\frac{3}{4} \sqrt{y^2+4 y-12}$.
543.1) Составить уравнение гиперболы, зная, что: расстояние между ее вершинами равно 24 и фокусы суть $F_1(-10 ; 2), F_2(16 ; 2)$;
543.2) Составить уравнение гиперболы, зная, что: фокусы суть $F_1(3 ; 4), F_2(-3 ;-4)$ и расстояние между директрисами равно 3,6 ;
543.3) Составить уравнение гиперболы, зная, что: угол между асимптотами равен $90^{\circ}$ и фокусы суть $F_1(4 ;-4), F_2(-2 ; 2)$.
544. Составить уравнение гиперболы, если известны ее эксцентриситет $\varepsilon=\frac{5}{4}$, фокус $F(5 ; 0)$ и уравнение состветствующей директрисы $5 x-16=0$.
545. Составить уравнение гиперболы, если известны ее эксцентриситет $\varepsilon=\frac{13}{12}$, фокус $F(0 ; 13)$ и уравнение соответствующей директрисы $13 y-144=0$.
546. Точка $A(-3 ;-5)$ лежит на гиперболе, фокус которой $F(-2 ;-3)$, а соответствующая директриса дана уравнением $x+1=0$. Составить уравнение этой гиперболы.
547. Составить уравнение гиперболы, если известны ее эксцентриситет $\varepsilon=\sqrt{5}$, фокус $F(2 ;-3)$ и уравнение соответствующей директрисы $3 x-y+3=0$.
548. Точка $M_1(1 ;-2)$ лежит на гиперболе, фокус которой $F(-2 ; 2)$, а соответствующая директриса дана уравнением $2 x-y-1=0$. Составить уравнение этой гиперболы.
549. Дано уравнение равносторонней гиперболы $x^2-y^2=a^2$. Найти ее уравнение в новой системе, приняв за оси координат ее асимптоты.
550. Установив, что каждое из следующих уравнений определяет гиперболу, найти для каждой из них центр, полуоси, уравнения асимптот и построить их на чертеже: 1) $x y=18$; 2) $2 x y-9=0$; 3) $2 x y+25=0$
551. Найти точки пересечения прямой $2 x-y$ -$-10=0$ и гиперболы $\frac{x^2}{20}-\frac{y^2}{5}=1$.
552. Найти точки пересечения прямой $4 x-3 y \rightarrow$ $-16=0$ и гиперболы $\frac{x^2}{25}-\frac{y^2}{16}=1$.
553. Найти точки пересечения прямой $2 x-y$ "† $+1=0$ и гиперболы $\frac{x^2}{9}-\frac{y^2}{4}=1$.
554.1) В следующих случаях определить, как расположена прямая относительно гиперболы: пересекает ли, касается или проходит вне ее: $x-y-3=0, \quad \frac{x^2}{12}-\frac{y^2}{3}=1$;
554.2) В следующих случаях определить, как расположена прямая относительно гиперболы: пересекает ли, касается или проходит вне ее: $x-2 y+1=0, \quad \frac{x^2}{16}-\frac{y^2}{9}=1$;
554.3) В следующих случаях определить, как расположена прямая относительно гиперболы: пересекает ли, касается или проходит вне ее: $7 x-5 y=0, \quad \frac{x^2}{25}-\frac{y^2}{16}=1$.
555.1) Определить, при каких значениях $m$ прямая $y=\frac{5}{2} x+m$ пересекает гиперболу $\frac{x^2}{9}-\frac{y^2}{36}=1$; 2) касается ее;
555.2) Определить, при каких значениях $m$ прямая $y=\frac{5}{2} x+m$ проходит вне этой гиперболы.
556. Вывести условие, при котором прямая $y=k x+m$ касается гиперболы $\frac{x^2}{a^2}-\frac{y^2}{b^2}=1$.
557. Составить уравнение касательной к гиперболе $\frac{x^2}{a^2}-\frac{y^2}{b^2}=1$ в ее точке $M_1\left(x_1 ; y_1\right)$.
558. Доказать, что касательные к гиперболе, проведенные в концах одного и того же диаметра, параллельны.
559. Составить уравнения касательных к гиперболе $\frac{x^2}{20}-\frac{y^2}{5}=1$, перпендикулярных к прямой $4 x+3 y-$ $-7=0$.
560. Составить уравнения касательных к гиперболе $\frac{x^2}{16}-\frac{y^2}{64}=1$, параллельных прямой $10 x-3 y+9=0$.
561. Провести касательные к гиперболе $\frac{x^2}{16}-\frac{y^2}{8}=$ $=-1$ параллельно прямой $2 x+4 y-5=0$ и вычис лить расстояние $d$ между ними.
562. На гиперболе $\frac{x^2}{24}-\frac{y^2}{18}=1$ найти точку $M_1$, ближайшую к прямой $3 x+2 y+1=0$, и вычислить расстояние $d$ от точки $M_1$ до этой прямой.
563. Составить уравнение касательных к гиперболе $x^2-y^2=16$, проведенных из точки $A(-1 ;-7)$.
564. Из точки $C(1 ;-10)$ проведены касательные к гиперболе $\frac{x^2}{8}-\frac{y^2}{32}=1$. Составить уравнение хорды, соединяющей точки касания.
565. Из точки $P(1 ;-5)$ проведены касательные к гиперболе $\frac{x^2}{3}-\frac{y^2}{5}=1$. Вычислить расстолние $d$ от точки $P$ до хорды гиперболы, соединяющей точки касания.
566. Гипербола проходит через точку $A(\sqrt{6} ; 3)$ и касается прямой $9 x+2 y-15=0$. Составить уравнение этой гиперболы при условии, что ее оси совпадают с осями координат.
567. Составить уравнение гиперболы, касающейся двух прямых: $\quad 5 x-6 y-16=0, \quad 13 x-10 y-48=0$, при условии, что ее оси совпадают с осями координат.
568. Убедившись, что точки пересечения эллипса $\frac{x^2}{20}+\frac{y^2}{5}=1$ и гиперболы $\frac{x^2}{12}-\frac{y^2}{3}=1$ являются вершинами прямоугольника, составить уравнения его сторон.
569. Даны гиперболы $\frac{x^2}{a^2}-\frac{y^2}{b^2}=1$ и какая-нибудь ее касательная: $P$-точка пересечения касательной с осью $O x, Q$ - проекция точки касания на ту же ось. Доказать, что $O P \cdot O Q=a^2$.
