1153. Установить, что плоскость $x-2=0$ пересекает эллипсоид $\frac{x^2}{16}+\frac{y^2}{12}+\frac{z^2}{4}=1$ по эллипсу; найти его полуоси и вершины.
1154. Установить, что плоскость $z+1=0$ пересекает однополостный гиперболоид $\frac{x^2}{32}-\frac{y^2}{18}+\frac{z^2}{2}=1$ по гиперболе; найти ее полуоси и вершины.
1155. Установить, что плоскость $y+6=0$ пересекает гиперболический параболоид $\frac{x^2}{5}-\frac{y^2}{4}=6 z$ по параболе; найти ее параметр и вершину.
1156. Найти уравнения проекций на координатные плоскости сечения эллиптического параболоида $y^2+z^2=x$ плоскостью $x+2 y-z=0$
1157. Установить, какая линия является сечением эллипсоида $\frac{x^2}{12}+\frac{y^2}{4}+\frac{z^2}{3}=1$ плоскостью $2 x-3 y+4 z-$ $-11=0$, и найти ее центр.
1158. Установить, какая линия является сечением гиперболического параболоида $\frac{x^2}{2}-\frac{z^2}{3}=y$ плоскостью $3 x-3 y+4 z+2=0$, и найти ее центр.
1159.1) Установить, какие линии определяются следующими уравнениями: $\left\{\begin{array}{l}\frac{x^2}{3}+\frac{y^2}{6}=2 z, \\ 3 x-y+6 z-14=0\end{array}\right.$
1159.2) Установить, какие линии определяются следующими уравнениями: $\left\{\begin{array}{l}\frac{x^2}{4}-\frac{y^2}{3}=2 z \\ x-2 y+2=0 ;\end{array}\right.$
1159.3) Установить, какие линии определяются следующими уравнениями: $\left\{\begin{array}{l}\frac{x^2}{.4}+\frac{y^2}{9}-\frac{z^2}{36}=1, \\ 9 x-6 y+2 z-28=0,\end{array}\right.$
1160. Установить, при каких значениях $m$ плоскость $x+m z-1=0$ пересекает двухполостный гиперболоид $x^2+y^2-z^2=-1$ а) по эллипсу, б) по гиперболе.
1161. Установить, при каких значениях $m$ плоскость $x+m y-2=0$ пересекает эллиптический параболоид $\frac{x^2}{2}+\frac{z^2}{3}=y$ а) по эллипсу, б) по параболе.
1162. Доказать, что эллиптический параболоид $\frac{x^2}{9}+\frac{z^2}{4}=2 y$ имеет одну общую точку с плоскостью $2 x-2 y-z-10=0$, и найти ее координаты.
1163. Доказать, что двухполостный гиперболоид $\frac{x^2}{3}+\frac{y^2}{4}-\frac{z^2}{25}=-1$ имеет одну общую точку с плоскостью $5 x+2 z+5=0$, и найти ее координаты.
1164. Доказать, что эллипсоид $\frac{x^2}{81}+\frac{y^2}{36}+\frac{z^2}{9}=1$ имеет одну общую точку с плоскостью $4 x-3 y+12 z-54=0$, и найти ее координаты.
1165. Определить, при каком значении $m$ плоскость $x-2 y-2 z+m=0$ касается эллипсоида $\frac{x^2}{144}+\frac{y^2}{36}+$ $+\frac{z^2}{9}=1$.
