1160. Установить, при каких значениях $m$ плоскость $x+m z-1=0$ пересекает двухполостный гиперболоид $x^2+y^2-z^2=-1$ а) по эллипсу, б) по гиперболе.
1161. Установить, при каких значениях $m$ плоскость $x+m y-2=0$ пересекает эллиптический параболоид $\frac{x^2}{2}+\frac{z^2}{3}=y$ а) по эллипсу, б) по параболе.
1162. Доказать, что эллиптический параболоид $\frac{x^2}{9}+\frac{z^2}{4}=2 y$ имеет одну общую точку с плоскостью $2 x-2 y-z-10=0$, и найти ее координаты.
1163. Доказать, что двухполостный гиперболоид $\frac{x^2}{3}+\frac{y^2}{4}-\frac{z^2}{25}=-1$ имеет одну общую точку с плоскостью $5 x+2 z+5=0$, и найти ее координаты.
1164. Доказать, что эллипсоид $\frac{x^2}{81}+\frac{y^2}{36}+\frac{z^2}{9}=1$ имеет одну общую точку с плоскостью $4 x-3 y+12 z-54=0$, и найти ее координаты.
1165. Определить, при каком значении $m$ плоскость $x-2 y-2 z+m=0$ касается эллипсоида $\frac{x^2}{144}+\frac{y^2}{36}+\frac{z^2}{9}=1$.