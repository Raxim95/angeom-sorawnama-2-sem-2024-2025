665.1) Установить, какие из следующих линий являются центральными (т.е. имеют единственный центр), какие имеют центра, какие имеют бесконечно много центров: $3 x^2-4 x y-2 y^2+3 x-12 y-7=0$;
665.2) Установить, какие из следующих линий являются центральными (т.е. имеют единственный центр), какие имеют центра, какие имеют бесконечно много центров: $4 x^2+5 x y+3 y^2-x+9 y-12=0$;
665.3) Установить, какие из следующих линий являются центральными (т.е. имеют единственный центр), какие имеют центра, какие имеют бесконечно много центров: $4 x^2-4 x y+y^2-6 x+8 y+13=0$;
665.4) Установить, какие из следующих линий являются центральными (т.е. имеют единственный центр), какие имеют центра, какие имеют бесконечно много центров: $4 x^2-4 x y+y^2-12 x+6 y-11=0$;
665.5) Установить, какие из следующих линий являются центральными (т.е. имеют единственный центр), какие имеют центра, какие имеют бесконечно много центров: $x^2-2 x y+4 y^2+5 x-7 y+12=0$;
665.6) Установить, какие из следующих линий являются центральными (т.е. имеют единственный центр), какие имеют центра, какие имеют бесконечно много центров: $x^2-2 x y+y^2-6 x+6 y-3=0$;
665.7) Установить, какие из следующих линий являются центральными (т.е. имеют единственный центр), какие имеют центра, какие имеют бесконечно много центров: $4 x^2-20 x y+25 y^2-14 x+2 y-15=0$;
665.8) Установить, какие из следующих линий являются центральными (т.е. имеют единственный центр), какие имеют центра, какие имеют бесконечно много центров: $4 x^2-6 x y-9 y^2+3 x-7 y+12=0$.
666.1) Установить, что следующие линии являются центральными, и для каждой из них найти координаты центра: $3 x^2+5 x y+y^2-8 x-11 y-7=0$;
666.2) Установить, что следующие линии являются центральными, и для каждой из них найти координаты центра: $5 x^2+4 x y+2 y^2+20 x+20 y-18=0$;
666.3) Установить, что следующие линии являются центральными, и для каждой из них найти координаты центра: $9 x^2-4 x y-7 y^2-12=0$;
666.4) Установить, что следующие линии являются центральными, и для каждой из них найти координаты центра: $2 x^2-6 x y+5 y^2+22 x-36 y+11=0$.
667.1) Установить, что каждая из следующих линий имєет бесконечно много цєнтров; для каждой их них составить уравнение геометрического места центров: $x^2-6 x y+9 y^2-12 x+36 y+20=0$;
667.2) Установить, что каждая из следующих линий имєет бесконечно много цєнтров; для каждой их них составить уравнение геометрического места центров: $4 x^2+4 x y+y^2-8 x-4 y-21=0$;
667.3) Установить, что каждая из следующих линий имєет бесконечно много цєнтров; для каждой их них составить уравнение геометрического места центров: $25 x^2-10 x y+y^2+40 x-8 y+7=0$.
668.1) Установить, что следующие уравнения определяют центральные линии; преобразовать каждое из них путем переноса начала координат в центр: $3 x^2-6 x y+2 y^2-4 x+2 y+1=0$;
668.2) Установить, что следующие уравнения определяют центральные линии; преобразовать каждое из них путем переноса начала координат в центр: $6 x^2+4 x y+y^2+4 x-2 y+2=0$;
668.3) Установить, что следующие уравнения определяют центральные линии; преобразовать каждое из них путем переноса начала координат в центр: $4 x^2+6 x y+y^2-10 x-10=0$;
668.4) Установить, что следующие уравнения определяют центральные линии; преобразовать каждое из них путем переноса начала координат в центр: $4 x^2+2 x y+6 y^2+6 x-10 y+9=0$.
669. При каких значениях $m$ и $n$ уравнение $m x^2+12 x y+9 y^2+4 x+n y-13=0$ определяет: 1) центральную линию; 2) линию без центра; 3) линию, имеющую бесконечно много центров.
670. Дано уравнение линии $4 x^2-4 x y+y^2+6 x+$ $+1=0$. Определить, при каких значениях углового коэффициента $k$ прямая $y=k x:$ 1) пересекает эту линию в одной точке; 2) касается этой линии; 3) пересекает эту линию в двух точках; 4) не имеет общих точек с этой линией.






