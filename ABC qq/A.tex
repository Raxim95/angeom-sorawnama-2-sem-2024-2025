\section{ Ellipstiń kanonikalıq teńlemesi}

\textbf{444.1)} Fokusları abscissa kósherinde jatqan hám koordinatalar basına salıstırǵanda simmetriyalı, yarım kósherleri 5 hám 2 bolǵan ellipstiń teńlemesin dúziń;

\textbf{444.2)} Fokusları abscissa kósherinde jatqan hám koordinatalar basına salıstırģanda simmetriyalı bolǵan ellipstiń teńlemesin dúziń, bunda onıń úlken kósheri 10 ģa, fokusları arasındaǵı aralıq bolsa $2c = 8$ ge teń;

\textbf{444.3)} Fokusları abscissa kósherinde jatqan hám koordinatalar basına salıstırģanda simmetriyalı bolǵan ellipstiń teńlemesin dúziń, bunda onıń kishi kósheri 24 ke, fokusları arasındaǵı aralıq bolsa $c = 10$ ga teń;

\textbf{444.4)} Fokusları abscissa kósherinde jatqan hám koordinatalar basına salıstırģanda simmetriyalı bolǵan ellipstiń teńlemesin dúziń, bunda: fokusları arasındaǵı aralıq $2 c=6$ hám ekssentrisiteti $\varepsilon=\frac{3}{5}$;

\textbf{444.5)} Fokusları abscissa kósherinde jatqan hám koordinatalar basına salıstırģanda simmetriyalı bolǵan ellipstiń teńlemesin dúziń, bunda: úlken kósheri 20, ekscentrisiteti $\varepsilon=\frac{3}{5}$;

\textbf{444.6)} Fokusları abscissa kósherinde jatqan hám koordinatalar basına salıstırganda simmetriyalı bolǵan ellipstiń teńlemesin dúziń, bunda: kishi kósheri 10, ekscentrisiteti $\varepsilon=\frac{12}{13}$;

\textbf{444.7)} Fokusları abscissa kósherinde jatqan hám koordinatalar basına salıstırģanda simmetriyalı bolǵan ellipstiń teńlemesin dúziń, bunda: direktrisaları arasındaǵı aralıq 5 hám fokusları arasındaǵı aralıq $2c=4$;

\textbf{444.8)} Fokusları abscissa kósherinde jatqan hám koordinatalar basına salıstırģanda simmetriyalı bolǵan ellipstiń teńlemesin dúziń, bunda: úlken kósheri 8, direktrisaları arasındaǵı aralıq 16;

\textbf{444.9)} Fokusları abscissa kósherinde jatqan hám koordinatalar basına salıstırģanda simmetriyalı bolǵan ellipstiń teńlemesin dúziń, bunda: kishi kósheri 6, direktrisaları arasındaǵı aralıq 13;

\textbf{444.9)} Fokusları abscissa kósherinde jatqan hám koordinatalar basına salıstırģanda simmetriyalı bolǵan ellipstiń teńlemesin dúziń, bunda: direktrisaları arasındaǵı aralıq 32 hám $\varepsilon=\frac{1}{2}$.

\textbf{445.1)} Fokusları ordinata kósherinde jatqan hám koordinatalar basına salıstırģanda simmetriyalı bolǵan ellipstiń teńlemesin dúziń, bunda: yarım kósherleri 7 hám 2;

\textbf{445.2)} Fokusları ordinata kósherinde jatqan hám koordinatalar basına salıstırģanda simmetriyalı bolǵan ellipstiń teńlemesin dúziń, bunda: úlken yarım kósheri 10, fokusları arasındaǵı aralıq $2 c=8$;

\textbf{445.3)} Fokusları ordinata kósherinde jatqan hám koordinatalar basına salıstırģanda simmetriyalı bolǵan ellipstiń teńlemesin dúziń, bunda: fokusları arasındaǵı aralıq $2 c=24$, ekssentrisiteti $\varepsilon=\frac{12}{13}$;

\textbf{445.4)} Fokusları abscissa kósherinde jatqan hám koordinatalar basına salıstırģanda simmetriyalı bolǵan ellipstiń teńlemesin dúziń, bunda: direktrisaları arasındaǵı aralıq 32 hám $\varepsilon=\frac{1}{2}$.

\textbf{445.4)} Fokusları ordinata kósherinde jatqan hám koordinatalar basına qarata simmetriyalı bolǵan ellipstiń teńlemesin dúziń, bunda: kishi kósheri 16, a ekssentrisiteti $\varepsilon=\frac{3}{5}$;

\textbf{445.5)} Fokusları ordinata kósherinde jatqan hám koordinatalar basına qarata simmetriyalı bolǵan ellipstiń teńlemesin dúziń, bunda: fokusları arasındaǵı aralıq $2 c=6$, direktrisaları arasındaǵı aralıq $16 \frac{2}{3}$;

\textbf{445.6)} Fokusları ordinata kósherinde jatqan hám koordinatalar basına salıstırģanda simmetriyalı bolǵan ellipstiń teńlemesin dúziń, bunda: direktrisaları arasındaģı qashıqlıq $frac{2}{3}$ hám ekssentrisiteti $frac{3}{4}$.

\textbf{450.} Eki tóbesi $9 x^2+5 y^2=1$ ellipstiń fokuslarında, qalǵan eki tóbesi onıń kishi kósheriniń tóbelerinde jaylasqan tórtmúyeshliktiń maydanın esaplań.

\textbf{462.} $\frac{x^2}{100}+\frac{y^2}{36}=1$ ellipsinde jaylasqan hám oń fokusına shekemgi aralıǵı 14 ke teń noqattı tabıń.

\textbf{463.} $\frac{x^2}{16}+\frac{y^2}{7}=1$, ellipsinde jaylasqan hám shep fokusına shekemgi aralıǵı 2,5 ge teń noqattı tabıń.

\textbf{463.1)} Fokusları abscissa kósherinde jatqan hám koordinatalar basına qarata simmetriyalı bolǵan ellipstiń teńlemesin dúziń, bunda: $M_1 (-2 \sqrt{5}; 2) $ noqatı ellipske tiyisli hám kishi yarım kósheri $b=3$;

\textbf{463.2)} Fokusları abscissa kósherinde jatqan hám koordinatalar basına qarata simmetriyalı bolǵan ellipstiń teńlemesi dúzilsin, bunda: $M_1 (2;-2) $ noqatı ellipske tiyisli hám úlken yarım kósheri $a=4$;

\textbf{463.3)} Fokusları abscissa kósherinde jatqan hám koordinatalar basına qarata simmetriyalı bolǵan ellipstiń teńlemesin dúziń, bunda: $M_1 (4;-\sqrt{3}) $ hám $M_2 (2 \sqrt{2}; 3)$ noqatları ellipske tiyisli;

\textbf{463.4)} Fokusları abscissa kósherinde jatqan hám koordinatalar basına qarata simmetriyalı bolǵan ellipstiń teńlemesin dúziń, bunda: $M_1 (\sqrt{15};-1) $ noqatı ellipske tiyisli hám fokusları arasındaǵı aralıq $2 c=8$;

\textbf{463.5)} Fokusları abscissa kósherinde jatqan hám koordinatalar basına qarata simmetriyalı bolǵan ellipstiń teńlemesi dúzilsin, bunda: $M_1 \left(2;-\frac{5}{3}\right) $ noqatı ellipske tiyisli hám ekscentrisiteti $\varepsilon=\frac{2}{3}$;

\textbf{463.6)} Fokusları abscissa kósherinde jatqan hám koordinatalar basına qarata simmetriyalı bolǵan ellipstiń teńlemesin dúziń, bunda: $M_1 (8; 12) $ ellipske tiyisli hám bul noqattan shep fokusına shekemgi aralıq $r_1=20$ qa teń;

\textbf{463.7)} Fokusları abscissa kósherinde jatqan hám koordinatalar basına qarata simmetriyalı bolǵan ellipstiń teńlemesin dúziń, bunda: $M_1 (-\sqrt{5}; 2)$ noqatı ellipske tiyisli hám onıń direktrisaları arasındaǵı aralıq 10 ģa teń.

\textbf {470.} $C (-3; 2)$ eki koordinata kósherine urınıwshı ellipstiń orayı. Bul ellipstiń simmetriya kósherleri koordinata kósherlerine parallel ekenligin bilgen halda onıń teńlemesin dúziń.

\textbf {474.} Ekscentrisiteti $\varepsilon=\frac{2}{3}$, fokusı $F (2; 1) $ hám usı fokus tárepindegi direktrisası $x-5=0$ bolǵan ellipstiń teńlemesin dúziń.

\textbf {475.} Ekscentrisiteti $\varepsilon=\frac{1}{2}$, fokusı $F (-4; 1) $ hám usı fokus tárepindegi direktrisası $y+3=0$ bolǵan ellipstiń teńlemesin dúziń.

\textbf {476.} $A (-3;-5) $ noqatı fokusı $F (-1;-4) $ bolǵan ellipste jatadı hám oǵan sáykes direktrisa $x-2=0$ teńlemesi menen berilgen. Usi ellipstiń teńlemesin dúziń.



\section{Giperbolanıń kanonikalıq teńlemesi}



\textbf{515.1)} Fokusları abscissa kósherinde jaylasqan, koordinatalar basına salıstırģanda simmetriyalı bolǵan giperbolanıń teńlemesin dúziń, bunda: onıń kósherleri $2 a=10$ hám $2 b=8$;

\textbf{515.2)} Fokusları abscissa kósherinde jaylasqan, koordinatalar basına qarata simmetriyalı bolǵan giperbolanıń teńlemesin dúziń, bunda: fokusları arasındaǵı aralıq $2 c=10$ hám kishi kósheri $2 b=8$;

\textbf{515.3)} Fokusları abscissa kósherinde jaylasqan, koordinatalar basına qarata simmetriyalı bolǵan giperbolanıń teńlemesin dúziń, bunda: fokusları arasındaǵı aralıq $2 c=6$ hám ekscentrisiteti $\varepsilon=\frac{3}{2}$;

\textbf{515.4)}  Fokusları abscissa kósherinde jaylasqan, koordinatalar basına qarata simmetriyalı bolǵan giperbolanıń teńlemesin dúziń, bunda: $2 a=16$ hám ekssentrisiteti $\varepsilon=\frac{5}{4}$;

\textbf{515.5)} Fokusları abscissa kósherinde jaylasqan, koordinatalar basına salıstırģanda simmetriyalı bolǵan giperbolanıń teńlemesin dúziń, bunda: asimtotalarınıń teńlemesi $y= \pm \frac{4}{3} x$ hám fokusları arasındaǵı aralıq $2 c=20$;

\textbf{515.6)} Fokusları abscissa kósherinde jaylasqan, koordinatalar basına salıstırģanda simmetriyalı bolǵan giperbolanıń teńlemesin dúziń, bunda: direktrisalarınıń arasındaģı aralıq $22 \frac{2}{13}$ hám fokusları arasındaǵı aralıq $2 c=26$;

\textbf{515.7)} Fokusları abscissa kósherinde jaylasqan, koordinatalar basına salıstırģanda simmetriyalı bolǵan giperbolanıń teńlemesin dúziń, bunda: direktrisalarınıń arasındaģı aralıq $\frac{32}{5}$ ham kishi kósheri $2 b=6$;

\textbf{515.8)} Fokusları abscissa kósherinde jaylasqan, koordinatalar basına salıstırģanda simmetriyalı bolǵan giperbolanıń teńlemesin dúziń, bunda: direktrisalarınıń arasındaģı aralıq $\frac{8}{3}$ hám ekssentrisiteti $\varepsilon=\frac{3}{2}$;

\textbf{515.9)} Fokusları abscissa kósherinde jaylasqan, koordinatalar basına salıstırģanda simmetriyalı bolǵan giperbolanıń teńlemesin dúziń, bunda: asimtotalarınıń teńlemesi $y= \pm \frac{3}{4} x$ hám direktrisalarınıń arasındaģı aralıq $12 \frac{4}{5}$.

\textbf{516.1)} Fokusları ordinata kósherinde, koordinatalar basına salıstırǵanda simmetriyalı jaylasqan giperbolanıń teńlemesi dúzilsin, bunda: onıń yarım kósherleri. $a=6, b=18$;

\textbf{516.2)} Fokusları ordinata kósherinde jaylasqan, koordinatalar basına salıstırģanda simmetriyalı bolǵan giperbolanıń teńlemesin dúziń, bunda: fokusları arasındaģı aralıq $2 c=10$ hám ekssentrisiteti $\varepsilon=\frac{5}{3}$;

\textbf{516.3)} Fokusları ordinata kósherinde jaylasqan, koordinatalar basına salıstırģanda simmetriyalı bolǵan giperbolanıń teńlemesin dúziń, bunda: asimtotalarınıń teńlemesi $y= \pm \frac{12}{5} x$ hám ushlarınıń arasındaģı aralıq 48;

\textbf{516.4)} Fokusları ordinata kósherinde jaylasqan, koordinatalar basına salıstırģanda simmetriyalı bolǵan giperbolanıń teńlemesin dúziń, bunda: asimtotalarınıń teńlemesi $7 \frac{1}{7}$ hám ekssentrisiteti $\varepsilon=\frac{7}{5}$;

\textbf{516.5)} Fokusları ordinata kósherinde jaylasqan, koordinatalar basına salıstırģanda simmetriyalı bolǵan giperbolanıń teńlemesin dúziń, bunda: asimtotalarınıń teńlemesi $y= \pm \frac{4}{3} x$ ham direktrisalarınıń arasındaģı aralıq $6 \frac{2}{5}$.

\textbf{518.} $16 x^2-9 y^2=144$ giperbola berilgen. Tabıń: 1) yarım kósherlerin; 2) fokusların; 3) ekssentrisitetin; 4) asimtotalarınıń teńlemesi; 5) direktrisaları tenlemelerin.

\textbf {520.} $\frac{x^2}{4}-\frac{y^2}{9}=1$ giperbolanıń asimptotalarınan hám $9 x+2 y-24=0$ tuwrı sızıqtan payda bolǵan úshmúyeshlik maydanın esaplań.

\textbf{521.1) } Berilgen teńleme qanday iymek sızıq ekenligin tabıń: $y=+\frac{2}{3} \sqrt{x^2-9}$

\textbf{521.2)} Berilgen teńleme qanday iymek sızıq ekenligin tabıń: $y=-3 \sqrt{x^2+1}$;

\textbf{521.3)} Berilgen teńleme qanday iymek sızıq ekenligin tabıń: $x=-\frac{4}{3} \sqrt{y^2+9} ;$

\textbf{521.4)} Berilgen teńleme qanday iymek sızıq ekenligin tabıń: $y=+\frac{2}{5} \sqrt{x^2+25}$

\textbf{524.} Giperbolanıń ekssentrisiteti $\varepsilon=2$ ǵa teń, $M$ noqatınıń bazı bir fokal radiusı 16 ǵa teń. $M$ noqattan sáykes direktrisaģa shekem bolǵan aralıqtı tabıń.

\textbf{525.} Giperbolanıń ekssentrisiteti $varepsilon=3$, $M$ noqatınıń bazı bir fokal radiusı 4 ke teń. $M$ noqattan sáykes direktrisaģa shekem bolǵan aralıqtı tabıń.

\textbf{526.} Fokusları abscissa kósherinde, koordinatalar basına salıstırģanda simmetriyalı jaylasqan giperbolanıń teńlemesin dúziń, bunda: $M_1 (6;-1) $ hám $M_2 (-8; 2 \sqrt{2}) noqatlar $ giperbolaga tiyisli;

\textbf{527.} Fokusları abscissa kósherinde, koordinatalar basına qarata simmetriyalı jaylasqan giperbolanıń teńlemesin dúziń, bunda: $M_1 (-5; 3)$ noqat giperbolaģa tiyisli hám ekssentrisiteti $\varepsilon=\sqrt{2}$;

\textbf{582.3)} Fokusları abscissa kósherinde, koordinatalar basına qarata simmetriyalı jaylasqan giperbolanıń teńlemesi dúzilsin, bunda: $M_1\left(\frac{9}{2};-1\right) $ noqatı giperbolaga tiyisli hám asimtota teńlemeleri $y= \pm \frac{2}{3} x$;

\textbf{582.4)} Fokusları abscissa kósherinde, koordinatalar basına qarata simmetriyalı jaylasqan giperbolanıń teńlemesin dúziń, bunda: $M_1\left(-3; \frac{5}{2}\right)$ noqatı giperbolaģa tiyisli hám direktrisalarınıń teńlemesi $x= \pm \frac{4}{3}$;

\textbf{582.4)} Fokusları abscissa kósherinde, koordinatalar basına qarata simmetriyalı jaylasqan giperbolanıń teńlemesi dúzilsin, bunda: asimptotanıń teńlemeleri $y= \pm \frac{3}{4} x$ hám direktrisalarınıń teńlemeleri $x= \pm \frac{16}{5}$.

\textbf{544.} Ekscentrisiteti $\varepsilon=\frac{5}{4}$, bir fokusı $F (5; 0) $ hám oǵan sáykes direktrisasınıń teńlemesi $5x-16=0$ bolǵan giperbolanıń teńlemesin dúziń.

\textbf{545.} Ekscentrisiteti $\varepsilon=\frac{13}{12}$, bir fokusi $F (0; 13) $ hám ogan sáykes direktrisasınıń teńlemesi $13 y-144=0$ bolǵ an giperbolanıń teńlemesin dúziń.


\section{Parabolanıń kanonikalıq teńlemesi}


\textbf{583.1)} Tóbesi koordinatalar basında bolǵan parabolanıń teńlemesin dúziń, bunda: parabola oń yarım tegislikte hám $Ox$ kósherine simmetriyalı jaylasqan, hám parametri $p=3$;

\textbf{583.2)} Tóbesi koordinatalar basında bolǵan parabolanıń teńlemesin dúziń, bunda: parabola oń yarım tegislikte hám $Ox$ kósherine simmetriyalı jaylasqan, hám parametri $p=0,5$;

\textbf{583.3} Tóbesi koordinatalar basında bolǵan parabolanıń teńlemesin dúziń, bunda: parabola oń yarım tegislikte hám $Oy$ kósherine simmetriyalı jaylasqan, hám parametri $p=\frac{1}{4}$;

\textbf{583.4)} Tóbesi koordinatalar basında bolǵan parabolanıń teńlemesin dúziń, bunda: parabola oń yarım tegislikte hám $Oy$ kósherine simmetriyalı jaylasqan, hám parametri $p=3$.

\textbf{585.1)} Tóbesi koordinatalar basında bolǵan parabolanıń teńlemesin dúziń, bunda: parabola $Ox$ kósherine simmetriyalı jaylasqan hám $A (9; 6) $ noqatınan ótedi;

\textbf{585.2)} Tóbesi koordinatalar basında bolǵan parabolanıń teńlemesin dúziń, bunda: parabola $Ox$ kósherine simmetriyalı jaylasqan hám $B (-1; 2) $ noqatınan ótedi;

\textbf{585.3)} Tóbesi koordinatalar basında bolǵan parabolanıń teńlemesin dúziń, bunda: parabola $Oy$ kósherine simmetriyalı jaylasqan hám $C (1; 1) $ noqatınan ótedi;

\textbf{585.4)} Tóbesi koordinatalar basında bolǵan parabolanıń teńlemesin dúziń, bunda: parabola $Oy$ kósherine simmetriyalı jaylasqan hám $D (4; -8) $ noqatınan ótedi;

\textbf{586.} Polat tros eki ushınan ildirilgen; bekkemlew noqatları birdey biyiklikte jaylasqan; olar arasındaǵı aralıq 20 m ge teń. Onıń bekkemleniw noqatınan 2 m aralıqtaǵı iyiliw shaması, gorizontal boyınsha esaplaǵanda, 14,4 sm ge teń. Trosttı shama menen parabola doǵası formasında dep esaplap, bekkemlew noqatları arasındaǵı bul trostıń iyiliw shamasın anıqlańız.

\textbf589. $y^2=24 x$ parabolanıń $F$ fokusın hám direktrisasınıń teńlemesin tabıń.

\textbf{590.} $M$ noqatı $y^2=20 x$ parabolaga tiyisli, eger onıń abscissası 7 ge teń bolsa fokal radiusların tabıń.

\textbf{594.1) } Parabolanıń tóbesi ($\alpha;\beta$) noqat penen ústpe-úst túsetuģının bilgen halda onıń teńlemesin dúziń. Parametri $p$ ǵa teń. Onıń kósheri $O x$ kósherine parallel bolıp, $O x$ kósheriniń oń baǵıtında sheksizlikke sozilgan;

\textbf{594.2)} Parabolanıń tóbesi ($\alpha;\beta$) noqat penen ústpe-úst túsetuģının bilgen halda onıń teńlemesin dúziń. Parametri $p$ ǵa teń. Onıń kósheri $O x$ kósherine parallel bolıp, $O x$ kósheriniń teris baǵıtında sheksizlikke sozilgan;.

\textbf{595.1)} Parabolanıń tóbesi ($\alpha;\beta$) noqat penen ústpe-úst túsetuģının bilgen halda onıń teńlemesin dúziń. Parametri $p$ ǵa teń. Onıń kósheri $O y$ kósherine parallel bolıp, $O y$ kósheriniń oń baǵıtında sheksizlikke sozilgan;

\textbf{595.2)} Parabolanıń tóbesi ($\alpha;\beta$) noqat penen ústpe-úst túsetuģının bilgen halda onıń teńlemesin dúziń. Parametri $p$ ǵa teń. Onıń kósheri $O y$ kósherine parallel bolıp, $O y$ kósheriniń teris baǵıtında sheksizlikke sozilgan;

\textbf{605.} $x+y-3=0$ tuwrı sızıģi hám $x^2=4 y$ parabolasınıń kesilisiw noqatın tabıń.

\textbf{617.} $y^2=36 x$ parabolanıń $A (2; 9) $ noqatındaǵı urınbasınıń teńlemesin dúziń.

\textbf{621.} $\frac{x^2}{100}+\frac{y^2}{225}=1$ ellips hám $y^2=24 x$ parabolanıń kesilisiw noqatların anıqlań.

\textbf{622.} $\frac{x^2}{20}-\frac{y^2}{5}=-1$ giperbola hám $y^2=3 x$ parabolanıń kesilisiw noqatların anıqlań.


\section{Ekinshi tártipli sızıq orayı}



\textbf{665.1) } Tómendegi sızıqlardan qaysı biri oraylıq (yaǵnıy birden-bir orayǵa iye), qaysı biri orayǵa iye emes, qaysı biri sheksiz kóp orayǵa iye ekenligin anıqlań: $3 x^2-4 x y-2 y^2+3 x-12 y-7=0$;

\textbf{665.2)} Tómendegi sızıqlardan qaysı biri oraylıq (yaǵnıy birden-bir orayǵa iye), qaysı biri orayǵa iye emes, qaysı biri sheksiz kóp orayǵa iye ekenligin anıqlań: $4 x^2+5 x y+3 y^2-x+9 y-12=0$;

\textbf{665.3)} Tómendegi sızıqlardan qaysı biri oraylıq (yaǵnıy birden-bir orayǵa iye), qaysı biri orayǵa iye emes, qaysı biri sheksiz kóp orayǵa iye ekenligin anıqlań: $4 x^2-4 x y+y^2-6 x+8 y+13=0$;

\textbf{665.4)} Tómendegi sızıqlardan qaysı biri oraylıq (yaǵnıy birden-bir orayǵa iye), qaysı biri orayǵa iye emes, qaysı biri sheksiz kóp orayǵa iye ekenligin anıqlań: $4 x^2-4 x y+y^2-12 x+6 y-11=0$;

\textbf{665.5)} Tómendegi sızıqlardan qaysı biri oraylıq (yaǵnıy birden-bir orayǵa iye), qaysı biri orayǵa iye emes, qaysı biri sheksiz kóp orayǵa iye ekenligin anıqlań:: $x^2-2 x y+4 y^2+5 x-7 y+12=0$;

\textbf{665.6)} Tómendegi sızıqlardan qaysı biri oraylıq (yaǵnıy birden-bir orayǵa iye), qaysı biri orayǵa iye emes, qaysı biri sheksiz kóp orayǵa iye ekenligin anıqlań:  $x^2-2 x y+y^2-6 x+6 y-3=0$;

\textbf{665.7)} Tómendegi sızıqlardan qaysı biri oraylıq (yaǵnıy birden-bir orayǵa iye), qaysı biri orayǵa iye emes, qaysı biri sheksiz kóp orayǵa iye ekenligin anıqlań: $4 x^2-20 x y+25 y^2-14 x+2 y-15=0$;

\textbf{665.8)} Tómendegi sızıqlardan qaysı biri oraylıq (yaǵnıy birden-bir orayǵa iye), qaysı biri orayǵa iye emes, qaysı biri sheksiz kóp orayǵa iye ekenligin anıqlań:  $4 x^2-6 x y-9 y^2+3 x-7 y+12=0$.

\textbf{667.1)} Tómendegi sızıqlardan qaysı biri oraylıq (yaǵnıy birden-bir orayǵa iye), qaysı biri orayǵa iye emes, qaysı biri sheksiz kóp orayǵa iye ekenligin anıqlań: $x^2-6 x y+9 y^2-12 x+36 y+20=0$;

\textbf{667.2)} Tómendegi sızıqlardan qaysı biri oraylıq (yaǵnıy birden-bir orayǵa iye), qaysı biri orayǵa iye emes, qaysı biri sheksiz kóp orayǵa iye ekenligin anıqlań:  $4 x^2+4 x y+y^2-8 x-4 y-21=0$;

\textbf{667.3)} Tómendegi sızıqlardan qaysı biri oraylıq (yaǵnıy birden-bir orayǵa iye), qaysı biri orayǵa iye emes, qaysı biri sheksiz kóp orayǵa iye ekenligin anıqlań: $25 x^2-10 x y+y^2+40 x-8 y+7=0$.



\section{Ekinshi tártipli oraylıq sızıq teńlemesin ápiwayı túrge keltiriw}



\textbf{675.1)} Diskriminantın esaplaw arqalı tómendegi teńlemelerdiń hár biriniń tipin anıqlań: $2 x^2+10 x y+12 y^2-7 x+18 y-15=0$;

\textbf{675.2)} Diskriminantın esaplaw arqalı tómendegi teńlemelerdiń hár biriniń tipin anıqlań: $3 x^2-8 x y+7 y^2+8 x-15 y+20=0$;

\textbf{675.3)} Diskriminantın esaplaw arqalı tómendegi teńlemelerdiń hár biriniń tipin anıqlań: $25 x^2-20 x y+4 y^2-12 x+20 y-17=0$;

\textbf{675.4)} Diskriminantın esaplaw arqalı tómendegi teńlemelerdiń hár biriniń tipin anıqlań: $5 x^2+14 x y+11 y^2+12 x-7 y+19=0$;

\textbf{675.5)} Diskriminantın esaplaw arqalı tómendegi teńlemelerdiń hár biriniń tipin anıqlań: $x^2-4 x y+4 y^2+7 x-12=0$;

\textbf{675.6)} Diskriminantın esaplaw arqalı tómendegi teńlemelerdiń hár biriniń tipin anıqlań: $3 x^2-2 x y-3 y^2+12 y-15=0$.

\textbf{678.1)} Koordinatalar sistemasın túrlendirmesten tómendegi teńlemelerdiń hár biri ellipsti anıqlawın kórsetiń hám onıń yarım kósherlerin tabıń: $41 x^2+24 x y+9 y^2+24 x+18 y-36=0$;

\textbf{678.2)} Koordinatalar sistemasın túrlendirmesten tómendegi teńlemelerdiń hár biri ellipsti anıqlawın kórsetiń hám onıń yarım kósherlerin tabıń: $8 x^2+4 x y+5 y^2+16 x+4 y-28=0$;

\textbf{678.3)} Koordinatalar sistemasın túrlendirmesten tómendegi teńlemelerdiń hár biri ellipsti anıqlawın kórsetiń hám onıń yarım kósherlerin tabıń: $13 x^2+18 x y+37 y^2-26 x-18 y+3=0$;

\textbf{678.4)} Koordinatalar sistemasın túrlendirmesten tómendegi teńlemelerdiń hár biri ellipsti anıqlawın kórsetiń hám onıń yarım kósherlerin tabıń: $13 x^2+10 x y+13 y^2+46 x+62 y+13=0$.

\textbf{679.1)} Koordinatalar sistemasın túrlendirmesten tómendegi teńlemelerdiń hár biri birden-bir noqattı anıqlawın kórsetiń hám onıń koordinataların tabıń: $5 x^2-6 x y+2 y^2-2 x+2=0$;

\textbf{679.2)} Koordinatalar sistemasın túrlendirmesten tómendegi teńlemelerdiń hár biri birden-bir noqattı anıqlawın kórsetiń hám onıń koordinataların tabıń: $x^2+2 x y+2 y^2+6 y+9=0$;

\textbf{679.3)} Koordinatalar sistemasın túrlendirmesten tómendegi teńlemelerdiń hár biri birden-bir noqattı anıqlawın kórsetiń hám onıń koordinataların tabıń: $5 x^2+4 x y+y^2-6 x-2 y+2=0$;

\textbf{679.4)} Koordinatalar sistemasın túrlendirmesten tómendegi teńlemelerdiń hár biri birden-bir noqattı anıqlawın kórsetiń hám onıń koordinataların tabıń: $x^2-6 x y+10 y^2+10 x-32 y+26=0$.

\textbf{680.1} Koordinatalar sistemasın túrlendirmesten tómendegi teńlemelerdiń hár biri giperbolanı anıqlawın kórsetiń hám onıń koordinataların tabıń: $4 x^2+24 x y+11 y^2+64 x+42 y+51=0$;

\textbf{680.2)} Koordinatalar sistemasın túrlendirmesten tómendegi teńlemelerdiń hár biri giperbolanı anıqlawın kórsetiń hám onıń koordinataların tabıń: $12 x^2+26 x y+12 y^2-52 x-48 y+73=0$

\textbf{680.3)} Koordinatalar sistemasın túrlendirmesten tómendegi teńlemelerdiń hár biri giperbolanı anıqlawın kórsetiń hám onıń koordinataların tabıń: $3 x^2+4 x y-12 x+16=0$;

\textbf{680.4)} Koordinatalar sistemasın túrlendirmesten tómendegi teńlemelerdiń hár biri giperbolanı anıqlawın kórsetiń hám onıń koordinataların tabıń: $x^2-6 x y-7 y^2+10 x-30 y+23=0$.

\textbf{681.1)} Koordinatalar sistemasın túrlendirmesten tómendegi teńlemelerdiń hár biri birden-bir noqattı anıqlawın kórsetiń hám onıń koordinataların tabıń: $3 x^2+4 x y+y^2-2 x-1=0$;

\textbf{681.2)} Koordinatalar sistemasın túrlendirmesten tómendegi teńlemelerdiń hár biri kesilisiwshi eki tuwrını anıqlawın kórsetiń hám onıń koordinataların tabıń: $x^2-6 x y+8 y^2-4 y-4=0$;

\textbf{681.3)} Koordinatalar sistemasın túrlendirmesten tómendegi teńlemelerdiń hár biri kesilisiwshi eki tuwrını anıqlawın kórsetiń hám onıń koordinataların tabıń: $x^2-4 x y+3 y^2=0$;

\textbf{681.4)} Koordinatalar sistemasın túrlendirmesten tómendegi teńlemelerdiń hár biri kesilisiwshi eki tuwrını anıqlawın kórsetiń hám onıń koordinataların tabıń: $x^2+4 x y+3 y^2-6 x-12 y+9=0$.

\textbf{682.1)} Koordinatalar sistemasın túrlendirmesten, tómendegi teńlemeler menen qanday geometriyalıq obrazdı anıqlanıwın tabıń: $8 x^2-12 x y+17 y^2+16 x-12 y+3=0$;

\textbf{682.2)} Koordinatalar sistemasın túrlendirmesten, tómendegi teńlemeler menen qanday geometriyalıq obrazdı anıqlanıwın tabıń: $17 x^2-18 x y-7 y^2+34 x-18 y+7=0$;

\textbf{682.3)} Koordinatalar sistemasın túrlendirmesten, tómendegi teńlemeler menen qanday geometriyalıq obrazdı anıqlanıwın tabıń: $2 x^2+3 x y-2 y^2+5 x+10 y=0$;

\textbf{682.4)} Koordinatalar sistemasın túrlendirmesten, tómendegi teńlemeler menen qanday geometriyalıq obrazdı anıqlanıwın tabıń: $6 x^2-6 x y+9 y^2-4 x+18 y+14=0$;



\section{Parabolik teńlemeni ápiwayı túrge keltiriw}



\textbf{697.1) } Koordinatalar sistemasın túrlendirmesten, tómendegi teńlemelerdiń hár biri parabolanı anıqlawın kórsetiń hám parametrin tabıń: $9 x^2+24 x y+16 y^2-120 x+90 y=0$;

\textbf{697.2)} Koordinatalar sistemasın túrlendirmesten, tómendegi teńlemelerdiń hár biri parabolanı anıqlawın kórsetiń hám parametrin tabıń: $9 x^2-24 x y+16 y^2-54 x-178 y+181=0$;

\textbf{697.3)} Koordinatalar sistemasın túrlendirmesten, tómendegi teńlemelerdiń hár biri parabolanı anıqlawın kórsetiń hám parametrin tabıń: $x^2-2 x y+y^2+6 x-14 y+29=0$;

\textbf{697.4)} Koordinatalar sistemasın túrlendirmesten, tómendegi teńlemelerdiń hár biri parabolanı anıqlawın kórsetiń hám parametrin tabıń: $9 x^2-6 x y+y^2-50 x+50 y-275=0$.



\section{Ekinshi tártipli betler}



\textbf{1153.} $x-2=0$ tegislik $\frac{x^2}{16}+\frac{y^2}{12}+\frac{z^2}{4}=1$ ellipsoidti ellips boyınsha kesip ótetuģının kórsetiń; onıń yarım kósherleri hám tóbelerin tabıń.

\textbf{1154.} $z+1=0$ tegislik bir qabatlı $\frac{x^2}{32}-\frac{y^2}{18}+\frac{z^2}{2}=1$ giperboloidti giperbola boyınsha kesip ótetuģının kórsetiń; onıń yarım kósherleri hám tóbelerin tabıń.

\textbf{1155.} $y+6=0$ tegislik $\frac{x^2}{5}-\frac{y^2}{4}=6 z$ giperbolik paraboloidti parabola boyınsha kesip ótetuģının kórsetiń; parametrin hám tóbesin tabıń.

\textbf{1156.} $y^2+z^2=x$ elliptik paraboloidtıń $x+2 y-z=0$ tegislik penen kesilisiwiniń koordinata tegisliklerindegi proekciyalarınıń teńlemelerin tabıń.

\textbf{1159.1)} Berilgen teńleme menen qanday iymek sızıq anıqlanıwın tabıń: $\left\{\begin{array}{l}\frac{x^2}{3}+\frac{y^2}{6}=2 z, \\ 3 x-y+6 z-14=0\end{array}\right.$

\textbf{1159.2)} Berilgen teńleme menen qanday iymek sızıq anıqlanıwın tabıń: $\left\{\begin{array}{l}\frac{x^2}{4}-\frac{y^2}{3}=2 z \\ x-2 y+2=0 ;\end{array}\right.$

\textbf{1159.3)} Berilgen teńleme menen qanday iymek sızıq anıqlanıwın tabıń: $\left\{\begin{array}{l}\frac{x^2}{.4}+\frac{y^2}{9}-\frac{z^2}{36}=1, \\ 9 x-6 y+2 z-28=0,\end{array}\right.$
