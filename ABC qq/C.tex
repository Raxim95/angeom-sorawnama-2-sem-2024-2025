\section{Ellipstiń kanonikalıq teńlemesi}


\textbf {468.} Ellipstiń yarım kósherleri $a$, $b$ hám orayı $C\left (x_0; y_0\right) $ noqatında bolıp, simmetriya kósherleri koordinata kósherlerine parallel ekenligi belgili bolsa, onıń teńlemesin dúziń.


\textbf {484.} $m$ nıń qanday mánislerinde $y=-x+m$ sızıq: 1) $\frac{x^2}{20}+\frac{y^2}{5}=1$ ellipsti kesip ótedi; 2) ellipske urinadı; 3) ellipsti kesip ótpeydi.


\textbf {486.} $\frac{x^2}{a^2}+\frac{y^2}{b^2}=1$ ellipstiń $M_1 (x_1; y_1) $ noqatındaǵı urinbasınıń teńlemesin dúziń.

\textbf{487.} $\frac{x^2}{a^2}+\frac{y^2}{b^2}=1$ ellipstiń bir diametriniń tóbelerine júrgizilgen urinbalar parallel boliwin dálilleń (ellipstiń diametri dep oniń orayınan ótiwshi xordaga aytıladı).

\textbf{490.} $\frac{x^2}{30}+\frac{y^2}{24}=1$ ellipsga $4x-2y+23=0$ parallel bolgan urinbalardı júrgiziń hám olar arasındaģı aralıqtı esaplań.

\textbf{492.} $A\left(\frac{10}{3}; \frac{5}{3}\right)$ nuqtadan $\frac{x2}{20}+\frac{y2}{5}=1$ ellipske urinbalar ótkerilgen. Olardıń teńlemelerin dúziń.

\textbf{497.} Ellips orayınan onıń qálegen urınbasınıń fokal kósher menen kesilisiw noqatına shekemgi hám uriniw noqatınan fokal kósherge túsirilgen perpendikulyar ultanına shekemgi aralıqlar kóbeymesi turaqlı shama bolıp, ellips úlken yarım kósheriniń kvadratına teń ekenligin dálilleń.

\textbf 498. Fokuslardan ellipstiń qálegen urinbasına shekem bolgan aralıqlar kóbeymesi kishi yarım kósherdiń kvadratına teń ekenligin dálilleń.



\section{Giperbolanıń kanonikalıq teńlemesi}





{537.} $\frac{x^2}{a^2}-\frac{y^2}{b^2}=1$ giperbolanıń fokusınan asimptotasına shekemgi aralıq $b$ qa teń ekenligin dálilleń.

\textbf{537.} $\frac{x^2}{a^2}-\frac{y^2}{b^2}=1$ giperbolanıń fokusınan asimptotasına shekemgi aralıq $b$ qa teń ekenligin dálilleń.

\textbf{538.} $\frac{x^2}{a^2}-\frac{y^2}{b^2}=1$ giperbolanıń qálegen noqatınan onıń eki asimptotasına shekemgi aralıqlar kóbeymesi $\frac{a^2 b^2}{a^2+b^2}$ ǵa teń turaqlı shama ekenligin dálilleń.

\textbf{539.} $\frac{x^2}{a^2}-\frac{y^2}{b^2}=1$ giperbolanıń asimptotaları hám onıń qálegen noqatınan asimptotalarga parallel etip ótkerilgen tuwrı sızıqlar menen shegaralangan parallelogrammnıń maydanı turaqlı san bolıp, $\frac{a b}{2}$ ga teń bolatuģının dálilleń.
\textbf{540.} Eger giperbolanıń yarım kósherleri $a$ hám $b$, orayı $C\left (x_0; y_0\right) $ hám fokuslar tómendegi tuwrı sızıqta jaylasqan: 1) $O x$ kósherine parallel; 2) $O y$ kósherine parallel bolsa, oniń teńlemesin dúziń.

\textbf{555.} $m$ nıń qanday mánislerinde $y=\frac{5}{2} x+m$ tuwrı sızıq $\frac{x^2}{9}-\frac{y^2}{36}=1$ giperbolanı 1) kesip ótiwin; 2) ogan urınıwın; 3) sırtınan ótiwin anıqlań.

\textbf{556.} $y=k x+m$ tuwri sızıqtıń $\frac{x^2}{a^2}-\frac{y^2}{b^2}=1$ giperbolaģa urinıw shártin keltirip shiģarıń.

\textbf{557.} $\frac{x^2}{a^2}-\frac{y^2}{b^2}=1$ giperbolaģa oniń $M_1\left (x_1; y_1\right) $ noqatındaǵı urinbasınıń teńlemesin dúziń.

\textbf{558.} Giperbolanıń bir diametr tóbelerinen ótkerilgen urinbalar parallel boliwin dálilleń.

\textbf {567.} Tómendegi eki tuwrı sızıqqa urınatuģın giperbolanıń teńlemesin dúziń:$5x-6y-16=0$, $13x-10y-48=0$, bunda onıń kósherleri koordinata kósherleri menen ústpe-úst túsedi.

\textbf{569.} $\frac{x^2}{a^2}-\frac{y^2}{b^2}=1$ giperbola hám onıń qanday da bir urinbası berilgen: $P$-urinbasınıń $O x$ kósheri menen kesilisiw noqatı, $Q$ - urinba noqatınıń sol kósherdegi proekciyası. $O P \cdot O Q=a^2$ ekenligin dálilleń.

\section{ Parabolanıń kanonikalıq teńlemesi}


\textbf{609.} Múyesh koefficienti $k$ tiń qanday mánislerinde $y=kx+2$ tuwri sızıq: 1) $y^2=4x$ parabolanı kesip ótedi; 2) ogan urinadı; 3) bul parabola sırtınan ótedi.

\textbf {610.} $y^2=2 p x$ parabolaga $y=k x+b$ tuwri sızıq urinıw shártin keltirip shigarıń.

\textbf {612.} $y^2=2 p x$ parabolaga oniń $M_1\left (x_1; y_1\right) $ noqatındaǵı urinbasınıń teńlemesin dúziń.

\textbf {626.} Uliwma kósherge hám tóbeleri arasında jaylasqan uliwma fokusqa iye bolgan eki parabola tuwri múyesh astında kesilisetuģının dálilleń.

\textbf{627.} Kósherleri óz ara perpendikulyar bolgan eki parabola tórt noqatta kesilisse, bul noqatlar bir sheńberde jatıwın dálilleń.



\section{ Ekinshi tártipli sızıq orayı }

\textbf{669.} $m$ hám $n$ tiń qanday mánislerinde $m x^2+12 x y+9 y^2+4 x+n y-13=0$ teńleme: 1) oraylıq sızıqtı; 2) orayga iye bolmagan sızıq; 3) sheksiz kóp orayga iye bolgan sızıqtı anlatadı.

\textbf {670.} Sızıq teńlemesi berilgen: $4 x^2-4 x y+y^2+6 x+1=0$. Keltirilgen mánisler boyinsha tómendegilerdi anıqlań: 1) tuwri sızıq penen bir noqatta kesilisedi; 2) usi sızıqqa tiyisli; 3) bul sızıqtı eki noqatta kesip ótedi; 4) bul sızıq penen uliwma noqatqa iye emes.
$4 x^2-4 x y+y^2+6 x+1=0$ ITECH teńlemesi berilgen. Múyesh koefficienti $k$ tiń qanday mánislerinde $y=kx$ tuwri sızıq: 1) bul sızıqtı bir noqatta kesip ótiwi; 2) usi sızıqqa urinadı; 3) bul sızıqtı eki noqatta kesip ótedi; 4) bul tuwri sızıq penen uliwma noqatqa iye bolmaytuģının anıqlań.



\section{Ekinshi tártipli oraylıq sızıq teńlemesin ápiwayı túrge keltiriw}


\textbf{676.1)} Berilgen teńleme kanonikalıq kóriniske keltirilsin; tipi anıqlansın; qanday geometriyalıq obrazdı anlatıwı anıqlansın; eski hám jana koordinatalar sistemasında geometriyalıq obrazı súwretlensin:$3 x^2+10 x y+3 y^2-2 x-14 y-13=0$;

\textbf{676.2)} Berilgen teńleme kanonikalıq kóriniske keltirilsin; tipi anıqlansın; qanday geometriyalıq obrazdı anlatıwı anıqlansın; eski hám jana koordinatalar sistemasında geometriyalıq obrazı súwretlensin:$25 x^2-14 x y+25 y^2+64 x-64 y-224=0$;

\textbf{676.3)} Berilgen teńleme kanonikalıq kóriniske keltirilsin; tipi anıqlansın; qanday geometriyalıq obrazdı anlatıwı anıqlansın; eski hám jana koordinatalar sistemasında geometriyalıq obrazı súwretlensin:$4 x y+3 y^2+16 x+12 y-36=0$;

\textbf{676.4)} Berilgen teńleme kanonikalıq kóriniske keltirilsin; tipi anıqlansın; qanday geometriyalıq obrazdı anlatıwı anıqlansın; eski hám jana koordinatalar sistemasında geometriyalıq obrazı súwretlensin:$7 x^2+6 x y-y^2+28 x+12 y+28=0$;

\textbf{676.5)} Berilgen teńleme kanonikalıq kóriniske keltirilsin; tipi anıqlansın; qanday geometriyalıq obrazdı anlatıwı anıqlansın; eski hám jana koordinatalar sistemasında geometriyalıq obrazı súwretlensin:$19 x^2+6 x y+11 y^2+38 x+6 y+29=0$;

\textbf{676.6)} Berilgen teńleme kanonikalıq kóriniske keltirilsin; tipi anıqlansın; qanday geometriyalıq obrazdı anlatıwı anıqlansın; eski hám jana koordinatalar sistemasında geometriyalıq obrazı súwretlensin:$5 x^2-2 x y+5 y^2-4 x+20 y+20=0$.

\textbf{677.1)} Berilgen teńleme kanonikalıq kóriniske keltirilsin; tipi anıqlansın; qanday geometriyalıq obrazdı anlatıwı anıqlansın; eski hám jana koordinatalar sistemasında geometriyalıq obrazı súwretlensin:$14 x^2+24 x y+21 y^2-4 x+18 y-139=0$;

\textbf{677.2)} Berilgen teńleme kanonikalıq kóriniske keltirilsin; tipi anıqlansın; qanday geometriyalıq obrazdı anlatıwı anıqlansın; eski hám jana koordinatalar sistemasında geometriyalıq obrazı súwretlensin:$11 x^2-20 x y-4 y^2-20 x-8 y+1=0$;

\textbf{677.3)} Berilgen teńleme kanonikalıq kóriniske keltirilsin; tipi anıqlansın; qanday geometriyalıq obrazdı anlatıwı anıqlansın; eski hám jana koordinatalar sistemasında geometriyalıq obrazı súwretlensin:$7 x^2+60 x y+32 y^2-14 x-60 y+7=0$;

\textbf{677.4)} Berilgen teńleme kanonikalıq kóriniske keltirilsin; tipi anıqlansın; qanday geometriyalıq obrazdı anlatıwı anıqlansın; eski hám jana koordinatalar sistemasında geometriyalıq obrazı súwretlensin:$50 x^2-8 x y+35 y^2+100 x-8 y+67=0$;

\textbf{677.5)} Berilgen teńleme kanonikalıq kóriniske keltirilsin; tipi anıqlansın; qanday geometriyalıq obrazdı anlatıwı anıqlansın; eski hám jana koordinatalar sistemasında geometriyalıq obrazı súwretlensin:$41 x^2+24 x y+34 y^2+34 x-112 y+129=0$;

\textbf{677.6)} Berilgen teńleme kanonikalıq kóriniske keltirilsin; tipi anıqlansın; qanday geometriyalıq obrazdı anlatıwı anıqlansın; eski hám jana koordinatalar sistemasında geometriyalıq obrazı súwretlensin:$29 x^2-24 x y+36 y^2+82 x-96 y-91=0$;

\textbf{677.7)} Berilgen teńleme kanonikalıq kóriniske keltirilsin; tipi anıqlansın; qanday geometriyalıq obrazdı anlatıwı anıqlansın; eski hám jana koordinatalar sistemasında geometriyalıq obrazı súwretlensin:$4 x^2+24 x y+11 y^2+64 x+42 y+51=0$;

\textbf{677.8)} Berilgen teńleme kanonikalıq kóriniske keltirilsin; tipi anıqlansın; qanday geometriyalıq obrazdı anlatıwı anıqlansın; eski hám jana koordinatalar sistemasında geometriyalıq obrazı súwretlensin:$41 x^2+24 x y+9 y^2+24 x+18 y-36=0$.

\textbf{683.} Qálegen elliptik teńleme ushin $a_{11}$ hám $a_{22}$ koefficientlerinin hesh biri nólge aylana almaytuģınlıǵın hám olar birdey belgige iye sanlar ekenligin dálilleń.

\textbf {684.} Elliptik túrdegi ($\delta>0$) teńleme $a_{11}$ hám $\Delta$ lardıń hár qıylı belgige iye sanlar bolganda gana ellipsti anıqlawın dálilleń.

\textbf {685.} Elliptik túrdegi ($\delta>0$) teńleme $a_{11}$ hám $\Delta$ birdey belgige iye san bolganda gana jormal ellips teńlemesi bolatuģının dálilleń.

\textbf {686.} Elliptik túrdegi ($\delta>0$) teńleme $\Delta=0$ bolganda gana eki bir-birin kesip ótiwshi jormal tuwri sızıq bolatuģının dálilleń.


\section{Parabolik teńlemeni ápiwayı túrge keltiriw}


\textbf{691.} Hár qanday parabolik teńleme ushın $a_{11}$ hám $a_{22}$ koefficientler hár qıylı belgige iye sanlar bola almaytuģının hám olar bir waqıtta nolge aylana almaytuģının dálilleń.

\textbf {692.} Hár qanday parabolik teńleme $ (\alpha x+\beta y) ^2+2a_{13}x+2a_{23}y+a_{33}=0 kórinisinde jazılıwı múmkinligin dálilleń. Sonday-aq, elliptikalıq hám giperbolikalıq teńlemelerdi bunday kóriniste jazıp bolmaytuģının dálilleń.

\textbf {694.} Eger ekinshi dárejeli teńleme parabolik bolip, $ (\alpha x+\beta y) ^2+2a_{13}x+2a_{23}y+a_{33}=0$ kórinisinde jazılsa, oniń shep tárepindegi diskriminant $\Delta=- (a_{13} \beta-a_{23} \alpha) ^2$ formula menen anıqlanıwın dálilleń.

\textbf {696.} Parabolik teńleme $\Delta \neq 0$ bolganda hám sondagina parabolanı anıqlaytuģının dálilleń. Bul jagdayda parabolanıń parametri $p=\sqrt{\frac{-\Delta}{ (a_{11}+a_{33}) ^3}}$ formula menen anıqlanıwın dálilleń.

\textbf{698.} Ekinshi dárejeli teńleme tek hám tek $\Delta=0$ bolganda gana aynıgan sızıq teńlemesi bolatuģının dálilleń.

\section{ Ekinshi tártipli betler}



\textbf {1160.} 5. x+mz-1=0 tegislik tómendegi eki qabatlı giperboloidti x2+y2−z2=−1 ge teń qanday mánislerde a) ellips boyinsha, b) giperbola boyinsha kesedi?



\textbf {1161.} $m$ nıń qanday mánislerinde $x+m y-2=0$ tegislik $\frac{x^2}{2}+\frac{z^2}{3}=y$ elliptik paraboloidti a) ellips boylap, b) parabola boylap kesip ótetuǵınlıǵın anıqlań.

\textbf{1162.} $\frac{x^2}{9}+\frac{z^2}{4}=2 y$ elliptik paraboloid $2 x-2 y-z-10=0$ tegislik penen bir uliwma noqatqa iye ekenligin dálilleń hám oniń koordinataların tabıń.

\textbf{1163.} Eki qabatlı $\frac{x^2}{3}+\frac{y^2}{4}-\frac{z^2}{25}=-1$ giperboloid $5 x+2 z+5=0$ tegislik penen bir uliwma noqatqa iye ekenligin dálilleń hám oniń koordinataların tabıń.

\textbf{1164.} $\frac{x^2}{81}+\frac{y^2}{36}+\frac{z^2}{9}=1$ ellipsoid $4 x-3 y+12 z-54=0$ tegislik penen bir uliwma noqatqa iye ekenligin dálilleń hám oniń koordinataların tabıń.

\textbf{1165.} $m$ nıń qanday mánisinde $x-2 y-2 z+m=0$ tegislik $\frac{x^2}{144}+\frac{y^2}{36}+\frac{z^2}{9}=1$ ellipsoidqa uriniwin anıqlań.
