\section{Ellipstiń kanonikalıq teńlemesi}


\textbf{468.} Ellipstiń yarım kósherleri $a$, $b$ hám orayı $C\left(x_0; y_0\right) $ noqatında bolıp, simmetriya kósherleri koordinata kósherlerine parallel ekenligi belgili bolsa, onıń teńlemesin dúziń.

\textbf{484.} $m$ nıń qanday mánislerinde $y=-x+m$ sızıq: 1) $\frac{x^2}{20}+\frac{y^2}{5}=1$ ellipsti kesip ótedi; 2) ellipske urınadı; 3) ellipsti kesip ótpeydi.

\textbf{486.} $\frac{x^2}{a^2}+\frac{y^2}{b^2}=1$ ellipstiń $M_1 (x_1; y_1) $ noqatındaǵı urınbasınıń teńlemesin dúziń.

\textbf{487.} $\frac{x^2}{a^2}+\frac{y^2}{b^2}=1$ ellipstiń bir diametriniń tóbelerine júrgizilgen urınbalar parallel bolıwın dálilleń (ellipstiń diametri dep onıń orayınan ótiwshi xordaǵa aytıladı).

\textbf{490.} $\frac{x^2}{30}+\frac{y^2}{24}=1$ ellipske $4x-2y+23=0$ parallel bolǵan urınbalardı júrgiziń hám olar arasındaģı aralıqtı esaplań.

\textbf{492.} $A\left(\frac{10}{3}; \frac{5}{3}\right)$ noqatınan $\frac{x2}{20}+\frac{y2}{5}=1$ ellipske urınbalar ótkerilgen. Olardıń teńlemelerin dúziń.

\textbf{497.} Ellips orayınan onıń qálegen urınbasınıń fokal kósher menen kesilisiw noqatına shekemgi hám urınıw noqatınan fokal kósherge túsirilgen perpendikulyar ultanına shekemgi aralıqlar kóbeymesi turaqlı shama bolıp, ellips úlken yarım kósheriniń kvadratına teń ekenligin dálilleń.

\textbf{498.} Fokuslardan ellipstiń qálegen urınbasına shekem bolǵan aralıqlar kóbeymesi kishi yarım kósherdiń kvadratına teń ekenligin dálilleń.



\section{Giperbolanıń kanonikalıq teńlemesi}





\textbf{537.} $\frac{x^2}{a^2}-\frac{y^2}{b^2}=1$ giperbolanıń fokusınan asimptotasına shekemgi aralıq $b$ qa teń ekenligin dálilleń.

\textbf{538.} $\frac{x^2}{a^2}-\frac{y^2}{b^2}=1$ giperbolanıń qálegen noqatınan onıń eki asimptotasına shekemgi aralıqlar kóbeymesi $\frac{a^2 b^2}{a^2+b^2}$ ǵa teń turaqlı shama ekenligin dálilleń.

\textbf{539.} $\frac{x^2}{a^2}-\frac{y^2}{b^2}=1$ giperbolanıń asimptotaları hám onıń qálegen noqatınan asimptotalarga parallel etip ótkerilgen tuwrı sızıqlar menen shegaralanǵan parallelogrammnıń maydanı turaqlı san bolıp, $\frac{a b}{2}$ ga teń bolatuģının dálilleń.

\textbf{540.} Eger giperbolanıń yarım kósherleri $a$ hám $b$, orayı $C\left(x_0; y_0\right) $ hám fokuslar tómendegi tuwrı sızıqta jaylasqan: 1) $O x$ kósherine parallel; 2) $O y$ kósherine parallel bolsa, onıń teńlemesin dúziń.

\textbf{555.} $m$ nıń qanday mánislerinde $y=\frac{5}{2} x+m$ tuwrı sızıq $\frac{x^2}{9}-\frac{y^2}{36}=1$ giperbolanı 1) kesip ótiwin; 2) oǵan urınıwın; 3) sırtınan ótiwin anıqlań.

\textbf{556.} $y=k x+m$ tuwrı sızıqtıń $\frac{x^2}{a^2}-\frac{y^2}{b^2}=1$ giperbolaģa urınıw shártin keltirip shıģarıń.

\textbf{557.} $\frac{x^2}{a^2}-\frac{y^2}{b^2}=1$ giperbolaģa onıń $M_1\left(x_1; y_1\right) $ noqatındaǵı urınbasınıń teńlemesin dúziń.

\textbf{558.} Giperbolanıń bir diametr tóbelerinen ótkerilgen urınbalar parallel bolıwın dálilleń.

\textbf{567.} Tómendegi eki tuwrı sızıqqa urınatuģın giperbolanıń teńlemesin dúziń: $5x-6y-16=0$, $13x-10y-48=0$, bunda onıń kósherleri koordinata kósherleri menen ústpe-úst túsedi.

\textbf{569.} $\frac{x^2}{a^2}-\frac{y^2}{b^2}=1$ giperbola hám onıń qanday da bir urınbası berilgen: $P$-urınbasınıń $O x$ kósheri menen kesilisiw noqatı, $Q$ - urınba noqatınıń sol kósherdegi proekciyası. $O P \cdot O Q=a^2$ ekenligin dálilleń.


\section{ Parabolanıń kanonikalıq teńlemesi}


\textbf{609.} Múyesh koefficienti $k$ tiń qanday mánislerinde $y=kx+2$ tuwrısı: 1) $y^2=4x$ parabolanı kesip ótedi; 2) oǵan urınadı; 3) bul parabola sırtınan ótedi.

\textbf{610.} $y^2=2 p x$ parabolaǵa $y=k x+b$ tuwrı sızıq urınıw shártin keltirip shigarıń.

\textbf{612.} $y^2=2 p x$ parabolaǵa onıń $M_1\left(x_1; y_1\right) $ noqatındaǵı urınbasınıń teńlemesin dúziń.

\textbf{626.} Ulıwma kósherge hám tóbeleri arasında jaylasqan ulıwma fokusqa iye bolǵan eki parabola tuwrı múyesh astında kesilisetuģının dálilleń.

\textbf{627.} Kósherleri óz ara perpendikulyar bolǵan eki parabola tórt noqatta kesilisse, bul noqatlar bir sheńberde jatıwın dálilleń.



\section{ Ekinshi tártipli sızıq orayı }


\textbf{669.} $m$ hám $n$ tiń qanday mánislerinde $m x^2+12 x y+9 y^2+4 x+n y-13=0$ teńleme: 1) oraylıq sızıqtı; 2) orayga iye bolmaǵan sızıq; 3) sheksiz kóp orayǵa iye bolǵan sızıqtı ańlatadı.

\textbf{670.} $4 x^2-4 x y+y^2+6 x+1=0$ ETIS teńlemesi berilgen. Múyesh koefficienti $k$ tiń qanday mánislerinde $y=kx$ tuwrı sızıq: 1) bul iymek sızıqtı bir noqatta kesip ótiwi; 2) urınadı; 3) eki noqatta kesip ótiwin; 4) bul tuwrı menen ulıwma noqatqa iye bolmaytuģının anıqlań.



\section{Ekinshi tártipli oraylıq sızıq teńlemesin ápiwayı túrge keltiriw}


\textbf{676.1)} Berilgen teńleme kanonikalıq kóriniske keltirilsin; tipi anıqlansın; qanday geometriyalıq obrazdı anlatıwı anıqlansın; eski hám jana koordinatalar sistemasında geometriyalıq obrazı súwretlensin: $3 x^2+10 x y+3 y^2-2 x-14 y-13=0$;

\textbf{676.2)} Berilgen teńleme kanonikalıq kóriniske keltirilsin; tipi anıqlansın; qanday geometriyalıq obrazdı anlatıwı anıqlansın; eski hám jana koordinatalar sistemasında geometriyalıq obrazı súwretlensin: $25 x^2-14 x y+25 y^2+64 x-64 y-224=0$;

\textbf{676.3)} Berilgen teńleme kanonikalıq kóriniske keltirilsin; tipi anıqlansın; qanday geometriyalıq obrazdı anlatıwı anıqlansın; eski hám jana koordinatalar sistemasında geometriyalıq obrazı súwretlensin: $4 x y+3 y^2+16 x+12 y-36=0$;

\textbf{676.4)} Berilgen teńleme kanonikalıq kóriniske keltirilsin; tipi anıqlansın; qanday geometriyalıq obrazdı anlatıwı anıqlansın; eski hám jana koordinatalar sistemasında geometriyalıq obrazı súwretlensin: $7 x^2+6 x y-y^2+28 x+12 y+28=0$;

\textbf{676.5)} Berilgen teńleme kanonikalıq kóriniske keltirilsin; tipi anıqlansın; qanday geometriyalıq obrazdı anlatıwı anıqlansın; eski hám jana koordinatalar sistemasında geometriyalıq obrazı súwretlensin: $19 x^2+6 x y+11 y^2+38 x+6 y+29=0$;

\textbf{676.6)} Berilgen teńleme kanonikalıq kóriniske keltirilsin; tipi anıqlansın; qanday geometriyalıq obrazdı anlatıwı anıqlansın; eski hám jana koordinatalar sistemasında geometriyalıq obrazı súwretlensin: $5 x^2-2 x y+5 y^2-4 x+20 y+20=0$.

\textbf{677.1)} Berilgen teńleme kanonikalıq kóriniske keltirilsin; tipi anıqlansın; qanday geometriyalıq obrazdı anlatıwı anıqlansın; eski hám jana koordinatalar sistemasında geometriyalıq obrazı súwretlensin: $14 x^2+24 x y+21 y^2-4 x+18 y-139=0$;

\textbf{677.2)} Berilgen teńleme kanonikalıq kóriniske keltirilsin; tipi anıqlansın; qanday geometriyalıq obrazdı anlatıwı anıqlansın; eski hám jana koordinatalar sistemasında geometriyalıq obrazı súwretlensin: $11 x^2-20 x y-4 y^2-20 x-8 y+1=0$;

\textbf{677.3)} Berilgen teńleme kanonikalıq kóriniske keltirilsin; tipi anıqlansın; qanday geometriyalıq obrazdı anlatıwı anıqlansın; eski hám jana koordinatalar sistemasında geometriyalıq obrazı súwretlensin: $7 x^2+60 x y+32 y^2-14 x-60 y+7=0$;

\textbf{677.4)} Berilgen teńleme kanonikalıq kóriniske keltirilsin; tipi anıqlansın; qanday geometriyalıq obrazdı anlatıwı anıqlansın; eski hám jana koordinatalar sistemasında geometriyalıq obrazı súwretlensin: $50 x^2-8 x y+35 y^2+100 x-8 y+67=0$;

\textbf{677.5)} Berilgen teńleme kanonikalıq kóriniske keltirilsin; tipi anıqlansın; qanday geometriyalıq obrazdı anlatıwı anıqlansın; eski hám jana koordinatalar sistemasında geometriyalıq obrazı súwretlensin: $41 x^2+24 x y+34 y^2+34 x-112 y+129=0$;

\textbf{677.6)} Berilgen teńleme kanonikalıq kóriniske keltirilsin; tipi anıqlansın; qanday geometriyalıq obrazdı anlatıwı anıqlansın; eski hám jana koordinatalar sistemasında geometriyalıq obrazı súwretlensin: $29 x^2-24 x y+36 y^2+82 x-96 y-91=0$;

\textbf{677.7)} Berilgen teńleme kanonikalıq kóriniske keltirilsin; tipi anıqlansın; qanday geometriyalıq obrazdı anlatıwı anıqlansın; eski hám jana koordinatalar sistemasında geometriyalıq obrazı súwretlensin: $4 x^2+24 x y+11 y^2+64 x+42 y+51=0$;

\textbf{677.8)} Berilgen teńleme kanonikalıq kóriniske keltirilsin; tipi anıqlansın; qanday geometriyalıq obrazdı anlatıwı anıqlansın; eski hám jana koordinatalar sistemasında geometriyalıq obrazı súwretlensin: $41 x^2+24 x y+9 y^2+24 x+18 y-36=0$.

\textbf{683.} Qálegen elliptik teńleme ushın $a_{11}$ hám $a_{22}$ koefficientleriniń hesh biri nolge aylana almaytuģınlıǵın hám olar birdey belgige iye sanlar ekenligin dálilleń.

\textbf{684.} Elliptik túrdegi ($\delta>0$) teńleme $a_{11}$ hám $\Delta$ lardıń hár qıylı belgige iye sanlar bolǵanda ǵana ellipsti anıqlawın dálilleń.

\textbf{685.} Elliptik túrdegi ($\delta>0$) teńleme $a_{11}$ hám $\Delta$ birdey belgige iye san bolǵanda ǵana jormal ellips teńlemesi bolatuģının dálilleń.

\textbf{686.} Elliptik túrdegi ($\delta>0$) teńleme $\Delta=0$ bolǵanda ǵana eki bir-birin kesip ótiwshi jormal tuwrı sızıq bolatuģının dálilleń.


\section{Parabolik teńlemeni ápiwayı túrge keltiriw}


\textbf{691.} Hár qanday parabolik teńleme ushın $a_{11}$ hám $a_{22}$ koefficientler hár qıylı belgige iye sanlar bola almaytuģının hám olar bir waqıtta nolge aylana almaytuģının dálilleń.

\textbf{692.} Hár qanday parabolik teńleme $ (\alpha x+\beta y) ^2+2a_{13}x+2a_{23}y+a_{33}=0$ kórinisinde jazılıwı múmkinligin dálilleń. Sonday-aq, elliptikalıq hám giperbolikalıq teńlemelerdi bunday kóriniste jazıp bolmaytuģının dálilleń.

\textbf{694.} Eger ekinshi dárejeli teńleme parabolik bolıp, $ (\alpha x+\beta y) ^2+2a_{13}x+2a_{23}y+a_{33}=0$ kórinisinde jazılsa, onıń shep tárepindegi diskriminant $\Delta=- (a_{13} \beta-a_{23} \alpha) ^2$ formula menen anıqlanıwın dálilleń.

\textbf{696.} Parabolik teńleme $\Delta \neq 0$ bolǵanda hám tek sonda ǵana parabolanı anıqlaytuģının dálilleń. Bul jaǵdayda parabolanıń parametri $p=\sqrt{\frac{-\Delta}{ (a_{11}+a_{33}) ^3}}$ formula menen anıqlanıwın dálilleń.

\textbf{698.} Ekinshi dárejeli teńleme tek hám tek $\Delta=0$ bolǵanda ǵana aynıǵan iymek sızıq teńlemesi bolatuģının dálilleń.



\section{ Ekinshi tártipli betler}



\textbf{1160.} $m$ niń qanday mánislerinde $x+mz-1=0$ tegislik tómendegi $x^2+y^2−z^2=−1$ eki gewekli giperboloidti a) ellips boyınsha, b) giperbola boyınsha kesedi?

\textbf{1161.} $m$ nıń qanday mánislerinde $x+m y-2=0$ tegislik $\frac{x^2}{2}+\frac{z^2}{3}=y$ elliptik paraboloidti a) ellips boyınsha, b) parabola boyınsha kesip ótetuǵınlıǵın anıqlań.

\textbf{1162.} $\frac{x^2}{9}+\frac{z^2}{4}=2 y$ elliptik paraboloid $2 x-2 y-z-10=0$ tegislik penen bir ulıwma noqatqa iye ekenligin dálilleń hám onıń koordinataların tabıń.

\textbf{1163.} Eki gewekli $\frac{x^2}{3}+\frac{y^2}{4}-\frac{z^2}{25}=-1$ giperboloid $5 x+2 z+5=0$ tegislik penen bir ulıwma noqatqa iye ekenligin dálilleń hám onıń koordinataların tabıń.

\textbf{1164.} $\frac{x^2}{81}+\frac{y^2}{36}+\frac{z^2}{9}=1$ ellipsoid $4 x-3 y+12 z-54=0$ tegislik penen bir ulıwma noqatqa iye ekenligin dálilleń hám onıń koordinataların tabıń.

\textbf{1165.} $m$ nıń qanday mánislerinde $x-2 y-2 z+m=0$ tegislik $\frac{x^2}{144}+\frac{y^2}{36}+\frac{z^2}{9}=1$ ellipsoidqa urınıwın anıqlań.
