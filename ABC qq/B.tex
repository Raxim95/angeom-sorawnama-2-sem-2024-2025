\section{Ellipstiń kanonikalıq teńlemesi}



\textbf{448.} Eki tóbesi $x^2+5 y^2=20$ ellipstiń fokuslarında jatıwshı, qalǵan ekewi bolsa onıń kishi kósheri tóbeleri menen ústpe-úst túsiwshi tórtmúyeshliktiń maydanın esaplań.


\textbf{456.} $\varepsilon=\frac{2}{3}$ ellipsiniń ekssentrisiteti, $M$ ellips noqatınıń fokal radiusı 10 ģa teń. $M$ noqattan usı fokusqa sáykes direktrisaǵa shekem bolǵan aralıqtı esaplań.


\textbf{457.} $\varepsilon=\frac{2}{5}$ ellipstiń ekssentrisiteti, ellipstiń $M$ noqatınan direktrisaģa shekemgi aralıq 20 ģa teń. $M$ noqattan usı direktrisa menen bir tárepleme fokusqa shekem bolǵan aralıqtı esaplań.

\textbf{460.} Ellipstiń ekssentrisiteti $\varepsilon=\frac{1}{3}$, onıń orayı koordinatalar bası menen ústpe-úst túsedi, fokuslarınan biri $ (-2; 0) $. Abcissası 2 ge teń bolǵan ellipstiń $M_1$ noqatınan berilgen fokusqa sáykes direktrisaģa shekem bolǵan aralıqtı tabıń.

\textbf{461.} Ellipstiń ekssentrisiteti $\varepsilon=\frac{1}{2}$, onıń orayı koordinatalar bası menen ústpe-úst túsedi, direktrisalardan biri $x=16$ teńleme menen berilgen. Abcissası $-4$ ke teń bolǵan ellipstiń $M_1$ noqatınan berilgen direktrisa menen bir tárepleme fokusqa shekem bolǵan aralıqtı esaplań.

\textbf{464.} $\frac{x^2}{25}+\frac{y^2}{15}=1$ ellipstiń fokusı arqalı onıń úlken kósherine perpendikulyar ótkerilgen. Bul perpendikulyardıń ellips penen kesilisken noqatlarınan fokuslarga shekem bolǵan aralıqlardı anıqlań.

\textbf{464.} Ellipstegi ekssentrisitetti anıqlań, eger: onıń kishi kósheri fokuslardan $60^{\circ}$ múyesh astında kórinse;

\textbf{464.} Ellipstegi ekssentrisitetti anıqlań, eger: fokusları arasındaǵı kesindiniń ózi kishi kósherdiń tóbesinen tuwrı múyesh astında kórinse;

\textbf{464.} Ellipstegi ekssentrisitetti anıqlań, eger: direktrisalar arasındaǵı aralıq fokuslar arasındaǵı aralıqtan úsh ese úlken bolsa;

\textbf{466.4)} Ellipstegi ekssentrisitetti anıqlań, eger: ellips orayınan onıń direktrisasına túsirilgen perpendikulyar kesindisi ellipstiń tóbesi menen teń ekige bólinedi.

\textbf{466.4)} Tómendegilerdi bilgen halda ellips teńlemesin dúziń: onıń úlken kósheri 26 ģa teń hám fokusları $F_1 (-10; 0), F_2 (14; 0) $;

\textbf{466.4)} Tómendegilerdi bilgen halda ellips teńlemesin dúziń: onıń kishi kósheri 2 ge teń hám fokusları $F_1 (-1;-1) $, $F_2 (1; 1) $;

\textbf{466.4)} Tómendegilerdi bilgen halda ellips teńlemesin dúziń: onıń fokusları $F_1\left(-2; \frac{3}{2}\right), F_2\left(2;-\frac{3}{2}\right) $ hám ekssentrisitet $\varepsilon=\frac{\sqrt{2}}{2}$;

\textbf{473.4)} Tómendegilerdi bilgen halda ellips teńlemesin dúziń: onıń fokusları $F_1 (1; 3), F_2 (3; 1) $ hám direktrisalar arasındaģı aralıq $12 \sqrt{2}$ qa teń.

\textbf{477.} Eger ellipstiń ekssentrisiteti $\varepsilon=\frac{1}{2}$ hám fokusı $F (3; 0) $ hám oǵan sáykes direktrisa teńlemesi $x+y-1=0$ belgili bolsa, onıń teńlemesin dúziń.

\textbf{478.} $M_1 (2;-1) $ noqatı fokusı $F (1;0) $ bolǵan ellipste jatadı. Bul fokusqa sáykes direktrisa bolsa $2x-y-10=0$ teńleme menen berilgen. Usı ellipstiń teńlemesin dúziń.


\section{Giperbolanıń kanonikalıq teńlemesi}


\textbf{522.} $\frac{x^2}{80}-\frac{y^2}{20}=1$ giperbolada $M_1 (10;-\sqrt{5}) $ noqat berilgen. $M_1$ noqatınıń fokal radiusları jatqan tuwrı sızıqlardıń teńlemelerin dúziń.

\textbf{528.} $\frac{x^2}{64}-\frac{y^2}{36}=1$ giperbolanıń oń fokusına shekemgi aralıǵı 4,5 ke teń bolǵan noqatların anıqlań.

\textbf{533.} Teń tárepli giperbolanıń ekssentrisiteti anıqlansın.

\textbf{536.} Fokusları $\frac{x^2}{100}+\frac{y^2}{64}=1$ ellipstiń tóbelerinde jatıwshı, direktrisaları bolsa usı ellipstiń fokuslarınan ótiwshi giperbolanıń teńlemesin dúziń.

\textbf{543.1)} Tómendegilerdi bilgen halda giperbolanıń teńlemesin dúziń: onıń tóbeleri arasındaǵı aralıq 24 ke teń hám fokusları $F_1 (-10; 2), F_2 (16; 2) $;

\textbf{543.2) } Tómendegilerdi bilgen halda giperbolanıń teńlemesin dúziń: fokuslar $F_1 (3; 4), F_2 (-3;-4) $ hám direktrisalar arasındaǵı aralıq 3,6;

\textbf{547.} Tómendegilerdi bilgen halda giperbolanıń teńlemesin dúziń: Asimptotaları arasındaǵı múyesh $90^{\circ}$ qa teń hám fokuslar $F_1 (4;-4), F_2 (-2; 2) $.

\textbf{547.} Eger giperbolanıń ekssentrisiteti $\varepsilon=\sqrt{5}$, fokusı $F (2;-3) $ oǵan sáykes direktrisasınıń teńlemesi $3 x-y+3=0$ belgili bolsa, onıń teńlemesin dúziń

\textbf{548.} $M_1 (1;-2) $ noqat fokusı $F (-2; 2) $, oǵan sáykes direktrisa bolsa $2x-y-1=0$ teńleme menen berilgen giperbolaǵa tiyisli. Bul giperbolanıń teńlemesin dúziń.

\textbf{560.} $\frac{x^2}{16}-\frac{y^2}{64}=1$ giperbolaǵa $10 x-3 y+9=0$ tuwrısına parallel bolǵan urınbalardıń teńlemelerin dúziń.

\textbf{561.} $\frac{x^2}{16}-\frac{y^2}{8}=-1$ giperbolaǵa $2 x+4 y-5=0$ tuwrısına parallel urınbalar ótkiziń hám olar arasındaǵı $d$ aralıqtı esaplań.

\textbf{563.} $x^2-y^2=16$ giperbolaǵa $A (-1;-7)$ noqattan ótkerilgen urınbalar teńlemesin dúziń.



\section{ Parabolanıń kanonikalıq teńlemesi}



\textbf{600.} Eger parabolanıń fokusı $F (7; 2) $ hám direktrisa $x-5=0$ teńlemesi berilgen bolsa, onıń teńlemesin dúziń.

\textbf{601.} Eger parabolanıń fokusı $F (4;3) $ hám direktrisa $x-1=0$ teńlemesi berilgen bolsa, onıń teńlemesin dúziń.

\textbf{602.} Eger parabolanıń fokusı $F(2;-1) $ hám direktrisa $x-y-1=0$ teńlemesi berilgen bolsa, onıń teńlemesin dúziń.

\textbf{603.} Berilgen parabola tóbesi $A (6;-3) $ hám onıń direktrisasınıń teńlemesi $3x-5y+1=0$ berilgen. Bul parabolanıń $F$ fokusın tabıń.

\textbf{604.} Parabola tóbesi $A (-2;-1) $ hám onıń direktrisasınıń teńlemesi $x+2y-1=0$ berilgen. Bul parabolanıń teńlemesin dúziń.

\textbf{613.} $y^2=8x$ parabolanıń $2x+2y-3=0$ tuwrısına parallel urınbasınıń teńlemesin dúziń.

\textbf{614.} $x^2=16y$ parabolanıń $2x+4y+7=0$ tuwrısına perpendikulyar bolǵan urınbasınıń teńlemesin dúziń.

\textbf{619.} $A (5;9) $ noqattan $y^2=5x$ parabolaǵa júrgizilgen urınbalardıń urınıw noqatların tutastırıwshı xordanıń teńlemesin dúziń.


\section{ Ekinshi tártipli sızıqtıń orayı }


\textbf{666.1)} Berilgen sızıqlar oraylıq ekenligin kórsetiń hám hárbir iymek sızıq ushın orayınıń koordinataların tabıń: $3x^2+5xy+y^2-8x-11y-7=0$.

\textbf{666.2)} Berilgen sızıqlar oraylıq ekenligin kórsetiń hám hárbir iymek sızıq ushın orayınıń koordinataların tabıń:$5 x^2+4 x y+2 y^2+20 x+20 y-18=0$;

\textbf{666.3)} Berilgen sızıqlar oraylıq ekenligin kórsetiń hám hárbir iymek sızıq ushın orayınıń koordinataların tabıń:$9 x^2-4 x y-7 y^2-12=0$;

\textbf{666.4)} Berilgen sızıqlar oraylıq ekenligin kórsetiń hám hárbir iymek sızıq ushın orayınıń koordinataların tabıń: $2 x^2-6 x y+5 y^2+22 x-36 y+11=0$.

\textbf{668.1)} Berilgen teńlemeler oraylıq iymek sızıqlar ekenligin kórsetiń hám hárbir teńlemeni koordinata basın orayģa kóshiriń: $3x^2-6xy+2y^2-4x+2y+1=0$.

\textbf{668.2)} Berilgen teńlemeler oraylıq iymek sızıqlar ekenligin kórsetiń hám hárbir teńlemeni koordinata basın orayģa kóshiriń: $6 x^2+4 x y+y^2+4 x-2 y+2=0$;

\textbf{668.3)} Berilgen teńlemeler oraylıq iymek sızıqlar ekenligin kórsetiń hám hárbir teńlemeni koordinata basın orayģa kóshiriń: $4 x^2+6 x y+y^2-10 x-10=0$;

\textbf{668.4)} Berilgen teńlemeler oraylıq iymek sızıqlar ekenligin kórsetiń hám hárbir teńlemeni koordinata basın orayģa kóshiriń:  $4 x^2+2 x y+6 y^2+6 x-10 y+9=0$.



\section{ Ekinshi tártipli oraylıq sızıq teńlemesin ápiwayı túrge keltiriw}

\textbf{673.1)} Berilgen teńlemeniń tipin anıqlań, koordinata kósherlerin parallel kóshiriw arqalı ápiwayı túrge keltiriń; qanday geometriyalıq obrazdı ańlatıwın anıqlań, eski hám jańa koordinata kósherlerine salıstırģanda sızılmada súwretleń: $4 x^2+9 y^2-40 x+36 y+100=0$;

\textbf{673.2)} Berilgen teńlemeniń tipin anıqlań, koordinata kósherlerin parallel kóshiriw arqalı ápiwayı túrge keltiriń; qanday geometriyalıq obrazdı ańlatıwın anıqlań, eski hám jańa koordinata kósherlerine salıstırģanda sızılmada súwretleń: $9 x^2-16 y^2-54 x-64 y-127=0$;

\textbf{673.3)} Berilgen teńlemeniń tipin anıqlań, koordinata kósherlerin parallel kóshiriw arqalı ápiwayı túrge keltiriń; qanday geometriyalıq obrazdı ańlatıwın anıqlań, eski hám jańa koordinata kósherlerine salıstırģanda sızılmada súwretleń:  $9 x^2+4 y^2+18 x-8 y+49=0$;

\textbf{673.4)} Berilgen teńlemeniń tipin anıqlań, koordinata kósherlerin parallel kóshiriw arqalı ápiwayı túrge keltiriń; qanday geometriyalıq obrazdı ańlatıwın anıqlań, eski hám jańa koordinata kósherlerine salıstırģanda sızılmada súwretleń: $4 x^2-y^2+8 x-2 y+3=0$;

\textbf{673.5)} Berilgen teńlemeniń tipin anıqlań, koordinata kósherlerin parallel kóshiriw arqalı ápiwayı túrge keltiriń; qanday geometriyalıq obrazdı ańlatıwın anıqlań, eski hám jańa koordinata kósherlerine salıstırģanda sızılmada súwretleń: $2 x^2+3 y^2+8 x-6 y+11=0$.

\textbf{674.1)} Berilgen teńlemeni ápiwayı túrge keltiriń; tipin anıqlań; qanday geometriyalıq obrazdı ańlatıwın anıqlań, eski hám de jańa koordinata kósherlerine qarata sızılmada súwretleń: $32 x^2+52 x y-7 y^2+180=0$;

\textbf{674.2)} Berilgen teńlemeni ápiwayı túrge keltiriń; tipin anıqlań; qanday geometriyalıq obrazdı ańlatıwın anıqlań, eski hám de jańa koordinata kósherlerine qarata sızılmada súwretleń: $5 x^2-6 x y+5 y^2-32=0$;

\textbf{674.3)} Berilgen teńlemeni ápiwayı túrge keltiriń; tipin anıqlań; qanday geometriyalıq obrazdı ańlatıwın anıqlań, eski hám de jańa koordinata kósherlerine qarata sızılmada súwretleń: $17 x^2-12 x y+8 y^2=0$;

\textbf{674.4)} Berilgen teńlemeni ápiwayı túrge keltiriń; tipin anıqlań; qanday geometriyalıq obrazdı ańlatıwın anıqlań, eski hám de jańa koordinata kósherlerine qarata sızılmada súwretleń:  $5 x^2+24 x y-5 y^2=0$;

\textbf{674.5)} Berilgen teńlemeni ápiwayı túrge keltiriń; tipin anıqlań; qanday geometriyalıq obrazdı ańlatıwın anıqlań, eski hám de jańa koordinata kósherlerine qarata sızılmada súwretleń: $5 x^2-6 x y+5 y^2+8=0$.



\section{ Parabolik teńlemeni ápiwayı túrge keltiriw}



\textbf{689.1)} Berilgen teńleme parabolik ekenligin kórsetiń; ápiwayı túrge keltiriń; qanday geometriyalıq obrazdı anlatıwın anıqlań, eski hám de jańa koordinata kósherlerine salıstırģanda sızılmada súwretleń:$9 x^2-24 x y+16 y^2-20 x+110 y-50=0$;

\textbf{689.2)} Berilgen teńleme parabolik ekenligin kórsetiń; ápiwayı túrge keltiriń; qanday geometriyalıq obrazdı anlatıwın anıqlań, eski hám de jańa koordinata kósherlerine salıstırģanda sızılmada súwretleń:$9 x^2+12 x y+4 y^2-24 x-16 y+3=0$;

\textbf{689.3)} Berilgen teńleme parabolik ekenligin kórsetiń; ápiwayı túrge keltiriń; qanday geometriyalıq obrazdı anlatıwın anıqlań, eski hám de jańa koordinata kósherlerine salıstırģanda sızılmada súwretleń:$16 x^2-24 x y+9 y^2-160 x+120 y+425=0$.

\textbf{690.1)} Berilgen teńleme parabolik ekenligin kórsetiń; ápiwayı túrge keltiriń; qanday geometriyalıq obrazdı anlatıwın anıqlań, eski hám de jańa koordinata kósherlerine salıstırģanda sızılmada súwretleń: $9 x^2+24 x y+16 y^2-18 x+226 y+209=0$;

\textbf{690.2)} Berilgen teńleme parabolik ekenligin kórsetiń; ápiwayı túrge keltiriń; qanday geometriyalıq obrazdı anlatıwın anıqlań, eski hám de jańa koordinata kósherlerine salıstırģanda sızılmada súwretleń: $x^2-2 x y+y^2-12 x+12 y-14=0$

\textbf{690.3)} Berilgen teńleme parabolik ekenligin kórsetiń; ápiwayı túrge keltiriń; qanday geometriyalıq obrazdı anlatıwın anıqlań, eski hám de jańa koordinata kósherlerine salıstırģanda sızılmada súwretleń:$4 x^2+12 x y+9 y^2-4 x-6 y+1=0$.

\textbf{693.1)} Berilgen teńlemelerdiń parabolik ekenligin kórsetiń hám olardıń hár birin $(\alpha x+\beta y)^2+2 a_{13} x+2 a_{23} y+a_{33}=0$ kórinisinde jazıń: $x^2+4 x y+4 y^2+4 x+y-15=0 ;$

\textbf{693.2)} Berilgen teńlemelerdiń parabolik ekenligin kórsetiń hám olardıń hár birin $(\alpha x+\beta y)^2+2 a_{13} x+2 a_{23} y+a_{33}=0$ kórinisinde jazıń: $9 x^2-6 x y+y^2-x+2 y-14=0$;

\textbf{693.3)} Berilgen teńlemelerdiń parabolik ekenligin kórsetiń hám olardıń hár birin $(\alpha x+\beta y)^2+2 a_{13} x+2 a_{23} y+a_{33}=0$ kórinisinde jazıń: $25 x^2-20 x y+4 y^2+3 x-y+11=0$;

\textbf{693.4)} Berilgen teńlemelerdiń parabolik ekenligin kórsetiń hám olardıń hár birin $(\alpha x+\beta y)^2+2 a_{13} x+2 a_{23} y+a_{33}=0$ kórinisinde jazıń:  $16 x^2+16 x y+4 y^2-5 x+7 y=0$;

\textbf{693.5)} Berilgen teńlemelerdiń parabolik ekenligin kórsetiń hám olardıń hár birin $(\alpha x+\beta y)^2+2 a_{13} x+2 a_{23} y+a_{33}=0$ kórinisinde jazıń:  $9 x^2-42 x y+49 y^2+3 x-2 y-24=0$.



\section{ Ekinshi tártipli betler}



\textbf{1157.} $\frac{x^2}{12}+\frac{y^2}{4}+\frac{z^2}{3}=1$ ellipsoidı hám $2x-3y+4z-11=0$ tegisliginiń kesilisiw sızıǵı qanday iymek sızıq yekenligin anıqlań hám onıń orayın tabıń.

\textbf{1158.} $\frac{x^2}{2}-\frac{z^2}{3}=y$ giperbolik paraboloidi hám $3x-3y+4z+2=0$ tegisliginiń kesilisiw sızıǵı qanday iymek sızıq ekenligin anıqlań hám onıń orayın tabıń.
