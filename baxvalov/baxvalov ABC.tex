\section{A}

\textbf{473.} $\frac{x^2}{36}+\frac{y^2}{20}=1$ ellips direktrisalarining tenglamalarini yozing.
\textbf{483.} $\frac{x^2}{32}+\frac{y^2}{18}=1$ elipsning $M(4,3)$ nuqtasida o'tkazilgan urinmasining tenglamasi tuzilsin.
\textbf{531.1)} Quyidagi malumotlarga ko'ra giperbolaning kanonik tenglamasi tuzilsin: haqiqiy o'qi $a=5$ mavhum o'qi $b=3$;
\textbf{531.2)} Quyidagi malumotlarga ko'ra giperbolaning kanonik tenglamasi tuzilsin: fokuslari orasidagi masofa 10 ga , haqiqiy o'qi esa 8 ga teng.
\textbf{532.1)} Quyidagi malumotlarga ko'ra giperbolaning kanonik tenglamasi tuzilsin: ekssentrisiteti $e=\frac{12}{13}$ haqiqiy o'qi 48 ga teng.
\textbf{532.2)} Quyidagi malumotlarga ko'ra giperbolaning kanonik tenglamasi tuzilsin: haqiqiy o'qi 16 ga, asimptotasi bilan abssissa o'qi orasidagi $\varphi$.
\textbf{533.} Teng tomonli giperbolaning ekssentrisiteti hisoblansin.
\textbf{534.} Giperbola asimptotalarining tenglamalari $y= \pm \frac{5}{12} x$ va giperbolada yotuvchi $M(24,5)$ nuqta berilgan. Giperbola tenglamasi tuzilsin.
\textbf{535.} $\frac{x^2}{25}-\frac{y^2}{144}=1$ giperbolaning fokuslarini aniqlang.
\textbf{536.} $\frac{x^2}{225}-\frac{y^2}{64}=-1$ giperbolaning fokuslarini aniqlang.
\textbf{537.1)} Quyidagi malumotlarga ko'ra giperbolaning kanonik tenglamasi tuzilsin: direktrisalari orasidagi masofa $\frac{32}{5}$ ga teng va ekssentrisiteti $e=\frac{5}{4}$;
\textbf{537.2)} Quyidagi malumotlarga ko'ra giperbolaning kanonik tenglamasi tuzilsin: asimptotalari orasidagi burchak $60^{\circ}$ ga teng va $c=2 \sqrt{3}$ giperbolaning kanonik tenglamasi tuzilsin
\textbf{627.} $y^2=4 x$ parabola fokusining koordinatalarini aniqlang.
\textbf{628.} $x^2=4 y$ parabola fokusining koordinatalarini aniqlang.
\textbf{629.} $y^2=-8 x$ parabola fokusining koordinatalarini aniqlang.
\textbf{630.} $y^2=6 x$ parabola direktrisasi tenglamasini tuzing.
\textbf{635.1)} Parabolaning tenglamasini tuzing agar: parabolaning uchidan fokusigacha bo'lgan masofa 3 ga teng va parabola $O x$ o'qiga nisbatan simmetrik bo'lib, $O y$ o'qiga urinsa;
\textbf{635.2)} Parabolaning tenglamasini tuzing agar: fokusi $(5,0)$ nuqtada bo'lib, ordinatalar o'qi direktrisa bo'lsa;
\textbf{635.3)} Parabolaning tenglamasini tuzing agar: parabola $O x$ o'qiga nisbatan simmetrik bo'lib, $M(1 ;-4)$ nuqtadan va koordinatalar boshidan o'tadi;
\textbf{635.4)} Parabolaning tenglamasini tuzing agar: parabolaning fokusi $(0,2)$ nuqtada va uchi koordiniatalar boshida yotadi;
\textbf{635.5)} Parabolaning tenglamasini tuzing agar: parabola $O y$ o'qiga nisbatan simmetrik bo'lib, $M(6,-2)$ nuqtadan va koordinatalar boshidan o'tadi.
\textbf{636.} $y^2=8 x$ paraboladagi fokal radius vektori 20 ga teng bo'lgan nuqta topilsin.


\section{B}


\textbf{470.} O'qlari koordinata o'qlari bilan ustma - ust tushuvchi va $P(2,2) ; Q(3,1)$ nuqtalar orqali o'tuvchi ellips tenglamasi tuzilsin.
\textbf{471.} Katta o'qi 2 birlikka teng, fokuslari $F_1(0,1), F_2(1,0)$ nuqtalarda bo'lgan ellipsning tenglamasi tuzilsin.
\textbf{472.} Ellips fokuslarining biridan katta o'qi uchlarigacha masofalar mos ravishda 7 va 1 ga teng. Bu ellips ning tenglamasini tuzing.
\textbf{485.} $\frac{x^2}{16}+\frac{y^2}{9}=1$ ellipsning $x+y-1=0$ to'g'ri chiziqqa parallel bo'lgan urinmalarini aniqlang.
\textbf{538.} Giperbolaning haqiqiy o'qiga perpendikular bo'lgan va giperbola fokusidan o'tgan vatar uzunligi topilsin.
\textbf{539.} $\frac{x^2}{49}+\frac{y^2}{24}=1$ ellips bilan fokusdosh va ekssentrisiteti $e=\frac{5}{4}$ bo'lgan giperbolaning tenglamasi yozilsin.
\textbf{543.1)} Giperbolaning yarim o'qlarini toping, agar: fokuslari orasidagi masofa 8 ga va direktrisalari orasidagi masofa 6 ga teng;
\textbf{543.2)} Giperbolaning yarim o'qlarini toping, agar: direktrisalari $x= \pm 3 \sqrt{2}$ tenglamalar bilan berilgan va asimptotalari orasidagi burchak - to'g'ri burchak;
\textbf{543.3)} Giperbolaning yarim o'qlarini toping, agar: asimptotalari $y= \pm 2 x$ tenglamalar bilan berilgan va fokuslari markazdan 5 birlik masofada;
\textbf{543.4)} Giperbolaning yarim o'qlarini toping, agar: asimptotalari $y= \pm \frac{5}{3} x$ tenglamalar bilan berilgan va giperbola $N(6,9)$ nuqtadan o'tadi.
\textbf{545.1)} Giperbolaning asimptotalari orasidagi burchagi topilsin, agar: ekssentrisiteti $e=2$;
\textbf{545.2)} Giperbolaning asimptotalari orasidagi burchagi topilsin, agar: fokuslari orasidagi masofa direktrisalari orasidagi masofadan ikki marta katta.
\textbf{547.1)} $\frac{x^2}{16}-\frac{y^2}{9}=1$ giperbolada fokal radiuslari o'zaro perpendikular bo'lgan nuqta topilsin.
\textbf{547.2)} $\frac{x^2}{16}-\frac{y^2}{9}=1$ giperbolada chap fokusgacha bo'lgan masofasi o'ng fokusgacha bo'lgan nuqta topilsin.
\textbf{549.} $\frac{x^2}{9}-\frac{y^2}{4}=1$ giperbolaning $M(5,1)$ nuqtada teng ikkiga bo'linadigan vatarining tenglamasi tuzilsin.
\textbf{552.} $\frac{x^2}{5}-\frac{y^2}{4}=1$ giperbolaga $(5,-4)$ nuqtada urinadigan to'g'ri chiziq tenglamasi yozilsin.
\textbf{553.} $x^2-y^2=8$ giperbolaga $M(3,-1)$ nuqtada urinadigan to'g'ri chiziq tenglamasi yozilsin.
\textbf{654.1)} Parabola uchining koordinatalari, parametri va o'qining yo'nalishi aniqlansin: $y^2-10 x-2 y-19=0$;
\textbf{654.2)} Parabola uchining koordinatalari, parametri va o'qining yo'nalishi aniqlansin: $y^2-6 x+14 y+49=0$,
\textbf{654.3)} Parabola uchining koordinatalari, parametri va o'qining yo'nalishi aniqlansin: $y^2+8 x-16=0$,
\textbf{654.4)} Parabola uchining koordinatalari, parametri va o'qining yo'nalishi aniqlansin: $x^2-6 x-4 y+29=0$,
\textbf{654.5)} Parabola uchining koordinatalari, parametri va o'qining yo'nalishi aniqlansin: $y=A x^2+B x+C$,
\textbf{654.6)} Parabola uchining koordinatalari, parametri va o'qining yo'nalishi aniqlansin: $y=x^2-8 x+15$,
\textbf{654.7)} Parabola uchining koordinatalari, parametri va o'qining yo'nalishi aniqlansin: $y=x^2+6 x$.
\textbf{687.} Beshta nuqtadan o'tuvchi ikkinchi tartibli chiziqning tenglamasi tuzilsin: $(0,0),(0,1),(1,0),(2,-5),(-5,2)$.
\textbf{689.} $5 x^2-3 x y+y^2-3 x+2 y-5=0$ chiziqning $x-2 y-1=0$ to'g'ri chiziq bilan kesishishidan hosil qilingan vatarning o'rtasidan o'tadigan diametr tenglamasi yozilsin.
\textbf{722.} ITECH turi, o'lchovlari va joylashishi aniqlansin: $5 x^2+4 x y+8 y^2-32 x-56 y+80=0$.
\textbf{723.} ITECH turi, o'lchovlari va joylashishi aniqlansin: $9 x^2+24 x y+16 y^2-230 x+110 y-475=0$.
\textbf{724.} ITECH turi, o'lchovlari va joylashishi aniqlansin: $5 x^2+12 x y-12 x-22 y-19=0$.
\textbf{725.} ITECH turi, o'lchovlari va joylashishi aniqlansin: $x^2-2 x y+y^2-10 x-6 y+25=0$.
\textbf{726.} ITECH turi, o'lchovlari va joylashishi aniqlansin: $x^2-5 x y+4 y^2+x+2 y-2=0$.
\textbf{727.} ITECH turi, o'lchovlari va joylashishi aniqlansin: $4 x^2-12 x y+9 y^2-2 x+3 y-2=0$.
\textbf{728.1)} ITECH turi, o'lchovlari va joylashishi aniqlansin: $2 x^2+4 x y+5 y^2-6 x-8 y-1=0$;
\textbf{728.2)} ITECH turi, o'lchovlari va joylashishi aniqlansin: $5 x^2+8 x y+5 y^2-18 x-18 y+9=0$;
\textbf{728.3)} ITECH turi, o'lchovlari va joylashishi aniqlansin: $5 x^2+6 x y+5 y^2-16 x-16 y-16=0$;
\textbf{728.4)} ITECH turi, o'lchovlari va joylashishi aniqlansin: $6 x y-8 y^2+12 x-26 y-11=0$;
\textbf{728.5)} ITECH turi, o'lchovlari va joylashishi aniqlansin: $7 x^2+16 x y-23 y^2-14 x-16 y-218=0$;
\textbf{728.6)} ITECH turi, o'lchovlari va joylashishi aniqlansin: $7 x^2-24 x y-38 x+24 y+175=0$;
\textbf{728.7)} ITECH turi, o'lchovlari va joylashishi aniqlansin: $9 x^2+24 x y+16 y^2-40 x-30 y=0$;
\textbf{728.8)} ITECH turi, o'lchovlari va joylashishi aniqlansin: $x^2+2 x y+y^2-8 x+4=0$;
\textbf{728.9)} ITECH turi, o'lchovlari va joylashishi aniqlansin: $4 x^2-4 x y+y^2-2 x-14 y+7=0$.
\textbf{1752.1)} Lagranj usulidan foydalanib, tenglamalarni kvadratlar yig'indisi shakliga keltirib, quyidagi sirtlarning ko'rinishi aniqlansin: $4 x^2+6 y^2+4 z^2+4 x z-8 y-4 z+3=0$;
\textbf{1752.2)} Lagranj usulidan foydalanib, tenglamalarni kvadratlar yig'indisi shakliga keltirib, quyidagi sirtlarning ko'rinishi aniqlansin: $x^2+5 y^2+z^2+2 x y+6 x z+2 y z-2 x+6 y-10 z=0$;
\textbf{1752.3)} Lagranj usulidan foydalanib, tenglamalarni kvadratlar yig'indisi shakliga keltirib, quyidagi sirtlarning ko'rinishi aniqlansin: $x^2+y^2-3 z^2-2 x y-6 x z-6 y z+2 x+2 y+4 z=0$;
\textbf{1752.4)} Lagranj usulidan foydalanib, tenglamalarni kvadratlar yig'indisi shakliga keltirib, quyidagi sirtlarning ko'rinishi aniqlansin: $x^2-2 y^2+z^2+4 x y-8 x z-4 y z-14 x-4 y+14 z+16=0$;
\textbf{1752.5)} Lagranj usulidan foydalanib, tenglamalarni kvadratlar yig'indisi shakliga keltirib, quyidagi sirtlarning ko'rinishi aniqlansin: $2 x^2+y^2+2 z^2-2 x y-2 y z+x-4 y-3 z+2=0$;
\textbf{1752.6)} Lagranj usulidan foydalanib, tenglamalarni kvadratlar yig'indisi shakliga keltirib, quyidagi sirtlarning ko'rinishi aniqlansin: $x^2-2 y^2+z^2+4 x y-10 x z+4 y z+x+y-z=0$;
\textbf{1752.7)} Lagranj usulidan foydalanib, tenglamalarni kvadratlar yig'indisi shakliga keltirib, quyidagi sirtlarning ko'rinishi aniqlansin: $2 x^2+y^2+2 z^2-2 x y-2 y z+4 x-2 y=0$;
\textbf{1752.8)} Lagranj usulidan foydalanib, tenglamalarni kvadratlar yig'indisi shakliga keltirib, quyidagi sirtlarning ko'rinishi aniqlansin: $x^2-2 y^2+z^2+4 x y-10 x z+4 y z+2 x+4 y-10 z-1=0$;
\textbf{1752.9)} Lagranj usulidan foydalanib, tenglamalarni kvadratlar yig'indisi shakliga keltirib, quyidagi sirtlarning ko'rinishi aniqlansin: $x^2+y^2+4 z^2+2 x y+4 x z+4 y z-6 z+1=0$;
\textbf{1752.10)} Lagranj usulidan foydalanib, tenglamalarni kvadratlar yig'indisi shakliga keltirib, quyidagi sirtlarning ko'rinishi aniqlansin: $4 x y+2 x+4 y-6 z-3=0$;
\textbf{1752.11)} Lagranj usulidan foydalanib, tenglamalarni kvadratlar yig'indisi shakliga keltirib, quyidagi sirtlarning ko'rinishi aniqlansin: $x y+x z+y z+2 x+2 y-2 z=0$.
\textbf{1753.1)} Parallel ko'chirish va burish almashtirishlari yoki hadlarni gruppalash yordamida quyidagi sirtlarning ko'rinishi va joylashishi aniqlansin: $z=2 x^2-4 y^2-6 x+8 y+1$;
\textbf{1753.2)} Parallel ko'chirish va burish almashtirishlari yoki hadlarni gruppalash yordamida quyidagi sirtlarning ko'rinishi va joylashishi aniqlansin: $z=x^2+3 y^2-6 y+1$;
\textbf{1753.3)} Parallel ko'chirish va burish almashtirishlari yoki hadlarni gruppalash yordamida quyidagi sirtlarning ko'rinishi va joylashishi aniqlansin: $x^2+2 y^2-3 z^2+2 x+4 y-6 z=0$;
\textbf{1753.4)} Parallel ko'chirish va burish almashtirishlari yoki hadlarni gruppalash yordamida quyidagi sirtlarning ko'rinishi va joylashishi aniqlansin: $x^2+2 x y+y^2-z^2=0$;
\textbf{1753.5)} Parallel ko'chirish va burish almashtirishlari yoki hadlarni gruppalash yordamida quyidagi sirtlarning ko'rinishi va joylashishi aniqlansin: $z^2=3 x+4 y+5$;
\textbf{1753.6)} Parallel ko'chirish va burish almashtirishlari yoki hadlarni gruppalash yordamida quyidagi sirtlarning ko'rinishi va joylashishi aniqlansin: $z=x^2+2 x y+y^2+1$;
\textbf{1753.7)} Parallel ko'chirish va burish almashtirishlari yoki hadlarni gruppalash yordamida quyidagi sirtlarning ko'rinishi va joylashishi aniqlansin: $z^2=x^2+2 x y+y^2+1$;
\textbf{1753.8)} Parallel ko'chirish va burish almashtirishlari yoki hadlarni gruppalash yordamida quyidagi sirtlarning ko'rinishi va joylashishi aniqlansin: $x^2+4 y^2+9 z^2-6 x+8 y-18 z-14=0$;
\textbf{1753.9)} Parallel ko'chirish va burish almashtirishlari yoki hadlarni gruppalash yordamida quyidagi sirtlarning ko'rinishi va joylashishi aniqlansin: $2 x y+z^2-2 z+1=0$;
\textbf{1753.10)} Parallel ko'chirish va burish almashtirishlari yoki hadlarni gruppalash yordamida quyidagi sirtlarning ko'rinishi va joylashishi aniqlansin: $x^2+y^2-z^2-2 x y+2 z-1=0$;
\textbf{1753.11)} Parallel ko'chirish va burish almashtirishlari yoki hadlarni gruppalash yordamida quyidagi sirtlarning ko'rinishi va joylashishi aniqlansin: $x^2+4 y^2-z^2-10 x-16 y+6 z+16=0$;
\textbf{1753.12)} Parallel ko'chirish va burish almashtirishlari yoki hadlarni gruppalash yordamida quyidagi sirtlarning ko'rinishi va joylashishi aniqlansin: $2 x y+2 x+2 y+2 z-1=0$;
\textbf{1753.13)} Parallel ko'chirish va burish almashtirishlari yoki hadlarni gruppalash yordamida quyidagi sirtlarning ko'rinishi va joylashishi aniqlansin: $3 x^2+6 x-8 y+6 z-7=0$;
\textbf{1753.14)} Parallel ko'chirish va burish almashtirishlari yoki hadlarni gruppalash yordamida quyidagi sirtlarning ko'rinishi va joylashishi aniqlansin: $x^2+y^2+2 z^2+2 x y+4 z=0$;
\textbf{1753.15)} Parallel ko'chirish va burish almashtirishlari yoki hadlarni gruppalash yordamida quyidagi sirtlarning ko'rinishi va joylashishi aniqlansin: $3 x^2+3 y^2+3 z^2-6 x+4 y-1=0$;
\textbf{1753.16)} Parallel ko'chirish va burish almashtirishlari yoki hadlarni gruppalash yordamida quyidagi sirtlarning ko'rinishi va joylashishi aniqlansin: $3 x^2+3 y^2-6 x+4 y-1=0$;
\textbf{1753.17)} Parallel ko'chirish va burish almashtirishlari yoki hadlarni gruppalash yordamida quyidagi sirtlarning ko'rinishi va joylashishi aniqlansin: $3 x^2+3 y^2-3 z^2-6 x+4 y+4 z+3=0$;
\textbf{1753.18)} Parallel ko'chirish va burish almashtirishlari yoki hadlarni gruppalash yordamida quyidagi sirtlarning ko'rinishi va joylashishi aniqlansin: $4 x^2-y^2-4 x+4 y-3=0$;


\section{C}


\textbf{477.} $\frac{x^2}{a^2}+\frac{y^2}{b^2}=1$ ellipsning $F(c, 0)$ fokusi orqali katta o'qiga perpendikular bo'lgan vatar o'tkazilgan. Bu vatar uzunligini toping.
\textbf{478.} $\frac{x^2}{a^2}+\frac{y^2}{b^2}=1$ ellipsga ichki chizilgan kvadrat tomonining uzunligi hisoblansin.
\textbf{480.} $\frac{x^2}{100}+\frac{y^2}{64}=1$ ellipsning $2 x-y+7=0,2 x-y-1=0$ vatarlarining o'rtalari orqali o'tadigan to'g'ri chiziq tenglamasini tuzing.
\textbf{490.} $A x+B y+C=0$ to'g'ri chiziqning $\frac{x^2}{a^2}+\frac{y^2}{b^2}=1$, ellipsga urinma bo'lishi uchun zaruriy va yetarli sharti topilsin.
\textbf{510.}* $A x+B y+C=0$ to'g'ri chiziq qanday zaruriy va yetarli shart bajarilganda $\frac{x^2}{a^2}+\frac{y^2}{b^2}=1$ ellips bilan 1) kesishadi; 2) kesishmaydi.
\textbf{541.} Giperbolaning asimptotalaridan direktrisalari ajratgan kesmalar (giperbolaning markazidan hisoblanganda) giperbolaning haqiqiy yarim o'qiga teng ekanligi isbotlansin. Bu xossadan foydalanib, giperbolaning direktrisalari yasalsin.
\textbf{556.} $\frac{x^2}{a^2}-\frac{y^2}{b^2}=1$ giperbolaning fokuslaridan urinmasigacha bo'lgan masofalarning ko'paytmasi topilsin.
\textbf{566.} Giperbola asimptotalarining tenglamalari $y= \pm \frac{1}{2} x$ va urinmalardan birining tenglamasi $5 x-6 y-8=0$ ma'lum bo'lsa, giperbola tenglamasini tuzing.
\textbf{641.} $A x+B y+C=0$ to'g'ri chiziq $y^2=2 p x$ parabolaga urinishi uchun zaruriy va yetarli shartni toping.
\textbf{642.} Berilgan $y=k x+b$ to'g'ri chiziqqa parallel va $y^2=2 p x$ parabolaga urinadigan to'g'ri chiziqning tenglamasini yozing.
\textbf{643.}* $y^2=4 x$ parabola bilan $\frac{x^2}{8}+\frac{y^2}{2}=1$ ellipsning umumiy urinmalarini aniqlang.
\textbf{650.} Parabolaning ix'tiyoriy urinmasi direktrisasini va o'qqa perpendikular bo'lgan fokal vatarni fokusdan teng uzoqlikdagi nuqtalarda kesishini isbotlang.
\textbf{655.}* Umumiy fokusga va ustma - ust tushgan, lekin qarama - qarshi yo'nalgan o'qlarga ega bo'lgan parabolalarning to'g'ri burchak ostida kesishishi isbotlansin.
\textbf{698.} Giperbolaning asimptotalari topilsin: $10 x^2+21 x y+9 y^2-41 x-39 y+4=0$.
\textbf{699.1)} Giperbolaning asimptotalari topilsin: $x^2-3 x y-10 y^2+6 x-8 y=0$;
\textbf{699.2)} Giperbolaning asimptotalari topilsin: $3 x^2+2 x y-y^2+8 x+10 y-14=0$;
\textbf{699.3)} Giperbolaning asimptotalari topilsin: $3 x^2+7 x y+4 y^2+5 x+2 y-6=0$;
\textbf{699.4)} Giperbolaning asimptotalari topilsin: $10 x y-2 y^2+6 x+4 y+21=0$
\textbf{1754.} Quyidagi sirtlarning kanonik tenglamasi va joylashishini aniqlansin: $x^2+5 y^2+z^2+2 x y+6 x z+2 y z-2 x+6 y+2 z=0$.
\textbf{1755.} Quyidagi sirtlarning kanonik tenglamasi va joylashishini aniqlansin: $2 x^2+y^2+2 z^2-2 x y+2 y z+4 x-2 y=0$.
\textbf{1756.} Quyidagi sirtlarning kanonik tenglamasi va joylashishini aniqlansin: $x^2+y^2+4 z^2+2 x y+4 x z+4 y z-6 z+1=0$.
\textbf{1757.} Quyidagi sirtlarning kanonik tenglamasi va joylashishini aniqlansin: $4 x^2+9 y^2+z^2-12 x y-6 y z+4 z x+4 x-6 y+2 z-5=0$.
\textbf{1758.} Quyidagi sirtlarning kanonik tenglamasi va joylashishini aniqlansin: $7 x^2+6 y^2+5 z^2-4 x y-4 y z-6 x-24 y+18 z+30=0$.
\textbf{1759.} Quyidagi sirtlarning kanonik tenglamasi va joylashishini aniqlansin: $2 x^2+2 y^2-5 z^2+2 x y-2 x-4 y-4 z+2=0$.
\textbf{1760.} Quyidagi sirtlarning kanonik tenglamasi va joylashishini aniqlansin: $x^2-2 y^2+z^2+4 x y-8 x z-4 y z-14 x-4 y+14 z+16=0$.
\textbf{1761.} Quyidagi sirtlarning kanonik tenglamasi va joylashishini aniqlansin: $2 x^2+2 y^2+3 z^2+4 x y+2 x z+2 y z-4 x+6 y-2 z+3=0$.
\textbf{1762.} Quyidagi sirtlarning kanonik tenglamasi va joylashishini aniqlansin: $2 x^2+5 y^2+2 z^2-2 x y+2 y z-4 x z+2 x-10 y-2 z-1=0$.
\textbf{1763.1)} Quyidagi sirtlarning kanonik tenglamasi va joylashishini aniqlansin: $x^2+5 y^2+z^2+2 x y+6 x z+2 y z-2 x+6 y+2 z=0$;
\textbf{1763.2)} Quyidagi sirtlarning kanonik tenglamasi va joylashishini aniqlansin: $5 x^2+2 y^2+5 z^2-4 x y-2 x y-4 y z+10 x-4 y-2 z+4=0$;
\textbf{1763.3)} Quyidagi sirtlarning kanonik tenglamasi va joylashishini aniqlansin: $x^2-2 y^2+z^2+4 x y-10 x z+4 y z+2 x+4 y-10 z-1=0$.
\textbf{1764.} Quyidagi sirtlarning kanonik tenglamasi va joylashishini aniqlansin: $5 x^2-y^2+z^2+4 x y+6 x z+2 x+4 y+6 z-8=0$.
\textbf{1765.} Quyidagi sirtlarning kanonik tenglamasi va joylashishini aniqlansin: $2 x^2+10 y^2-2 z^2+12 x y+8 y z+12 x+4 y+8 z-1=0$.
