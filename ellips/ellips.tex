444.1) Составить уравнение эллипса, фокусы которого лежат на оси абсцисс, симметрично относительно начала координат, зная, кроме того, что: его полуоси равны 5 и 2 ;
444.2) Составить уравнение эллипса, фокусы которого лежат на оси абсцисс, симметрично относительно начала координат, зная, кроме того, что: его большая ось равна 10 , а расстояние между фокусами $2 c=8$
444.3) Составить уравнение эллипса, фокусы которого лежат на оси абсцисс, симметрично относительно начала координат, зная, кроме того, что: его малая ось равна 24 , а расстояние между фокусами $2 c=10$;
444.4) Составить уравнение эллипса, фокусы которого лежат на оси абсцисс, симметрично относительно начала координат, зная, кроме того, что: расстояние между его фокусами $2 c=6$ и эксцентриситет $\varepsilon=\frac{3}{5}$;
444.5) Составить уравнение эллипса, фокусы которого лежат на оси абсцисс, симметрично относительно начала координат, зная, кроме того, что: его большая ось равна 20, а эксцентриситет $\varepsilon=\frac{3}{5}$;
444.6) Составить уравнение эллипса, фокусы которого лежат на оси абсцисс, симметрично относительно начала координат, зная, кроме того, что: его малая ось равна 10 , а эксцентриситет $\varepsilon=\frac{12}{13}$;
444.7) Составить уравнение эллипса, фокусы которого лежат на оси абсцисс, симметрично относительно начала координат, зная, кроме того, что: расстояние между его директрисами равно 5 и расстояние между фокусами $2 c=4$;
444.8) Составить уравнение эллипса, фокусы которого лежат на оси абсцисс, симметрично относительно начала координат, зная, кроме того, что: его большая ось равна 8, а расстояние между директрисами равно 16 ;
444.9) Составить уравнение эллипса, фокусы которого лежат на оси абсцисс, симметрично относительно начала координат, зная, кроме того, что: его малая ось равна 6, а расстояние между дирек трисами равно 13 ;
444.10) Составить уравнение эллипса, фокусы которого лежат на оси абсцисс, симметрично относительно начала координат, зная, кроме того, что: расстояние между его директрисами равно 32 и $\varepsilon=\frac{1}{2}$.
445.1) Составить уравнение эллипса, фокусы которого лежат на оси ординат, симметрично относительно начала координат, зная, кроме того, что: его полуоси равны соответственно 7 и 2 ;
445.2) Составить уравнение эллипса, фокусы которого лежат на оси ординат, симметрично относительно начала координат, зная, кроме того, что: его большая ось равна 10 , а расстояние между фокусами $2 c=8$;
445.3) Составить уравнение эллипса, фокусы которого лежат на оси ординат, симметрично относительно начала координат, зная, кроме того, что: расстояние между его фокусами $2 c=24$ и эксцен• триситет $\varepsilon=\frac{12}{13}$;
445.4) Составить уравнение эллипса, фокусы которого лежат на оси ординат, симметрично относительно начала координат, зная, кроме того, что: его малая ось равна 16 , а эксцентриситет $\varepsilon=\frac{3}{5}$;
445.5) Составить уравнение эллипса, фокусы которого лежат на оси ординат, симметрично относительно начала координат, зная, кроме того, что: расстояние между его фокусами $2 c=6$ и расстояние между директрисами равно $16 \frac{2}{3}$;
445.6) Составить уравнение эллипса, фокусы которого лежат на оси ординат, симметрично относительно начала координат, зная, кроме того, что: расстояние между его директрисами равно $10 \frac{2}{3}$ и эксцентриситет $\varepsilon=\frac{3}{4}$.
446.1) Определить полуоси каждого из следующих эллипсов: $\frac{x^2}{16}+\frac{y^2}{9}=1$;
446.2) Определить полуоси каждого из следующих эллипсов: $\frac{x^2}{4}+y^2=1$;
446.3) Определить полуоси каждого из следующих эллипсов: $x^2+25 y^2=25$;
446.4) Определить полуоси каждого из следующих эллипсов: $x^2+5 y^2=15$;
446.5) Определить полуоси каждого из следующих эллипсов: $4 x^2+9 y^2=25$;
446.6) Определить полуоси каждого из следующих эллипсов: $9 x^2+25 y^2=1 ;$
446.7) Определить полуоси каждого из следующих эллипсов: $x^2+4 y^2=1$;
446.8) Определить полуоси каждого из следующих эллипсов: $16 x^2+y^2=16$
446.9) Определить полуоси каждого из следующих эллипсов: $25 x^2+9 y^2=1$;
446.10) Определить полуоси каждого из следующих эллипсов: $9 x^2+y^2=1$.
447. Дан эллипс $9 x^2+25 y^2=225$. Найти: 1) его полуоси; 2). фокусы; 3) эксцентриситет; 4) уравнения директрис.
448. Вычислить площадь четырехугольника, две вершины которого лежат в фокусах эллипса $x^2+5 y^2=20$, а две другие совпадают с концами его малой оси.
449. Дан эллипс $9 x^2+5 y^2=45$. Найти: 1) его полуоси; 2) фокусы; 3) эксцентриситет; 4) уравнения директрис.
450. Вычислить площадь четырехугольника, две вершины которого лежат в фокусах эллипса $9 x^2+5 y^2=1$, две другие совпадают с концами его малой оси.
451. Вычислить расстояние от фокуса $F(c ; 0)$ эллипса $\frac{x^2}{a^2}+\frac{y^2}{b^2}=1$ до односторонней с этим фокусом директрисы.
452. Пользуясь одним циркулем, построить фокусы эллипса $\frac{x^2}{16}+\frac{y^2}{9}=1$ (считая, что изображены осн координат и задана масштабная единица).
453. На эллипсе $\frac{x^2}{25}+\frac{y^2}{4}=1$ найти точки, абсцисса которых равна -3.
454. Определить, какие из точек $A_1(-2 ; 3), A_2(2 ;-2)$, $A_3(2 ;-4), A_4(-1 ; 3), A_5(-4 ;-3), A_6(3 ;-1), A_7(3 ;-2)$, $A_8(2 ; 1), A_9(0 ; 15)$ и $A_{10}(0 ;-16)$ лежат на эллипсе $8 x^2+5 y^2=77$, какие внутри и какие вне его.
455. Установить, какие линии определяются следующими уравнениями: 1) $y=+\frac{3}{4} \sqrt{16-x^2}$; 2) $y=$ $=-\frac{5}{3} \sqrt{9-x^2}$; 3) $x=-\frac{2}{3} \sqrt{9-y^2}$; 4) $x=+\frac{1}{7} \sqrt{49-y^2}$. Изобразить эти линии на чертеже.
456. Эксцентриситет эллипса $\varepsilon=\frac{2}{3}$, фокальный радиус точки $M$ эллипса равен 10 . Вычислить расстояние от точки $M$ до односторонней с этим фокусом директрисы.
457. Эксцентриситет эллипса $\varepsilon=\frac{2}{5}$, расстояние от точки $M$ эллипса до директрисы равно 20. Вычислить расстояние от точки $M$ до фокуса, одностороннего с этой директрисой.
458. Дана точка $M_1\left(2 ;-\frac{5}{3}\right)$ на эллипсе $\frac{x^2}{9}+\frac{y^2}{5}=1$; составить уравнения прямых, на которых лежат фокальные радиусы точки $M_1$.
459. Уббедившись, что точка $M_1(-4 ; 2,4)$ лежит на эллипсе $\frac{x^2}{25}+\frac{y^2}{16}=1$, определить фокальные радиусы точки $M_1$.
460. Эксцентриситет эллипса $\varepsilon=\frac{1}{3}$, центр его совпадает с началом координат, один из фокусов $(-2 ; 0)$. Вычиелить расстояние от точки $M_1$ эллипса с абсциссой, равной 2, до директрисы, односторонней с данным фокусом.
461. Эксцентриситет эллипса $\varepsilon=\frac{1}{2}$, центр его совпадает с началом координат, одна из директрис дана уравнением $x=16$. Вычислить расстояние от точки $M_1$ эллипса с абсциссой, равной -4, до фокуса, одностороннего с данной директрисой.
462. Определить точки эллипса $\frac{x^2}{100}+\frac{y^2}{36}=1$, pacстояние которых до правого фокуса равно 14.
463. Определить точки эллипса $\frac{x^2}{16}+\frac{y^2}{7}=1$, pacстояние которых до левого фокуса равно $2,5$.
464. Через фокус эллипса $\frac{x^2}{25}+\frac{y^2}{15}=1$ проведен перпендикуляр к его большой оси. Определить расстояния от точек пересечения этого перпе́ндикуляра с эллипсом до фокусов.
465.1) Составить уравнение эллипса, фокусы которого расположены на оси абсцисс, симметрично относительно начала координат, если даны: точка $M_1(-2 \sqrt{5} ; 2)$ эллипса и его малая полуось $b=3$;
465.2) Составить уравнение эллипса, фокусы которого расположены на оси абсцисс, симметрично относительно начала координат, если даны: точка $M_1(2 ;-2)$ эллипса и его большая полуось $a=4$;
465.3) Составить уравнение эллипса, фокусы которого расположены на оси абсцисс, симметрично относительно начала координат, если даны: точки $M_1(4 ;-\sqrt{3})$ и $M_2(2 \sqrt{2} ; 3)$ эллипса;
465.4) Составить уравнение эллипса, фокусы которого расположены на оси абсцисс, симметрично относительно начала координат, если даны: точка $M_1(\sqrt{15} ;-1)$ эллипса и расстояние между его фокусами $2 c=8$;
465.5) Составить уравнение эллипса, фокусы которого расположены на оси абсцисс, симметрично относительно начала координат, если даны: точка $M_1\left(2 ;-\frac{5}{3}\right)$ эллипса и его эксцентриситет $\varepsilon=\frac{2}{3}$;
465.6) Составить уравнение эллипса, фокусы которого расположены на оси абсцисс, симметрично относительно начала координат, если даны: точка $M_1(8 ; 12)$ эллипса и расстояние $r_1=20$ от нее до левого фокуса;
465.7) Составить уравнение эллипса, фокусы которого расположены на оси абсцисс, симметрично относительно начала координат, если даны: точка $M_1(-\sqrt{5} ; 2)$ эллипса и расстояние между его директрисами равно 10.
466.1) Определить эксцентриситет в эллипса, если: его малая ось видна из фокусов под углом в $60^{\circ}$;
466.2) Определить эксцентриситет в эллипса, если: отрезок между фоку сами виден из вєршин малой оси под прямым углом;
466.3) Определить эксцентриситет в эллипса, если: расстояние между директрисами в три раза больше расстояния между фокусами;
466.4) Определить эксцентриситет в эллипса, если: отрезок перпендикуляра, опущенного из центра эллипса на его директрису, делится вершиной эллипса пополам.
467. Через фокус $F$ эллипса проведен перпендикуляр к его большой оси (рис. 15). Определить, при каком значении эксцентриситета эллипса отрезки $\overline{A B}$ и $\overline{O C}$ будут параллельны,
468. Составить уравне்ние эллипса с полуосями $a, b$ и центром $C\left(x_0 ; y_0\right)$, если известно, что оси симметрии эллипса параллельны осям координат.
469. Эллипс касается оси абсцисс в точке $A(3 ; 0)$ и оси ординат в точке $B(0 ;-4)$. Составить уравнение этого эллипса, зная, что его оси симметрии паралллельны координатным осям.
470. Точка $C(-3 ; 2)$ является центром эллипса, касающегося обеих координатных осей. Составить уравнение этого эллипса, зная, что его оси симметрии параллельны координатным осям.
471.1) Установить, что каждое из следующих уравнений определяет эллипс, и найти координаты его центра $C$, полуоси, эксцентриситет и уравнения директрис: $5 x^2+9 y^2-30 x+18 y+9=0$;
471.2) Установить, что каждое из следующих уравнений определяет эллипс, и найти координаты его центра $C$, полуоси, эксцентриситет и уравнения директрис: $16 x^2+25 y^2+32 x-100 y-284=0$;
471.3) Установить, что каждое из следующих уравнений определяет эллипс, и найти координаты его центра $C$, полуоси, эксцентриситет и уравнения директрис: $4 x^2+3 y^2-8 x+12 y-32=0$.
472.1) Установить, какие линии определяются следующими уравнениями: $y=-7+\frac{2}{5} \sqrt{16+6 x-x^2}$
472.2) Установить, какие линии определяются следующими уравнениями: $y=1-\frac{4}{3} \sqrt{-6 x-x^2}$;
472.3) Установить, какие линии определяются следующими уравнениями: $x=-2 \sqrt{-5-6 y-y^2}$;
472.4) Установить, какие линии определяются следующими уравнениями: $x=-5+\frac{2}{3} \sqrt{8+2 y-y^2}$.
473.1) Составить уравнение эллипса, зная, что: его большая ось равна 26 и фокусы суть $F_1(-10 ; 0), F_2(14 ; 0)$;
473.2) Составить уравнение эллипса, зная, что: его малая ось равна 2 и фокусы суть $F_1(-1 ;-1)$, $F_2(1 ; 1)$;
473.3) Составить уравнение эллипса, зная, что: его фокусы суть $F_1\left(-2 ; \frac{3}{2}\right), F_2\left(2 ;-\frac{3}{2}\right)$ и эксцентриситет $\varepsilon=\frac{\sqrt{2}}{2}$;
473.4) Составить уравнение эллипса, зная, что: его фокусы суть $F_1(1 ; 3), F_2(3 ; 1)$ и расстояние между директрисами равно $12 \sqrt{2}$.
474. Составить уравнение эллипса, єсли известны его эксцентриситет $\varepsilon=\frac{2}{3}$, фокус $F(2 ; 1)$ и уравнение соответствующей директрисы $x-5=0$.
475. Составить уравнение эллипса, если известны его өксцентриситет $\varepsilon=\frac{1}{2}$, фокус $F(-4 ; 1)$ и уравнение соответствующей директрисы $y+3=0$.
476. Точка $A(-3 ;-5)$ лежит на эллипсе, фокус которого $F(-1 ;-4)$, а соответствующая директриса дана уравнением $x-2=0$. Составить уравнение этого эллипса.
477. Составить уравнение эллипса, если известны его эксцентриситет $\varepsilon=\frac{1}{2}$, фокус $F(3 ; 0)$ и уравнение соответствующей директрисы $x+y-1=0$.
478. Точка $M_1(2 ;-1)$ лежит на эллипсе, фокус которого $F(1 ; 0)$, а соответствующая директриса дана уравнением $2 x-y-10=0$. Составить уравнение этого эллипса.
479. Точка $M_1(3 ;-1)$ является концом малой оси эллипса, фокусы которого лежат на прямой $y+6=0$. Составить уравнение этого эллипса, зная его эксцентриситет $\varepsilon=\frac{\sqrt{2}}{2}$.
480. Найти точки пересечения прямой $x+2 y-7=0$ и эллипса $x^2+4 y^2=25$.
481. Найти точки пересечения прямой $3 x+10 y-25=0$ и эллипса $\frac{x^2}{25}+\frac{y^2}{4}=1$.
482. Найти точки пересечения прямой $3 x-4 y-40=0$ и эллипса $\frac{x^2}{16}+\frac{y^2}{9}=1$.
483.1) Определить, как расположена прямая относительно эллипса; пересекает ли, касается или проходит вне его, если прямая и эллипс заданы следующими уравнениямі: $2 x-y-3=0$, $\frac{x^2}{16}+\frac{y^2}{9}=1$.
483.2) Определить, как расположена прямая относительно эллипса; пересекает ли, касается или проходит вне его, если прямая и эллипс заданы следующими уравнениямі: $2 x+y-10=0$, $frac{x^2}{9}+\frac{y^2}{4}=1$.
483.3) Определить, как расположена прямая относительно эллипса; пересекает ли, касается или проходит вне его, если прямая и эллипс заданы следующими уравнениямі: $3 x+2 y-20=0$, $\frac{x^2}{40}+\frac{y^2}{10}=1$.
484. Определить, при каких значениях $m$ прямая $y=-x+m$ 1) пересекает эллипс $\frac{x^2}{20}+\frac{y^2}{5}=1$; 2) касается его; 3) проходит вне этого эллипса.
435. Вывести условие, при котором прямая $y=k x+m$ касается эллипса $\frac{x^2}{a^2}+\frac{y^2}{b^2}=1$.
486. Составить уравнение касательной к эллипсу $\frac{x^2}{a^2}+\frac{y^2}{b^2}=1$ в его точке $M_1\left(x_1 ; y_1\right)$.
487. Доказать, что касательные к эллипсу $\frac{x^2}{a^2}+\frac{y^2}{b^2}=$ $=1$, проведенные в концах одного и того же диаметра, параллельны. (Диаметром эллипса называется его хорда, проходящая через центр.)
488. Составить уравнения касательных к эллипсу $\frac{x^2}{10}+\frac{2 y^2}{5}=1$, параллельных прямой $3 x+2 y+7=0$.
489. Составить уравнения касательных к эллипсу $x^2+4 y^2=20$, перпендикулярных к прямой $2 x-2 y-13=$ $=0$.
490. Провести касательные к эллиису $\frac{x^2}{30}+\frac{y^2}{24}=1$ параллельно прямой $4 x-2 y+23=0$ и вычислить расстояние $d$ между ними.
491. На эллипсе $\frac{x^2}{18}+\frac{y^2}{8}=1$ найти точку $M_1$, ближайшую к прямой $2 x-3 y+25=0$, и вычислить расстояние $d$ от точки $M_1$ до этой прямой.
492. Из точки $A\left(\frac{10}{3} ; \frac{5}{3}\right)$ проведены касательные к эллипсу $\frac{x^2}{20}+\frac{y^2}{5}=1$. Составить их уравнения.
493. Из точки $C(10 ;-8)$ проведены касательные к эллипсу $\frac{x^2}{25}+\frac{y^2}{16}=1$. Составить уравнение хорды, соединяющей точки касания.
494. Из точки $P(-16 ; 9)$ проведены касательные к эллипсу $\frac{x^2}{4}+\frac{y^2}{3}=1$. Вычислить расстояние $d$ от точки P до хорды эллипса, соединяющей точки касания.
495. Эллипс проходит через точку $A(4 ;-1)$ и касается прямой $x+4 y-10=0$. Составить уравнение этого эллипса при условии, что его оси совпадают с осями координат.
496. Составить уравнение эллипса, касающегося двух прямых $3 x-2 y-20=0, x+6 y-20=0$, при условии, что его оси совпадают с осями коордннат.
497. Доказать, что произведение расстояний от центра эллипса до точки пересечения любой его касательной с фокальной осью и до основания перпендикуляря, опущенного из точки касания на фокальную ось, есть величина постоянная, равная квадрату большой полуоси эллипса.
498. Доказать, что произведение расстояний от фокусов до любой касательной к эллипсу равно квадрату малой полуоси.
