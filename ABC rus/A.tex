\section{Каноническое уравнение эллипса}



\textbf{444.1)} Составить уравнение эллипса, фокусы которого лежат на оси абсцисс, симметрично относительно начала координат, зная, кроме того, что: его полуоси равны 5 и 2 ;

\textbf{444.2)} Составить уравнение эллипса, фокусы которого лежат на оси абсцисс, симметрично относительно начала координат, зная, кроме того, что: его большая ось равна 10 ,

\textbf{444.3)} Составить уравнение эллипса, фокусы которого лежат на оси абсцисс, симметрично относительно начала координат, зная, кроме того, что: его малая ось равна 24 , а расстояние между фокусами $2 c=10$;

\textbf{444.4)} Составить уравнение эллипса, фокусы которого лежат на оси абсцисс, симметрично относительно начала координат, зная, кроме того, что: расстояние между его фокусами $2 c=6$ и эксцентриситет $\varepsilon=\frac{3}{5}$;

\textbf{444.5)} Составить уравнение эллипса, фокусы которого лежат на оси абсцисс, симметрично относительно начала координат, зная, кроме того, что: его большая ось равна 20, а эксцентриситет $\varepsilon=\frac{3}{5}$;

\textbf{444.6)} Составить уравнение эллипса, фокусы которого лежат на оси абсцисс, симметрично относительно начала координат, зная, кроме того, что: его малая ось равна 10 , а эксцентриситет $\varepsilon=\frac{12}{13}$;

\textbf{444.7)} Составить уравнение эллипса, фокусы которого лежат на оси абсцисс, симметрично относительно начала координат, зная, кроме того, что: расстояние между его директрисами равно 5 и расстояние между фокусами $2 c=4$;

\textbf{444.8)} Составить уравнение эллипса, фокусы которого лежат на оси абсцисс, симметрично относительно начала координат, зная, кроме того, что: его большая ось равна 8, а расстояние между директрисами равно 16 ;

\textbf{444.9)} Составить уравнение эллипса, фокусы которого лежат на оси абсцисс, симметрично относительно начала координат, зная, кроме того, что: его малая ось равна 6, а расстояние между дирек трисами равно 13 ;

\textbf{444.10)} Составить уравнение эллипса, фокусы которого лежат на оси абсцисс, симметрично относительно начала координат, зная, кроме того, что: расстояние между его директрисами равно 32 и $\varepsilon=\frac{1}{2}$.

\textbf{445.1)} Составить уравнение эллипса, фокусы которого лежат на оси ординат, симметрично относительно начала координат, зная, кроме того, что: его полуоси равны соответственно 7 и 2 ;

\textbf{445.2)} Составить уравнение эллипса, фокусы которого лежат на оси ординат, симметрично относительно начала координат, зная, кроме того, что: его большая ось равна 10 , а расстояние между фокусами $2 c=8$;

\textbf{445.3)} Составить уравнение эллипса, фокусы которого лежат на оси ординат, симметрично относительно начала координат, зная, кроме того, что: расстояние между его фокусами $2 c=24$ и эксцентриситет $\varepsilon=\frac{12}{13}$;

\textbf{445.4)} Составить уравнение эллипса, фокусы которого лежат на оси ординат, симметрично относительно начала координат, зная, кроме того, что: его малая ось равна 16 , а эксцентриситет $\varepsilon=\frac{3}{5}$;

\textbf{445.5)} Составить уравнение эллипса, фокусы которого лежат на оси ординат, симметрично относительно начала координат, зная, кроме того, что: расстояние между его фокусами $2 c=6$ и расстояние между директрисами равно $16 \frac{2}{3}$;

\textbf{445.6)} Составить уравнение эллипса, фокусы которого лежат на оси ординат, симметрично относительно начала координат, зная, кроме того, что: расстояние между его директрисами равно $10 \frac{2}{3}$ и эксцентриситет $\varepsilon=\frac{3}{4}$.

\textbf{450.} Вычислить площадь четырехугольника, две вершины которого лежат в фокусах эллипса $9 x^2+5 y^2=1$, две другие совпадают с концами его малой оси.

\textbf{462.} Определить точки эллипса $\frac{x^2}{100}+\frac{y^2}{36}=1$, pacстояние которых до правого фокуса равно 14.

\textbf{463.} Определить точки эллипса $\frac{x^2}{16}+\frac{y^2}{7}=1$, pacстояние которых до левого фокуса равно $2,5$.

\textbf{465.1)} Составить уравнение эллипса, фокусы которого расположены на оси абсцисс, симметрично относительно начала координат, если даны: точка $M_1(-2 \sqrt{5} ; 2)$ эллипса и его малая полуось $b=3$;

\textbf{465.2)} Составить уравнение эллипса, фокусы которого расположены на оси абсцисс, симметрично относительно начала координат, если даны: точка $M_1(2 ;-2)$ эллипса и его большая полуось $a=4$;

\textbf{465.3)} Составить уравнение эллипса, фокусы которого расположены на оси абсцисс, симметрично относительно начала координат, если даны: точки $M_1(4 ;-\sqrt{3})$ и $M_2(2 \sqrt{2} ; 3)$ эллипса;

\textbf{465.4)} Составить уравнение эллипса, фокусы которого расположены на оси абсцисс, симметрично относительно начала координат, если даны: точка $M_1(\sqrt{15} ;-1)$ эллипса и расстояние между его фокусами $2 c=8$;

\textbf{465.5)} Составить уравнение эллипса, фокусы которого расположены на оси абсцисс, симметрично относительно начала координат, если даны: точка $M_1\left(2 ;-\frac{5}{3}\right)$ эллипса и его эксцентриситет $\varepsilon=\frac{2}{3}$;

\textbf{465.6)} Составить уравнение эллипса, фокусы которого расположены на оси абсцисс, симметрично относительно начала координат, если даны: точка $M_1(8 ; 12)$ эллипса и расстояние $r_1=20$ от нее до левого фокуса;

\textbf{465.7)} Составить уравнение эллипса, фокусы которого расположены на оси абсцисс, симметрично относительно начала координат, если даны: точка $M_1(-\sqrt{5} ; 2)$ эллипса и расстояние между его директрисами равно 10.

\textbf{470.} Точка $C(-3 ; 2)$ является центром эллипса, касающегося обеих координатных осей. Составить уравнение этого эллипса, зная, что его оси симметрии параллельны координатным осям.

\textbf{474.} Составить уравнение эллипса, єсли известны его эксцентриситет $\varepsilon=\frac{2}{3}$, фокус $F(2 ; 1)$ и уравнение соответствующей директрисы $x-5=0$.

\textbf{475.} Составить уравнение эллипса, если известны его өксцентриситет $\varepsilon=\frac{1}{2}$, фокус $F(-4 ; 1)$ и уравнение соответствующей директрисы $y+3=0$.

\textbf{476.} Точка $A(-3 ;-5)$ лежит на эллипсе, фокус которого $F(-1 ;-4)$, а соответствующая директриса дана уравнением $x-2=0$. Составить уравнение этого эллипса.



\section{Каноническое уравнение гиперболы}



\textbf{515.1)}Составить уравнение гиперболы, фокусы когорой расположены на оси абсцисс симметрично относительно начала координат, зная, кроме того, что: ее оси $2 a=10$ и $2 b=8$;

\textbf{515.2)}Составить уравнение гиперболы, фокусы когорой расположены на оси абсцисс симметрично относительно начала координат, зная, кроме того, что: расстсяние между фокусами $2 c=10$ и ось $2 b=8$;

\textbf{515.3)}Составить уравнение гиперболы, фокусы когорой расположены на оси абсцисс симметрично относительно начала координат, зная, кроме того, что: расстояние между фокусами $2 c=6$ и эксцентриситет $\varepsilon=\frac{3}{2}$;

\textbf{515.4)}Составить уравнение гиперболы, фокусы когорой расположены на оси абсцисс симметрично относительно начала координат, зная, кроме того, что: ось $2 a=16$ и эксцентриситет $\varepsilon=\frac{5}{4}$;

\textbf{515.5)}Составить уравнение гиперболы, фокусы когорой расположены на оси абсцисс симметрично относительно начала координат, зная, кроме того, что: уравнения асимптот $y= \pm \frac{4}{3} x$ и расстояние между фокусами $2 c=20$;

\textbf{515.6)}Составить уравнение гиперболы, фокусы когорой расположены на оси абсцисс симметрично относительно начала координат, зная, кроме того, что: расстояние между директрисами равно $22 \frac{2}{13}$ и расстояние между фокусами $2 c=26$;

\textbf{515.7)}Составить уравнение гиперболы, фокусы когорой расположены на оси абсцисс симметрично относительно начала координат, зная, кроме того, что: расстояние между директрисами равно $\frac{32}{5}$ и ось $2 b=6$;

\textbf{515.8)}Составить уравнение гиперболы, фокусы когорой расположены на оси абсцисс симметрично относительно начала координат, зная, кроме того, что: расстояние между директрисами равно $\frac{8}{3}$ и эксцентриситет $\varepsilon=\frac{3}{2}$;

\textbf{515.9)}Составить уравнение гиперболы, фокусы когорой расположены на оси абсцисс симметрично относительно начала координат, зная, кроме того, что: уравнения асимптот $y= \pm \frac{3}{4} x$ и расстояние между директрисами равно $12 \frac{4}{5}$.

\textbf{516.1)}Составить уравнение гиперболы, фокусы которой расположены на оси ординат симметрично относительно начала координат, зная, кроме того, что: ее полуоси $a=6, b=18$ (буквой $a$ мы обозначаем полуось гинерболы, расположенную на оси абсцисс) ;

\textbf{516.2)}Составить уравнение гиперболы, фокусы которой расположены на оси ординат симметрично относительно начала координат, зная, кроме того, что: расстояние между фокусами $2 c=10$ и эксцентриситет $\varepsilon=\frac{5}{3}$;

\textbf{516.3)}Составить уравнение гиперболы, фокусы которой расположены на оси ординат симметрично относительно начала координат, зная, кроме того, что: уравнения асимптот $y= \pm \frac{12}{5} x$ и расстояние между вершинами равно 48;

\textbf{516.4)}Составить уравнение гиперболы, фокусы которой расположены на оси ординат симметрично относительно начала координат, зная, кроме того, что: расстояние между директрисами равно $7 \frac{1}{7}$ и эксцентриситет $\varepsilon=\frac{7}{5}$;

\textbf{516.5)}Составить уравнение гиперболы, фокусы которой расположены на оси ординат симметрично относительно начала координат, зная, кроме того, что: уравнения асимптот $y= \pm \frac{4}{3} x$ и расстояние между директрисами равно $6 \frac{2}{5}$.

\textbf{518.} Дана гипербола $16 x^2-9 y^2=144$. Найти: 1) полуоси $a$ и $b ; 2$ ) фокусы; 3) эксцентриситет; 4) уравнения асимптот; 5) уравнения директрис.

\textbf{520.} Вычислить площадь треугольника, образованного асимптотами гиперболы $\frac{x^2}{4}-\frac{y^2}{9}=1$ и прямой $9 x+$ $+2 y-24=0$

\textbf{521.1)} Установить, какие линии определяются следующими уравнениями: $y=+\frac{2}{3} \sqrt{x^2-9}$

\textbf{521.2)} Установить, какие линии определяются следующими уравнениями: $y=-3 \sqrt{x^2+1}$;

\textbf{521.3)} Установить, какие линии определяются следующими уравнениями: $x=-\frac{4}{3} \sqrt{y^2+9} ;$

\textbf{521.4)} Установить, какие линии определяются следующими уравнениями: $y=+\frac{2}{5} \sqrt{x^2+25}$

\textbf{524.} Эксцентриситет гиперболы $\varepsilon=2$, фокальный радиус ее точки $M$, проведенный из некоторого фокуса, равен 16. Вычислить расстояние от точки $M$ до односто* ронней с этим фокусом директрисы.

\textbf{525.} Эксцентриситет гиперболы $\varepsilon=3$, расстояние от точки. $M$ гиперболы до директрисы равно 4 . Вычислить расстояние от точки $M$ до фокуса, одностороннего с этой директрисой.

\textbf{532.1)} Составить уравнение гиперболы, фокусы которой лежат на оси абсцисс симметрично относительно начала координат, если даны: точки $M_1(6 ;-1)$ и $M_2(-8 ; 2 \sqrt{2})$ гиперболы;

\textbf{532.2)} Составить уравнение гиперболы, фокусы которой лежат на оси абсцисс симметрично относительно начала координат, если даны: точка $M_1(-5 ; 3)$ гиперболы и эксцентриситет $\varepsilon=\sqrt{2}$;

\textbf{532.3)} Составить уравнение гиперболы, фокусы которой лежат на оси абсцисс симметрично относительно начала координат, если даны: точка $M_1\left(\frac{9}{2} ;-1\right)$ гиперболы и уравнения асимптот $y= \pm \frac{2}{3} x$;

\textbf{532.4)} Составить уравнение гиперболы, фокусы которой лежат на оси абсцисс симметрично относительно начала координат, если даны: точка $M_1\left(-3 ; \frac{5}{2}\right)$ гиперболы и уравнения директрис $x= \pm \frac{4}{3}$;

\textbf{532.5)} Составить уравнение гиперболы, фокусы которой лежат на оси абсцисс симметрично относительно начала координат, если даны: уравнения асимптот $y= \pm \frac{3}{4} x$ и уравнения директрис $x= \pm \frac{16}{5}$.

\textbf{544.} Составить уравнение гиперболы, если известны ее эксцентриситет $\varepsilon=\frac{5}{4}$, фокус $F(5 ; 0)$ и уравнение состветствующей директрисы $5 x-16=0$.

\textbf{545.} Составить уравнение гиперболы, если известны ее эксцентриситет $\varepsilon=\frac{13}{12}$, фокус $F(0 ; 13)$ и уравнение соответствующей директрисы $13 y-144=0$.



\section{Каноническое уравнение параболы}



\textbf{583.1)} Составить уравнение параболы, вершина которой находится в начале координат, зная, что: парабола расположена в правой полуплоскости симметрично относительно оси $O x$, и ее параметр $p=3$;

\textbf{583.2)} Составить уравнение параболы, вершина которой находится в начале координат, зная, что: парабола расположена в левой полуплоскости симметрично относительно оси $O x$, и ее параметр $p=0,5$;

\textbf{583.3)} Составить уравнение параболы, вершина которой находится в начале координат, зная, что: парабола расположена в верхней полуплоскости симметррично относительно оси $O y$, и ее параметр $p=\frac{1}{4}$;

\textbf{583.4)} Составить уравнение параболы, вершина которой находится в начале координат, зная, что: парабола расположена в нижней полуплоскости симметрично относительно оси $O y$, и ее параметр $p=3$.

\textbf{585.1)} Составить уравнение параболы, вершина которой находится в начале координат, зная, что: парабола расположена симметрично относительно оси $O x$ и проходит через точку $A(9 ; 6)$;

\textbf{585.2)} Составить уравнение параболы, вершина которой находится в начале координат, зная, что: парабола расположена симметрично относительно оси $O x$ и проходит через точку $B(-1 ; 3)$;

\textbf{585.3)} Составить уравнение параболы, вершина которой находится в начале координат, зная, что: парабола расположена симметрично относительно оси $O y$ и проходит через точку $C(1 ; 1)$.

\textbf{585.4)} Составить уравнение параболы, вершина которой находится в начале координат, зная, что: парабола расположена симметрично отнлсительно оси $O y$ и проходит через точку $D(4 ;-8)$.

\textbf{586.} Стальной трос подвешен за два конца; точки креп.тения расположены на одинаковой высоте; расстояние между ними равно 20 м. Величина его прогиба на расстолиии 2 m от точки крепления, считая по горизонтали, равна 14,4 см. Определить величину прогиба этого троса в ссредине между точками крепления, приближенно считая, что трос имеет форму дуги параболы.

\textbf{589.} Найти фокус $F$ и уравнение директрисы параболы $y^2=24 x$.

\textbf{590.} Вычислить фокальный радиус точки $M$ параболы $y^2=20 x$, если абсцисса точки $M$ равна 7 .

\textbf{605.} Определить точки пересечения прямой $x+y$ -$-3=0$ и параболы $x^2=4 y$.

\textbf{617.} Составить уравнения касательных к параболе $y^2=36 x$, проведенных из точки $A(2 ; 9)$.

\textbf{621.} Определить точки пересечения эллипса $\frac{x^2}{100}+\frac{y^2}{225}=1$ и параболы $y^2=24 x$

\textbf{622.} Определить точки пересечения гиперболы $\frac{x^2}{20}$ -$-\frac{y^2}{5}=-1$ и параболы $y^2=3 x$



\section{ Центр линии второго порядка }



\textbf{665.1)} Установить, какие из следующих линий являются центральными (т.е. имеют единственный центр), какие имеют центра, какие имеют бесконечно много центров: $3 x^2-4 x y-2 y^2+3 x-12 y-7=0$;

\textbf{665.2)} Установить, какие из следующих линий являются центральными (т.е. имеют единственный центр), какие имеют центра, какие имеют бесконечно много центров: $4 x^2+5 x y+3 y^2-x+9 y-12=0$;

\textbf{665.3)} Установить, какие из следующих линий являются центральными (т.е. имеют единственный центр), какие имеют центра, какие имеют бесконечно много центров: $4 x^2-4 x y+y^2-6 x+8 y+13=0$;

\textbf{665.4)} Установить, какие из следующих линий являются центральными (т.е. имеют единственный центр), какие имеют центра, какие имеют бесконечно много центров: $4 x^2-4 x y+y^2-12 x+6 y-11=0$;

\textbf{665.5)} Установить, какие из следующих линий являются центральными (т.е. имеют единственный центр), какие имеют центра, какие имеют бесконечно много центров: $x^2-2 x y+4 y^2+5 x-7 y+12=0$;

\textbf{665.6)} Установить, какие из следующих линий являются центральными (т.е. имеют единственный центр), какие имеют центра, какие имеют бесконечно много центров: $x^2-2 x y+y^2-6 x+6 y-3=0$;

\textbf{665.7)} Установить, какие из следующих линий являются центральными (т.е. имеют единственный центр), какие имеют центра, какие имеют бесконечно много центров: $4 x^2-20 x y+25 y^2-14 x+2 y-15=0$;

\textbf{665.8)} Установить, какие из следующих линий являются центральными (т.е. имеют единственный центр), какие имеют центра, какие имеют бесконечно много центров: $4 x^2-6 x y-9 y^2+3 x-7 y+12=0$.

\textbf{667.1)} Установить, что каждая из следующих линий имєет бесконечно много цєнтров; для каждой их них составить уравнение геометрического места центров: $x^2-6 x y+9 y^2-12 x+36 y+20=0$;

\textbf{667.2)} Установить, что каждая из следующих линий имєет бесконечно много цєнтров; для каждой их них составить уравнение геометрического места центров: $4 x^2+4 x y+y^2-8 x-4 y-21=0$;

\textbf{667.3)} Установить, что каждая из следующих линий имєет бесконечно много цєнтров; для каждой их них составить уравнение геометрического места центров: $25 x^2-10 x y+y^2+40 x-8 y+7=0$.



\section{Приведение к простейшему виду уравнекия центральной линии второго порядка}



\textbf{675.1)} Определить тип каждого из следующих уравнений при помощи вычисления дискриминанта старших членов: $2 x^2+10 x y+12 y^2-7 x+18 y-15=0$;

\textbf{675.2)} Определить тип каждого из следующих уравнений при помощи вычисления дискриминанта старших членов: $3 x^2-8 x y+7 y^2+8 x-15 y+20=0$;

\textbf{675.3)} Определить тип каждого из следующих уравнений при помощи вычисления дискриминанта старших членов: $25 x^2-20 x y+4 y^2-12 x+20 y-17=0$;

\textbf{675.4)} Определить тип каждого из следующих уравнений при помощи вычисления дискриминанта старших членов: $5 x^2+14 x y+11 y^2+12 x-7 y+19=0$;

\textbf{675.5)} Определить тип каждого из следующих уравнений при помощи вычисления дискриминанта старших членов: $x^2-4 x y+4 y^2+7 x-12=0$;

\textbf{675.6)} Определить тип каждого из следующих уравнений при помощи вычисления дискриминанта старших членов: $3 x^2-2 x y-3 y^2+12 y-15=0$.

\textbf{678.1)} He проводя преобразования координат, установить, что каждое из следующих уравнений определяет эллипс, и найти величины его полуосей: $41 x^2+24 x y+9 y^2+24 x+18 y-36=0$;

\textbf{678.2)} He проводя преобразования координат, установить, что каждое из следующих уравнений определяет эллипс, и найти величины его полуосей: $8 x^2+4 x y+5 y^2+16 x+4 y-28=0$;

\textbf{678.3)} He проводя преобразования координат, установить, что каждое из следующих уравнений определяет эллипс, и найти величины его полуосей: $13 x^2+18 x y+37 y^2-26 x-18 y+3=0$;

\textbf{678.4)} He проводя преобразования координат, установить, что каждое из следующих уравнений определяет эллипс, и найти величины его полуосей: $13 x^2+10 x y+13 y^2+46 x+62 y+13=0$.

\textbf{679.1)} Не проводя преобразования координат, установить, что каждое из следующих уравнений определяет единственную точку (вырожденный эллипс), и найти ее координаты: $5 x^2-6 x y+2 y^2-2 x+2=0$;

\textbf{679.2)} Не проводя преобразования координат, установить, что каждое из следующих уравнений определяет единственную точку (вырожденный эллипс), и найти ее координаты: $x^2+2 x y+2 y^2+6 y+9=0$;

\textbf{679.3)} Не проводя преобразования координат, установить, что каждое из следующих уравнений определяет единственную точку (вырожденный эллипс), и найти ее координаты: $5 x^2+4 x y+y^2-6 x-2 y+2=0$;

\textbf{679.4)} Не проводя преобразования координат, установить, что каждое из следующих уравнений определяет единственную точку (вырожденный эллипс), и найти ее координаты: $x^2-6 x y+10 y^2+10 x-32 y+26=0$.

\textbf{680.1)} Не проводя преобразования координат, установить, что каждое из следующих уравнений определяет гиперболу, и найти величины ее полуосей: $4 x^2+24 x y+11 y^2+64 x+42 y+51=0$;

\textbf{680.2)} Не проводя преобразования координат, установить, что каждое из следующих уравнений определяет гиперболу, и найти величины ее полуосей: $12 x^2+26 x y+12 y^2-52 x-48 y+73=0$

\textbf{680.3)} Не проводя преобразования координат, установить, что каждое из следующих уравнений определяет гиперболу, и найти величины ее полуосей: $3 x^2+4 x y-12 x+16=0$;

\textbf{680.4)} Не проводя преобразования координат, установить, что каждое из следующих уравнений определяет гиперболу, и найти величины ее полуосей: $x^2-6 x y-7 y^2+10 x-30 y+23=0$.

\textbf{681.1)} Не проводя преобразования координат, установить, что каждое из следующих уравнений определяет пару пересекающихся прямых (вырожденную гиперболу), и найти их уравнения: $3 x^2+4 x y+y^2-2 x-1=0$;

\textbf{681.2)} Не проводя преобразования координат, установить, что каждое из следующих уравнений определяет пару пересекающихся прямых (вырожденную гиперболу), и найти их уравнения: $x^2-6 x y+8 y^2-4 y-4=0$;

\textbf{681.3)} Не проводя преобразования координат, установить, что каждое из следующих уравнений определяет пару пересекающихся прямых (вырожденную гиперболу), и найти их уравнения: $x^2-4 x y+3 y^2=0$;

\textbf{681.4)} Не проводя преобразования координат, установить, что каждое из следующих уравнений определяет пару пересекающихся прямых (вырожденную гиперболу), и найти их уравнения: $x^2+4 x y+3 y^2-6 x-12 y+9=0$.

\textbf{682.1)} Не проводя преобразования координат, установить, какие геометрические образы определяются следующими уравнениями: $8 x^2-12 x y+17 y^2+16 x-12 y+3=0$;

\textbf{682.2)} Не проводя преобразования координат, установить, какие геометрические образы определяются следующими уравнениями: $17 x^2-18 x y-7 y^2+34 x-18 y+7=0$;

\textbf{682.3)} Не проводя преобразования координат, установить, какие геометрические образы определяются следующими уравнениями: $2 x^2+3 x y-2 y^2+5 x+10 y=0$;

\textbf{682.4)} Не проводя преобразования координат, установить, какие геометрические образы определяются следующими уравнениями: $6 x^2-6 x y+9 y^2-4 x+18 y+14=0$;



\section{Приведение к простейшему виду параболического уравнения}



\textbf{697.1)} Не проводя преобразования координат, установить, что каждое из следующих уравнений определяет параболу, и найти параметр этой параболы: $9 x^2+24 x y+16 y^2-120 x+90 y=0$;

\textbf{697.2)} Не проводя преобразования координат, установить, что каждое из следующих уравнений определяет параболу, и найти параметр этой параболы: $9 x^2-24 x y+16 y^2-54 x-178 y+181=0$;

\textbf{697.3)} Не проводя преобразования координат, установить, что каждое из следующих уравнений определяет параболу, и найти параметр этой параболы: $x^2-2 x y+y^2+6 x-14 y+29=0$;

\textbf{697.4)} Не проводя преобразования координат, установить, что каждое из следующих уравнений определяет параболу, и найти параметр этой параболы: $9 x^2-6 x y+y^2-50 x+50 y-275=0$.



\section{Поверхности второго порядка}



\textbf{1153.} Установить, что плоскость $x-2=0$ пересекает эллипсоид $\frac{x^2}{16}+\frac{y^2}{12}+\frac{z^2}{4}=1$ по эллипсу; найти его полуоси и вершины.

\textbf{1154.} Установить, что плоскость $z+1=0$ пересекает однополостный гиперболоид $\frac{x^2}{32}-\frac{y^2}{18}+\frac{z^2}{2}=1$ по гиперболе; найти ее полуоси и вершины.

\textbf{1155.} Установить, что плоскость $y+6=0$ пересекает гиперболический параболоид $\frac{x^2}{5}-\frac{y^2}{4}=6 z$ по параболе; найти ее параметр и вершину.

\textbf{1156.} Найти уравнения проекций на координатные плоскости сечения эллиптического параболоида $y^2+z^2=x$ плоскостью $x+2 y-z=0$

\textbf{1159.1)} Установить, какие линии определяются следующими уравнениями: $\left\{\begin{array}{l}\frac{x^2}{3}+\frac{y^2}{6}=2 z, \\ 3 x-y+6 z-14=0\end{array}\right.$

\textbf{1159.2)} Установить, какие линии определяются следующими уравнениями: $\left\{\begin{array}{l}\frac{x^2}{4}-\frac{y^2}{3}=2 z \\ x-2 y+2=0 ;\end{array}\right.$

\textbf{1159.3)} Установить, какие линии определяются следующими уравнениями: $\left\{\begin{array}{l}\frac{x^2}{.4}+\frac{y^2}{9}-\frac{z^2}{36}=1, \\ 9 x-6 y+2 z-28=0,\end{array}\right.$