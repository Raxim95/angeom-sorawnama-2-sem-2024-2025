\section{Каноническое уравнение эллипса}



\textbf{448.} Вычислить площадь четырехугольника, две вершины которого лежат в фокусах эллипса $x^2+5 y^2=20$, а две другие совпадают с концами его малой оси.

\textbf{456.} Эксцентриситет эллипса $\varepsilon=\frac{2}{3}$, фокальный радиус точки $M$ эллипса равен 10 . Вычислить расстояние от точки $M$ до односторонней с этим фокусом директрисы.

\textbf{457.} Эксцентриситет эллипса $\varepsilon=\frac{2}{5}$, расстояние от точки $M$ эллипса до директрисы равно 20. Вычислить расстояние от точки $M$ до фокуса, одностороннего с этой директрисой.

\textbf{460.} Эксцентриситет эллипса $\varepsilon=\frac{1}{3}$, центр его совпадает с началом координат, один из фокусов $(-2 ; 0)$. Вычиелить расстояние от точки $M_1$ эллипса с абсциссой, равной 2, до директрисы, односторонней с данным фокусом.

\textbf{461.} Эксцентриситет эллипса $\varepsilon=\frac{1}{2}$, центр его совпадает с началом координат, одна из директрис дана уравнением $x=16$. Вычислить расстояние от точки $M_1$ эллипса с абсциссой, равной -4, до фокуса, одностороннего с данной директрисой.

\textbf{464.} Через фокус эллипса $\frac{x^2}{25}+\frac{y^2}{15}=1$ проведен перпендикуляр к его большой оси. Определить расстояния от точек пересечения этого перпендикуляра с эллипсом до фокусов.

\textbf{466.1)} Определить эксцентриситет в эллипса, если: его малая ось видна из фокусов под углом в $60^{\circ}$;

\textbf{466.2)} Определить эксцентриситет в эллипса, если: отрезок между фоку сами виден из вєршин малой оси под прямым углом;

\textbf{466.3)} Определить эксцентриситет в эллипса, если: расстояние между директрисами в три раза больше расстояния между фокусами;

\textbf{466.4)} Определить эксцентриситет в эллипса, если: отрезок перпендикуляра, опущенного из центра эллипса на его директрису, делится вершиной эллипса пополам.

\textbf{473.1)} Составить уравнение эллипса, зная, что: его большая ось равна 26 и фокусы суть $F_1(-10 ; 0), F_2(14 ; 0)$;

\textbf{473.2)} Составить уравнение эллипса, зная, что: его малая ось равна 2 и фокусы суть $F_1(-1 ;-1)$, $F_2(1 ; 1)$;

\textbf{473.3)} Составить уравнение эллипса, зная, что: его фокусы суть $F_1\left(-2 ; \frac{3}{2}\right), F_2\left(2 ;-\frac{3}{2}\right)$ и эксцентриситет $\varepsilon=\frac{\sqrt{2}}{2}$;

\textbf{473.4)} Составить уравнение эллипса, зная, что: его фокусы суть $F_1(1 ; 3), F_2(3 ; 1)$ и расстояние между директрисами равно $12 \sqrt{2}$.

\textbf{477.} Составить уравнение эллипса, если известны его эксцентриситет $\varepsilon=\frac{1}{2}$, фокус $F(3 ; 0)$ и уравнение соответствующей директрисы $x+y-1=0$.

\textbf{478.} Точка $M_1(2 ;-1)$ лежит на эллипсе, фокус которого $F(1 ; 0)$, а соответствующая директриса дана уравнением $2 x-y-10=0$. Составить уравнение этого эллипса.



\section{Каноническое уравнение гиперболы}



\textbf{522.} Дана точка $M_1(10 ;-\sqrt{5})$ на гиперболе $\frac{x^2}{80}-\frac{y^2}{20}=1$. Составить уравнения прямых, на которых лежат фокальные радиусы точки $M_1$.

\textbf{528.} Определить точки гиперболы $\frac{x^2}{64}-\frac{y^2}{36}=1$, расстояние которых до правого фокуса равно 4,5 .

\textbf{533.} Определить эксцентриситет равносторонней гиперболы.

\textbf{536.} Составить уравнение гиперболы, фокусы которой лежат в вершинах эллинса $\frac{x^2}{100}+\frac{y^2}{64}=1$, а директрисы проходят через фокусы этого эллипса.

\textbf{543.1)} Составить уравнение гиперболы, зная, что: расстояние между ее вершинами равно 24 и фокусы суть $F_1(-10 ; 2), F_2(16 ; 2)$;

\textbf{543.2)} Составить уравнение гиперболы, зная, что: фокусы суть $F_1(3 ; 4), F_2(-3 ;-4)$ и расстояние между директрисами равно 3,6 ;

\textbf{543.3)} Составить уравнение гиперболы, зная, что: угол между асимптотами равен $90^{\circ}$ и фокусы суть $F_1(4 ;-4), F_2(-2 ; 2)$.

\textbf{547.} Составить уравнение гиперболы, если известны ее эксцентриситет $\varepsilon=\sqrt{5}$, фокус $F(2 ;-3)$ и уравнение соответствующей директрисы $3 x-y+3=0$.

\textbf{548.} Точка $M_1(1 ;-2)$ лежит на гиперболе, фокус которой $F(-2 ; 2)$, а соответствующая директриса дана уравнением $2 x-y-1=0$. Составить уравнение этой гиперболы.

\textbf{560.} Составить уравнения касательных к гиперболе $\frac{x^2}{16}-\frac{y^2}{64}=1$, параллельных прямой $10 x-3 y+9=0$.

\textbf{561.} Провести касательные к гиперболе $\frac{x^2}{16}-\frac{y^2}{8}=-1$ параллельно прямой $2 x+4 y-5=0$ и вычис лить расстояние $d$ между ними.

\textbf{563.} Составить уравнение касательных к гиперболе $x^2-y^2=16$, проведенных из точки $A(-1 ;-7)$.



\section{Каноническое уравнение параболы}



\textbf{600.} Составить уравнение параболы, если даны ее фокус $F(7 ; 2)$ и директриса $x-5=0$

\textbf{601.} Составить уравнение параболы, если даны ее фокус $F(4 ; 3)$ и директриса $y+1=0$.

\textbf{602.} Составить уравнение параболы, если даны ее фокус $F(2 ;-1)$ и директриса $x-y-1=0$.

\textbf{603.} Даны вершина параболы $A(6 ;-3)$ и уравнение ее директрисы $3 x-5 y+1=0$. Найти фокус $F$ этой параболы.

\textbf{604.} Даны вершина параболы $A(-2 ;-1)$ и урав нение ее директрисы $x+2 y-1=0$. Составить уравнение этой параболы.

\textbf{613.} Составить уравнение прямой, которая касается параболы $y^2=8 x$ и параллельна прямой $2 x+2 y-3=0$.

\textbf{614.} Составить уравнение прямой, которая касается параболы $x^2=16 y$ и перпендикулярна к прямой $2 x+4 y+7=0$.

\textbf{619.} Из точки $A(5 ; 9)$ проведены касательные к параболе $y^2=5 x$. Составить уравнение хорды, соединяющей точки касания.



\section{ Центр линии второго порядка }



\textbf{666.1)} Установить, что следующие линии являются центральными, и для каждой из них найти координаты центра: $3 x^2+5 x y+y^2-8 x-11 y-7=0$;

\textbf{666.2)} Установить, что следующие линии являются центральными, и для каждой из них найти координаты центра: $5 x^2+4 x y+2 y^2+20 x+20 y-18=0$;

\textbf{666.3)} Установить, что следующие линии являются центральными, и для каждой из них найти координаты центра: $9 x^2-4 x y-7 y^2-12=0$;

\textbf{666.4)} Установить, что следующие линии являются центральными, и для каждой из них найти координаты центра: $2 x^2-6 x y+5 y^2+22 x-36 y+11=0$.

\textbf{668.1)} Установить, что следующие уравнения определяют центральные линии; преобразовать каждое из них путем переноса начала координат в центр: $3 x^2-6 x y+2 y^2-4 x+2 y+1=0$;

\textbf{668.2)} Установить, что следующие уравнения определяют центральные линии; преобразовать каждое из них путем переноса начала координат в центр: $6 x^2+4 x y+y^2+4 x-2 y+2=0$;

\textbf{668.3)} Установить, что следующие уравнения определяют центральные линии; преобразовать каждое из них путем переноса начала координат в центр: $4 x^2+6 x y+y^2-10 x-10=0$;

\textbf{668.4)} Установить, что следующие уравнения определяют центральные линии; преобразовать каждое из них путем переноса начала координат в центр: $4 x^2+2 x y+6 y^2+6 x-10 y+9=0$.



\section{Приведение к простейшему виду уравнекия центральной линии второго порядка}



\textbf{673.1)} Определить тип каждого из следующих уравнений каждое из них путем параллельного переноса осей координат привести к простейшему виду; установить, какие геометрические образы они определяют, и изобразить на чертеже расположение этих образов относительно старых и новых осей координат: $4 x^2+9 y^2-40 x+36 y+100=0$;

\textbf{673.2)} Определить тип каждого из следующих уравнений каждое из них путем параллельного переноса осей координат привести к простейшему виду; установить, какие геометрические образы они определяют, и изобразить на чертеже расположение этих образов относительно старых и новых осей координат: $9 x^2-16 y^2-54 x-64 y-127=0$;

\textbf{673.3)} Определить тип каждого из следующих уравнений каждое из них путем параллельного переноса осей координат привести к простейшему виду; установить, какие геометрические образы они определяют, и изобразить на чертеже расположение этих образов относительно старых и новых осей координат: $9 x^2+4 y^2+18 x-8 y+49=0$;

\textbf{673.4)} Определить тип каждого из следующих уравнений каждое из них путем параллельного переноса осей координат привести к простейшему виду; установить, какие геометрические образы они определяют, и изобразить на чертеже расположение этих образов относительно старых и новых осей координат: $4 x^2-y^2+8 x-2 y+3=0$;

\textbf{673.5)} Определить тип каждого из следующих уравнений каждое из них путем параллельного переноса осей координат привести к простейшему виду; установить, какие геометрические образы они определяют, и изобразить на чертеже расположение этих образов относительно старых и новых осей координат: $2 x^2+3 y^2+8 x-6 y+11=0$.

\textbf{674.1)} Каждое из следующих уравнений привести к простейшему виду; определить тип каждого из них; установить, какие геометрические образы они определяют, и изобразить на чертеже расположение этих образов относительно старых и новых осей координат: $32 x^2+52 x y-7 y^2+180=0$;

\textbf{674.2)} Каждое из следующих уравнений привести к простейшему виду; определить тип каждого из них; установить, какие геометрические образы они определяют, и изобразить на чертеже расположение этих образов относительно старых и новых осей координат: $5 x^2-6 x y+5 y^2-32=0$;

\textbf{674.3)} Каждое из следующих уравнений привести к простейшему виду; определить тип каждого из них; установить, какие геометрические образы они определяют, и изобразить на чертеже расположение этих образов относительно старых и новых осей координат: $17 x^2-12 x y+8 y^2=0$;

\textbf{674.4)} Каждое из следующих уравнений привести к простейшему виду; определить тип каждого из них; установить, какие геометрические образы они определяют, и изобразить на чертеже расположение этих образов относительно старых и новых осей координат: $5 x^2+24 x y-5 y^2=0$;

\textbf{674.5)} Каждое из следующих уравнений привести к простейшему виду; определить тип каждого из них; установить, какие геометрические образы они определяют, и изобразить на чертеже расположение этих образов относительно старых и новых осей координат: $5 x^2-6 x y+5 y^2+8=0$.



\section{Приведение к простейшему виду параболического уравнения}



\textbf{689.1)} Установить, что каждое из следующих уравнений является параболическим; каждое из них привести к простейшему виду; установить, какие геометрические образы они определяют; для каждого случая изобразить на чертеже оси первоначальной координатной системы, оси других координатных систем, которые вводятся по ходу решения, и геометрический образ, определяемый данным уравнением: $9 x^2-24 x y+16 y^2-20 x+110 y-50=0$;

\textbf{689.2)} Установить, что каждое из следующих уравнений является параболическим; каждое из них привести к простейшему виду; установить, какие геометрические образы они определяют; для каждого случая изобразить на чертеже оси первоначальной координатной системы, оси других координатных систем, которые вводятся по ходу решения, и геометрический образ, определяемый данным уравнением: $9 x^2+12 x y+4 y^2-24 x-16 y+3=0$;

\textbf{689.3)} Установить, что каждое из следующих уравнений является параболическим; каждое из них привести к простейшему виду; установить, какие геометрические образы они определяют; для каждого случая изобразить на чертеже оси первоначальной координатной системы, оси других координатных систем, которые вводятся по ходу решения, и геометрический образ, определяемый данным уравнением: $16 x^2-24 x y+9 y^2-160 x+120 y+425=0$.

\textbf{690.1)} То же задание, что и в предыдушей задаче, выполнить для уравнений: $9 x^2+24 x y+16 y^2-18 x+226 y+209=0$;

\textbf{690.2)} То же задание, что и в предыдушей задаче, выполнить для уравнений: $x^2-2 x y+y^2-12 x+12 y-14=0$

\textbf{690.3)} То же задание, что и в предыдушей задаче, выполнить для уравнений: $4 x^2+12 x y+9 y^2-4 x-6 y+1=0$.

\textbf{693.1)} Установить, что следующие уравнения являются параболическими, и записать каждое из них в виде $(\alpha x+\beta y)^2+2 a_{13} x+2 a_{23} y+a_{33}=0$: $x^2+4 x y+4 y^2+4 x+y-15=0 ;$

\textbf{693.2)} Установить, что следующие уравнения являются параболическими, и записать каждое из них в виде $(\alpha x+\beta y)^2+2 a_{13} x+2 a_{23} y+a_{33}=0$: $9 x^2-6 x y+y^2-x+2 y-14=0$;

\textbf{693.3)} Установить, что следующие уравнения являются параболическими, и записать каждое из них в виде $(\alpha x+\beta y)^2+2 a_{13} x+2 a_{23} y+a_{33}=0$: $25 x^2-20 x y+4 y^2+3 x-y+11=0$;

\textbf{693.4)} Установить, что следующие уравнения являются параболическими, и записать каждое из них в виде $(\alpha x+\beta y)^2+2 a_{13} x+2 a_{23} y+a_{33}=0$: $16 x^2+16 x y+4 y^2-5 x+7 y=0$;

\textbf{693.5)} Установить, что следующие уравнения являются параболическими, и записать каждое из них в виде $(\alpha x+\beta y)^2+2 a_{13} x+2 a_{23} y+a_{33}=0$: $9 x^2-42 x y+49 y^2+3 x-2 y-24=0$.



\section{Поверхности второго порядка}



\textbf{1157.} Установить, какая линия является сечением эллипсоида $\frac{x^2}{12}+\frac{y^2}{4}+\frac{z^2}{3}=1$ плоскостью $2 x-3 y+4 z-11=0$, и найти ее центр.

\textbf{1158.} Установить, какая линия является сечением гиперболического параболоида $\frac{x^2}{2}-\frac{z^2}{3}=y$ плоскостью $3 x-3 y+4 z+2=0$, и найти ее центр.