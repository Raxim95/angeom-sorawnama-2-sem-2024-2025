\section{A}\label{a}

\textbf{473.} \(\frac{x^{2}}{36} + \frac{y^{2}}{20} = 1\) ellips direktrisalarining tenglamalarini yozing.

\textbf{483.} \(\frac{x^{2}}{32} + \frac{y^{2}}{18} = 1\) elipsning \(M(4,3)\) nuqtasida o'tkazilgan urinmasining tenglamasi tuzilsin.

\textbf{531.1)} Quyidagi malumotlarga ko'ra giperbolaning kanonik tenglamasi tuzilsin: haqiqiy o'qi \(a = 5\) mavhum o'qi \(b = 3\);

\textbf{531.2)} Quyidagi malumotlarga ko'ra giperbolaning kanonik tenglamasi tuzilsin: fokuslari orasidagi masofa 10 ga , haqiqiy o'qi esa 8 ga teng.

\textbf{532.1)} Quyidagi malumotlarga ko'ra giperbolaning kanonik tenglamasi tuzilsin: ekssentrisiteti \(e = \frac{12}{13}\) haqiqiy o'qi 48 ga teng.

\textbf{532.2)} Quyidagi malumotlarga ko'ra giperbolaning kanonik tenglamasi tuzilsin: haqiqiy o'qi 16 ga, asimptotasi bilan abssissa o'qi orasidagi \(\varphi\).

\textbf{533.} Teng tomonli giperbolaning ekssentrisiteti hisoblansin.

\textbf{534.} Giperbola asimptotalarining tenglamalari \(y = \pm \frac{5}{12}x\) va giperbolada yotuvchi \(M(24,5)\) nuqta berilgan. Giperbola tenglamasi tuzilsin.

\textbf{535.} \(\frac{x^{2}}{25} - \frac{y^{2}}{144} = 1\) giperbolaning fokuslarini aniqlang.

\textbf{536.} \(\frac{x^{2}}{225} - \frac{y^{2}}{64} = - 1\) giperbolaning fokuslarini aniqlang.

\textbf{537.1)} Quyidagi malumotlarga ko'ra giperbolaning kanonik tenglamasi tuzilsin: direktrisalari orasidagi masofa \(\frac{32}{5}\) ga teng va ekssentrisiteti \(e = \frac{5}{4}\);

\textbf{537.2)} Quyidagi malumotlarga ko'ra giperbolaning kanonik tenglamasi tuzilsin: asimptotalari orasidagi burchak \(60^{\circ}\) ga teng va \(c = 2\sqrt{3}\) giperbolaning kanonik tenglamasi tuzilsin

\textbf{627.} \(y^{2} = 4x\) parabola fokusining koordinatalarini aniqlang.

\textbf{628.} \(x^{2} = 4y\) parabola fokusining koordinatalarini aniqlang.

\textbf{629.} \(y^{2} = - 8x\) parabola fokusining koordinatalarini aniqlang.

\textbf{630.} \(y^{2} = 6x\) parabola direktrisasi tenglamasini tuzing.

\textbf{635.1)} Parabolaning tenglamasini tuzing agar: parabolaning uchidan fokusigacha bo'lgan masofa 3 ga teng va parabola \(Ox\) o'qiga nisbatan simmetrik bo'lib, \(Oy\) o'qiga urinsa;

\textbf{635.2)} Parabolaning tenglamasini tuzing agar: fokusi \((5,0)\) nuqtada bo'lib, ordinatalar o'qi direktrisa bo'lsa;

\textbf{635.3)} Parabolaning tenglamasini tuzing agar: parabola \(Ox\) o'qiga nisbatan simmetrik bo'lib, \(M(1; - 4)\) nuqtadan va koordinatalar boshidan o'tadi;

\textbf{635.4)} Parabolaning tenglamasini tuzing agar: parabolaning fokusi \((0,2)\) nuqtada va uchi koordiniatalar boshida yotadi;

\textbf{635.5)} Parabolaning tenglamasini tuzing agar: parabola \(Oy\) o'qiga nisbatan simmetrik bo'lib, \(M(6, - 2)\) nuqtadan va koordinatalar boshidan o'tadi.

\textbf{636.} \(y^{2} = 8x\) paraboladagi fokal radius vektori 20 ga teng bo'lgan nuqta topilsin.

\section{B}\label{b}

\textbf{470.} O'qlari koordinata o'qlari bilan ustma - ust tushuvchi va \(P(2,2);Q(3,1)\) nuqtalar orqali o'tuvchi ellips tenglamasi tuzilsin.

\textbf{471.} Katta o'qi 2 birlikka teng, fokuslari \(F_{1}(0,1),F_{2}(1,0)\) nuqtalarda bo'lgan ellipsning tenglamasi tuzilsin.

\textbf{472.} Ellips fokuslarining biridan katta o'qi uchlarigacha masofalar mos ravishda 7 va 1 ga teng. Bu ellips ning tenglamasini tuzing.

\textbf{485.} \(\frac{x^{2}}{16} + \frac{y^{2}}{9} = 1\) ellipsning \(x + y - 1 = 0\) to'g'ri chiziqqa parallel bo'lgan urinmalarini aniqlang.

\textbf{538.} Giperbolaning haqiqiy o'qiga perpendikular bo'lgan va giperbola fokusidan o'tgan vatar uzunligi topilsin.

\textbf{539.} \(\frac{x^{2}}{49} + \frac{y^{2}}{24} = 1\) ellips bilan fokusdosh va ekssentrisiteti \(e = \frac{5}{4}\) bo'lgan giperbolaning tenglamasi yozilsin.

\textbf{543.1)} Giperbolaning yarim o'qlarini toping, agar: fokuslari orasidagi masofa 8 ga va direktrisalari orasidagi masofa 6 ga teng;

\textbf{543.2)} Giperbolaning yarim o'qlarini toping, agar: direktrisalari \(x = \pm 3\sqrt{2}\) tenglamalar bilan berilgan va asimptotalari orasidagi burchak - to'g'ri burchak;

\textbf{543.3)} Giperbolaning yarim o'qlarini toping, agar: asimptotalari \(y = \pm 2x\) tenglamalar bilan berilgan va fokuslari markazdan 5 birlik masofada;

\textbf{543.4)} Giperbolaning yarim o'qlarini toping, agar: asimptotalari \(y = \pm \frac{5}{3}x\) tenglamalar bilan berilgan va giperbola \(N(6,9)\) nuqtadan o'tadi.

\textbf{545.1)} Giperbolaning asimptotalari orasidagi burchagi topilsin, agar: ekssentrisiteti \(e = 2\);

\textbf{545.2)} Giperbolaning asimptotalari orasidagi burchagi topilsin, agar: fokuslari orasidagi masofa direktrisalari orasidagi masofadan ikki marta katta.

\textbf{547.1)} \(\frac{x^{2}}{16} - \frac{y^{2}}{9} = 1\) giperbolada fokal radiuslari o'zaro perpendikular bo'lgan nuqta topilsin.

\textbf{547.2)} \(\frac{x^{2}}{16} - \frac{y^{2}}{9} = 1\) giperbolada chap fokusgacha bo'lgan masofasi o'ng fokusgacha bo'lgan nuqta topilsin.

\textbf{549.} \(\frac{x^{2}}{9} - \frac{y^{2}}{4} = 1\) giperbolaning \(M(5,1)\) nuqtada teng ikkiga bo'linadigan vatarining tenglamasi tuzilsin.

\textbf{552.} \(\frac{x^{2}}{5} - \frac{y^{2}}{4} = 1\) giperbolaga \((5, - 4)\) nuqtada urinadigan to'g'ri chiziq tenglamasi yozilsin.

\textbf{553.} \(x^{2} - y^{2} = 8\) giperbolaga \(M(3, - 1)\) nuqtada urinadigan to'g'ri chiziq tenglamasi yozilsin.

\textbf{654.1)} Parabola uchining koordinatalari, parametri va o'qining yo'nalishi aniqlansin: \(y^{2} - 10x - 2y - 19 = 0\);

\textbf{654.2)} Parabola uchining koordinatalari, parametri va o'qining yo'nalishi aniqlansin: \(y^{2} - 6x + 14y + 49 = 0\),

\textbf{654.3)} Parabola uchining koordinatalari, parametri va o'qining yo'nalishi aniqlansin: \(y^{2} + 8x - 16 = 0\),

\textbf{654.4)} Parabola uchining koordinatalari, parametri va o'qining yo'nalishi aniqlansin: \(x^{2} - 6x - 4y + 29 = 0\),

\textbf{654.5)} Parabola uchining koordinatalari, parametri va o'qining yo'nalishi aniqlansin: \(y = Ax^{2} + Bx + C\),

\textbf{654.6)} Parabola uchining koordinatalari, parametri va o'qining yo'nalishi aniqlansin: \(y = x^{2} - 8x + 15\),

\textbf{654.7)} Parabola uchining koordinatalari, parametri va o'qining yo'nalishi aniqlansin: \(y = x^{2} + 6x\).

\textbf{687.} Beshta nuqtadan o'tuvchi ikkinchi tartibli chiziqning tenglamasi tuzilsin: \((0,0),(0,1),(1,0),(2, - 5),( - 5,2)\).

\textbf{689.} \(5x^{2} - 3xy + y^{2} - 3x + 2y - 5 = 0\) chiziqning \(x - 2y - 1 = 0\) to'g'ri chiziq bilan kesishishidan hosil qilingan vatarning o'rtasidan o'tadigan diametr tenglamasi yozilsin.

\textbf{722.} ITECH turi, o'lchovlari va joylashishi aniqlansin: \(5x^{2} + 4xy + 8y^{2} - 32x - 56y + 80 = 0\).

\textbf{723.} ITECH turi, o'lchovlari va joylashishi aniqlansin: \(9x^{2} + 24xy + 16y^{2} - 230x + 110y - 475 = 0\).

\textbf{724.} ITECH turi, o'lchovlari va joylashishi aniqlansin: \(5x^{2} + 12xy - 12x - 22y - 19 = 0\).

\textbf{725.} ITECH turi, o'lchovlari va joylashishi aniqlansin: \(x^{2} - 2xy + y^{2} - 10x - 6y + 25 = 0\).

\textbf{726.} ITECH turi, o'lchovlari va joylashishi aniqlansin: \(x^{2} - 5xy + 4y^{2} + x + 2y - 2 = 0\).

\textbf{727.} ITECH turi, o'lchovlari va joylashishi aniqlansin: \(4x^{2} - 12xy + 9y^{2} - 2x + 3y - 2 = 0\).

\textbf{728.1)} ITECH turi, o'lchovlari va joylashishi aniqlansin: \(2x^{2} + 4xy + 5y^{2} - 6x - 8y - 1 = 0\);

\textbf{728.2)} ITECH turi, o'lchovlari va joylashishi aniqlansin: \(5x^{2} + 8xy + 5y^{2} - 18x - 18y + 9 = 0\);

\textbf{728.3)} ITECH turi, o'lchovlari va joylashishi aniqlansin: \(5x^{2} + 6xy + 5y^{2} - 16x - 16y - 16 = 0\);

\textbf{728.4)} ITECH turi, o'lchovlari va joylashishi aniqlansin: \(6xy - 8y^{2} + 12x - 26y - 11 = 0\);

\textbf{728.5)} ITECH turi, o'lchovlari va joylashishi aniqlansin: \(7x^{2} + 16xy - 23y^{2} - 14x - 16y - 218 = 0\);

\textbf{728.6)} ITECH turi, o'lchovlari va joylashishi aniqlansin: \(7x^{2} - 24xy - 38x + 24y + 175 = 0\);

\textbf{728.7)} ITECH turi, o'lchovlari va joylashishi aniqlansin: \(9x^{2} + 24xy + 16y^{2} - 40x - 30y = 0\);

\textbf{728.8)} ITECH turi, o'lchovlari va joylashishi aniqlansin: \(x^{2} + 2xy + y^{2} - 8x + 4 = 0\);

\textbf{728.9)} ITECH turi, o'lchovlari va joylashishi aniqlansin: \(4x^{2} - 4xy + y^{2} - 2x - 14y + 7 = 0\).

\textbf{1752.1)} Lagranj usulidan foydalanib, tenglamalarni kvadratlar yig'indisi shakliga keltirib, quyidagi sirtlarning ko'rinishi aniqlansin: \(4x^{2} + 6y^{2} + 4z^{2} + 4xz - 8y - 4z + 3 = 0\);

\textbf{1752.2)} Lagranj usulidan foydalanib, tenglamalarni kvadratlar yig'indisi shakliga keltirib, quyidagi sirtlarning ko'rinishi aniqlansin: \(x^{2} + 5y^{2} + z^{2} + 2xy + 6xz + 2yz - 2x + 6y - 10z = 0\);

\textbf{1752.3)} Lagranj usulidan foydalanib, tenglamalarni kvadratlar yig'indisi shakliga keltirib, quyidagi sirtlarning ko'rinishi aniqlansin: \(x^{2} + y^{2} - 3z^{2} - 2xy - 6xz - 6yz + 2x + 2y + 4z = 0\);

\textbf{1752.4)} Lagranj usulidan foydalanib, tenglamalarni kvadratlar yig'indisi shakliga keltirib, quyidagi sirtlarning ko'rinishi aniqlansin: \(x^{2} - 2y^{2} + z^{2} + 4xy - 8xz - 4yz - 14x - 4y + 14z + 16 = 0\);

\textbf{1752.5)} Lagranj usulidan foydalanib, tenglamalarni kvadratlar yig'indisi shakliga keltirib, quyidagi sirtlarning ko'rinishi aniqlansin: \(2x^{2} + y^{2} + 2z^{2} - 2xy - 2yz + x - 4y - 3z + 2 = 0\);

\textbf{1752.6)} Lagranj usulidan foydalanib, tenglamalarni kvadratlar yig'indisi shakliga keltirib, quyidagi sirtlarning ko'rinishi aniqlansin: \(x^{2} - 2y^{2} + z^{2} + 4xy - 10xz + 4yz + x + y - z = 0\);

\textbf{1752.7)} Lagranj usulidan foydalanib, tenglamalarni kvadratlar yig'indisi shakliga keltirib, quyidagi sirtlarning ko'rinishi aniqlansin: \(2x^{2} + y^{2} + 2z^{2} - 2xy - 2yz + 4x - 2y = 0\);

\textbf{1752.8)} Lagranj usulidan foydalanib, tenglamalarni kvadratlar yig'indisi shakliga keltirib, quyidagi sirtlarning ko'rinishi aniqlansin: \(x^{2} - 2y^{2} + z^{2} + 4xy - 10xz + 4yz + 2x + 4y - 10z - 1 = 0\);

\textbf{1752.9)} Lagranj usulidan foydalanib, tenglamalarni kvadratlar yig'indisi shakliga keltirib, quyidagi sirtlarning ko'rinishi aniqlansin: \(x^{2} + y^{2} + 4z^{2} + 2xy + 4xz + 4yz - 6z + 1 = 0\);

\textbf{1752.10)} Lagranj usulidan foydalanib, tenglamalarni kvadratlar yig'indisi shakliga keltirib, quyidagi sirtlarning ko'rinishi aniqlansin: \(4xy + 2x + 4y - 6z - 3 = 0\);

\textbf{1752.11)} Lagranj usulidan foydalanib, tenglamalarni kvadratlar yig'indisi shakliga keltirib, quyidagi sirtlarning ko'rinishi aniqlansin: \(xy + xz + yz + 2x + 2y - 2z = 0\).

\textbf{1753.1)} Parallel ko'chirish va burish almashtirishlari yoki hadlarni gruppalash yordamida quyidagi sirtlarning ko'rinishi va joylashishi aniqlansin: \(z = 2x^{2} - 4y^{2} - 6x + 8y + 1\);

\textbf{1753.2)} Parallel ko'chirish va burish almashtirishlari yoki hadlarni gruppalash yordamida quyidagi sirtlarning ko'rinishi va joylashishi aniqlansin: \(z = x^{2} + 3y^{2} - 6y + 1\);

\textbf{1753.3)} Parallel ko'chirish va burish almashtirishlari yoki hadlarni gruppalash yordamida quyidagi sirtlarning ko'rinishi va joylashishi aniqlansin: \(x^{2} + 2y^{2} - 3z^{2} + 2x + 4y - 6z = 0\);

\textbf{1753.4)} Parallel ko'chirish va burish almashtirishlari yoki hadlarni gruppalash yordamida quyidagi sirtlarning ko'rinishi va joylashishi aniqlansin: \(x^{2} + 2xy + y^{2} - z^{2} = 0\);

\textbf{1753.5)} Parallel ko'chirish va burish almashtirishlari yoki hadlarni gruppalash yordamida quyidagi sirtlarning ko'rinishi va joylashishi aniqlansin: \(z^{2} = 3x + 4y + 5\);

\textbf{1753.6)} Parallel ko'chirish va burish almashtirishlari yoki hadlarni gruppalash yordamida quyidagi sirtlarning ko'rinishi va joylashishi aniqlansin: \(z = x^{2} + 2xy + y^{2} + 1\);

\textbf{1753.7)} Parallel ko'chirish va burish almashtirishlari yoki hadlarni gruppalash yordamida quyidagi sirtlarning ko'rinishi va joylashishi aniqlansin: \(z^{2} = x^{2} + 2xy + y^{2} + 1\);

\textbf{1753.8)} Parallel ko'chirish va burish almashtirishlari yoki hadlarni gruppalash yordamida quyidagi sirtlarning ko'rinishi va joylashishi aniqlansin: \(x^{2} + 4y^{2} + 9z^{2} - 6x + 8y - 18z - 14 = 0\);

\textbf{1753.9)} Parallel ko'chirish va burish almashtirishlari yoki hadlarni gruppalash yordamida quyidagi sirtlarning ko'rinishi va joylashishi aniqlansin: \(2xy + z^{2} - 2z + 1 = 0\);

\textbf{1753.10)} Parallel ko'chirish va burish almashtirishlari yoki hadlarni gruppalash yordamida quyidagi sirtlarning ko'rinishi va joylashishi aniqlansin: \(x^{2} + y^{2} - z^{2} - 2xy + 2z - 1 = 0\);

\textbf{1753.11)} Parallel ko'chirish va burish almashtirishlari yoki hadlarni gruppalash yordamida quyidagi sirtlarning ko'rinishi va joylashishi aniqlansin: \(x^{2} + 4y^{2} - z^{2} - 10x - 16y + 6z + 16 = 0\);

\textbf{1753.12)} Parallel ko'chirish va burish almashtirishlari yoki hadlarni gruppalash yordamida quyidagi sirtlarning ko'rinishi va joylashishi aniqlansin: \(2xy + 2x + 2y + 2z - 1 = 0\);

\textbf{1753.13)} Parallel ko'chirish va burish almashtirishlari yoki hadlarni gruppalash yordamida quyidagi sirtlarning ko'rinishi va joylashishi aniqlansin: \(3x^{2} + 6x - 8y + 6z - 7 = 0\);

\textbf{1753.14)} Parallel ko'chirish va burish almashtirishlari yoki hadlarni gruppalash yordamida quyidagi sirtlarning ko'rinishi va joylashishi aniqlansin: \(x^{2} + y^{2} + 2z^{2} + 2xy + 4z = 0\);

\textbf{1753.15)} Parallel ko'chirish va burish almashtirishlari yoki hadlarni gruppalash yordamida quyidagi sirtlarning ko'rinishi va joylashishi aniqlansin: \(3x^{2} + 3y^{2} + 3z^{2} - 6x + 4y - 1 = 0\);

\textbf{1753.16)} Parallel ko'chirish va burish almashtirishlari yoki hadlarni gruppalash yordamida quyidagi sirtlarning ko'rinishi va joylashishi aniqlansin: \(3x^{2} + 3y^{2} - 6x + 4y - 1 = 0\);

\textbf{1753.17)} Parallel ko'chirish va burish almashtirishlari yoki hadlarni gruppalash yordamida quyidagi sirtlarning ko'rinishi va joylashishi aniqlansin: \(3x^{2} + 3y^{2} - 3z^{2} - 6x + 4y + 4z + 3 = 0\);

\textbf{1753.18)} Parallel ko'chirish va burish almashtirishlari yoki hadlarni gruppalash yordamida quyidagi sirtlarning ko'rinishi va joylashishi aniqlansin: \(4x^{2} - y^{2} - 4x + 4y - 3 = 0\);

\section{C}\label{c}

\textbf{477.} \(\frac{x^{2}}{a^{2}} + \frac{y^{2}}{b^{2}} = 1\) ellipsning \(F(c,0)\) fokusi orqali katta o'qiga perpendikular bo'lgan vatar o'tkazilgan. Bu vatar uzunligini toping.

\textbf{478.} \(\frac{x^{2}}{a^{2}} + \frac{y^{2}}{b^{2}} = 1\) ellipsga ichki chizilgan kvadrat tomonining uzunligi hisoblansin.

\textbf{480.} \(\frac{x^{2}}{100} + \frac{y^{2}}{64} = 1\) ellipsning \(2x - y + 7 = 0,2x - y - 1 = 0\) vatarlarining o'rtalari orqali o'tadigan to'g'ri chiziq tenglamasini tuzing.

\textbf{490.} \(Ax + By + C = 0\) to'g'ri chiziqning \(\frac{x^{2}}{a^{2}} + \frac{y^{2}}{b^{2}} = 1\), ellipsga urinma bo'lishi uchun zaruriy va yetarli sharti topilsin.

\textbf{510.}* \(Ax + By + C = 0\) to'g'ri chiziq qanday zaruriy va yetarli shart bajarilganda \(\frac{x^{2}}{a^{2}} + \frac{y^{2}}{b^{2}} = 1\) ellips bilan 1) kesishadi; 2) kesishmaydi.

\textbf{541.} Giperbolaning asimptotalaridan direktrisalari ajratgan kesmalar (giperbolaning markazidan hisoblanganda) giperbolaning haqiqiy yarim o'qiga teng ekanligi isbotlansin. Bu xossadan foydalanib, giperbolaning direktrisalari yasalsin.

\textbf{556.} \(\frac{x^{2}}{a^{2}} - \frac{y^{2}}{b^{2}} = 1\) giperbolaning fokuslaridan urinmasigacha bo'lgan masofalarning ko'paytmasi topilsin.

\textbf{566.} Giperbola asimptotalarining tenglamalari \(y = \pm \frac{1}{2}x\) va urinmalardan birining tenglamasi \(5x - 6y - 8 = 0\) ma'lum bo'lsa, giperbola tenglamasini tuzing.

\textbf{641.} \(Ax + By + C = 0\) to'g'ri chiziq \(y^{2} = 2px\) parabolaga urinishi uchun zaruriy va yetarli shartni toping.

\textbf{642.} Berilgan \(y = kx + b\) to'g'ri chiziqqa parallel va \(y^{2} = 2px\) parabolaga urinadigan to'g'ri chiziqning tenglamasini yozing.

\textbf{643.}* \(y^{2} = 4x\) parabola bilan \(\frac{x^{2}}{8} + \frac{y^{2}}{2} = 1\) ellipsning umumiy urinmalarini aniqlang.

\textbf{650.} Parabolaning ix'tiyoriy urinmasi direktrisasini va o'qqa perpendikular bo'lgan fokal vatarni fokusdan teng uzoqlikdagi nuqtalarda kesishini isbotlang.

\textbf{655.}* Umumiy fokusga va ustma - ust tushgan, lekin qarama - qarshi yo'nalgan o'qlarga ega bo'lgan parabolalarning to'g'ri burchak ostida kesishishi isbotlansin.

\textbf{698.} Giperbolaning asimptotalari topilsin: \(10x^{2} + 21xy + 9y^{2} - 41x - 39y + 4 = 0\).

\textbf{699.1)} Giperbolaning asimptotalari topilsin: \(x^{2} - 3xy - 10y^{2} + 6x - 8y = 0\);

\textbf{699.2)} Giperbolaning asimptotalari topilsin: \(3x^{2} + 2xy - y^{2} + 8x + 10y - 14 = 0\);

\textbf{699.3)} Giperbolaning asimptotalari topilsin: \(3x^{2} + 7xy + 4y^{2} + 5x + 2y - 6 = 0\);

\textbf{699.4)} Giperbolaning asimptotalari topilsin: \(10xy - 2y^{2} + 6x + 4y + 21 = 0\)

\textbf{1754.} Quyidagi sirtlarning kanonik tenglamasi va joylashishini aniqlansin: \(x^{2} + 5y^{2} + z^{2} + 2xy + 6xz + 2yz - 2x + 6y + 2z = 0\).

\textbf{1755.} Quyidagi sirtlarning kanonik tenglamasi va joylashishini aniqlansin: \(2x^{2} + y^{2} + 2z^{2} - 2xy + 2yz + 4x - 2y = 0\).

\textbf{1756.} Quyidagi sirtlarning kanonik tenglamasi va joylashishini aniqlansin: \(x^{2} + y^{2} + 4z^{2} + 2xy + 4xz + 4yz - 6z + 1 = 0\).

\textbf{1757.} Quyidagi sirtlarning kanonik tenglamasi va joylashishini aniqlansin: \(4x^{2} + 9y^{2} + z^{2} - 12xy - 6yz + 4zx + 4x - 6y + 2z - 5 = 0\).

\textbf{1758.} Quyidagi sirtlarning kanonik tenglamasi va joylashishini aniqlansin: \(7x^{2} + 6y^{2} + 5z^{2} - 4xy - 4yz - 6x - 24y + 18z + 30 = 0\).

\textbf{1759.} Quyidagi sirtlarning kanonik tenglamasi va joylashishini aniqlansin: \(2x^{2} + 2y^{2} - 5z^{2} + 2xy - 2x - 4y - 4z + 2 = 0\).

\textbf{1760.} Quyidagi sirtlarning kanonik tenglamasi va joylashishini aniqlansin: \(x^{2} - 2y^{2} + z^{2} + 4xy - 8xz - 4yz - 14x - 4y + 14z + 16 = 0\).

\textbf{1761.} Quyidagi sirtlarning kanonik tenglamasi va joylashishini aniqlansin: \(2x^{2} + 2y^{2} + 3z^{2} + 4xy + 2xz + 2yz - 4x + 6y - 2z + 3 = 0\).

\textbf{1762.} Quyidagi sirtlarning kanonik tenglamasi va joylashishini aniqlansin: \(2x^{2} + 5y^{2} + 2z^{2} - 2xy + 2yz - 4xz + 2x - 10y - 2z - 1 = 0\).

\textbf{1763.1)} Quyidagi sirtlarning kanonik tenglamasi va joylashishini aniqlansin: \(x^{2} + 5y^{2} + z^{2} + 2xy + 6xz + 2yz - 2x + 6y + 2z = 0\);

\textbf{1763.2)} Quyidagi sirtlarning kanonik tenglamasi va joylashishini aniqlansin: \(5x^{2} + 2y^{2} + 5z^{2} - 4xy - 2xy - 4yz + 10x - 4y - 2z + 4 = 0\);

\textbf{1763.3)} Quyidagi sirtlarning kanonik tenglamasi va joylashishini aniqlansin: \(x^{2} - 2y^{2} + z^{2} + 4xy - 10xz + 4yz + 2x + 4y - 10z - 1 = 0\).

\textbf{1764.} Quyidagi sirtlarning kanonik tenglamasi va joylashishini aniqlansin: \(5x^{2} - y^{2} + z^{2} + 4xy + 6xz + 2x + 4y + 6z - 8 = 0\).

\textbf{1765.} Quyidagi sirtlarning kanonik tenglamasi va joylashishini aniqlansin: \(2x^{2} + 10y^{2} - 2z^{2} + 12xy + 8yz + 12x + 4y + 8z - 1 = 0\).
