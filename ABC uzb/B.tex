\section{Ellipsning kanonik tenglamasi}



\textbf{448.} Ikkita uchi $x^2+5 y^2=20$ ellipsning fokuslarida yotuvchi, qolgan ikkitasi esa uning kichik o‘qi uchlari bilan ustma-ust tushuvchi to‘rtburchakning yuzi hisoblansin.

\textbf{456.} $\varepsilon=\frac{2}{3}$ ellipsning ekssentrisiteti, $M$ ellips nuqtasining fokal radiusi 10 ga teng. $M$ nuqtadan shu fokusga mos direktrisagacha bo‘lgan masofani hisoblang.

\textbf{457.} $\varepsilon=\frac{2}{5}$ ellipsning ekssentrisiteti, ellipsning $M$ nuqtasidan direktrisagacha bo‘lgan masofa 20 ga teng. $M$ nuqtadan shu direktrisa bilan bir tomonlama fokusgacha bo‘lgan masofani hisoblang.

\textbf{460.} Ellipsning ekssentrisiteti $\varepsilon=\frac{1}{3}$, uning markazi koordinatalar boshi bilan ustma-ust tushadi, fokuslaridan biri $ (-2; 0) $. Abssissasi 2 ga teng bo‘lgan ellipsning $M_1$ nuqtasidan berilgan fokusga mos direktrisagacha bo‘lgan masofani ayiring.

\textbf{461.} Ellipsning ekssentrisiteti $\varepsilon=\frac{1}{2}$, uning markazi koordinatalar boshi bilan ustma-ust tushadi, direktrisalardan biri $x=16$ tenglama bilan berilgan. Abssissasi -4 ga teng bo‘lgan ellipsning $M_1$ nuqtasidan berilgan direktrisa bilan bir tomonlama fokusgacha bo‘lgan masofani hisoblang.

\textbf{464.} $\frac{x^2}{25}+\frac{y^2}{15}=1$ ellipsning fokusi orqali uning katta o‘qiga perpendikulyar o‘tkazilgan. Bu perpendikulyarning ellips bilan kesishgan nuqtalaridan fokuslargacha bo‘lgan masofalar aniqlansin.

\textbf{466.1)} Ellipsdagi ekssentrisitetni aniqlang, agar: uning kichik o‘qi fokuslardan $60^{\circ}$ burchak ostida ko‘rinadi;

\textbf{466.2)} Ellipsdagi ekssentrisitetni aniqlang, agar: fokuslari orasidagi kesmaning o‘zi kichik o‘qning uchidan to‘g‘ri burchak ostida ko‘rinadi.;

\textbf{466.3)} Ellipsdagi ekssentrisitetni aniqlang, agar: direktrisalar orasidagi masofa fokuslar orasidagi masofadan uch marta katta.;

\textbf{466.4)} Ellipsdagi ekssentrisitetni aniqlang, agar: ellips markazidan uning direktrisasiga tushirilgan perpendikulyar kesmasi ellipsning uchi bilan teng ikkiga bo‘linadi.

\textbf{473.1)} Quyidagilarni bilgan holda ellips tenglamasini tuzing: uning katta o‘qi 26 ga teng va fokuslari $F_1 (-10; 0), F2 (14; 0) $;

\textbf{473.2)} Quyidagilarni bilgan holda ellips tenglamasini tuzing: uning kichik o‘qi 2 ga teng va fokuslari $F_1 (-1;-1) $, $F_2 (1; 1) $;

\textbf{473.3)} Quyidagilarni bilgan holda ellips tenglamasini tuzing: uning fokuslari $F_1\left(-2; \frac{3}{2}\right), F_2\left(2;-\frac{3}{2}\right) $ va ekssentrisitet $\varepsilon=\frac{\sqrt{2}}{2}$;

\textbf{473.4)} Quyidagilarni bilgan holda ellips tenglamasini tuzing: uning fokuslari $F_1 (1; 3), F_2 (3; 1) $ va direktrisalar orasidagi masofa $12 \sqrt{2}$ ga teng.

\textbf{477.} Agar ellipsning ekssentrisiteti $\varepsilon=\frac{1}{2}$ va fokusi $F(3 ; 0)$ va unga mos direktrisa tenglamasi $x+y-1=0$ ma’lum bo‘lsa, uning tenglamasi tuzilsin.

\textbf{478.} $M_1 (2;-1)$ nuqta fokusi $F (1;0)$ bo‘lgan ellipsda yotadi. Bu fokusga mos direktrisa esa $2x-y-10=0$ tenglama bilan berilgan. Shu ellipsning tenglamasi tuzilsin.


\section{Giperbolaning kanonik tenglamasi}



\textbf{522.} $\frac{x^2}{80}-\frac{y^2}{20}=1$ giperbolada $M_1 (10;-\sqrt{5}) $ nuqta berilgan. $M_1$ nuqtaning fokal radiuslari yotgan to‘g‘ri chiziqlarning tenglamalari tuzilsin.

\textbf{528.} $\frac{x^2}{64}-\frac{y^2}{36}=1$ giperbolaning o‘ng fokusigacha bo‘lgan masofasi 4,5 ga teng bo‘lgan nuqtalari aniqlansin.

\textbf{533.} Teng tomonli giperbolaning ekssentrisiteti aniqlansin.

\textbf{536.} Fokuslari $\frac{x^2}{100}+\frac{y^2}{64}=1$ ellipsning uchlarida yotuvchi, direktrisalari esa shu ellipsning fokuslaridan o‘tuvchi giperbolaning tenglamasi tuzilsin.

\textbf{543.1)} Quyidagilarni bilgan holda giperbola tenglamasini tuzing: uning uchlari orasidagi masofa 24 ga teng va fokuslari $F_1 (-10; 2), F_2 (16; 2) $;

\textbf{543.2)} Quyidagilarni bilgan holda giperbola tenglamasini tuzing: fokuslar $F_1 (3; 4), F_2 (-3;-4)$ va direktrisalar orasidagi masofa 3,6;

\textbf{543.3)} Quyidagilarni bilgan holda giperbola tenglamasini tuzing: Asimptotalar orasidagi burchak $90^{\circ}$ ga teng va fokuslar $F_1 (4;-4), F_2 (-2; 2) $.

\textbf{547.} Agar giperbolaning ekssentrisiteti $\varepsilon=\sqrt{5}$, fokusi $F (2;-3) $ va unga mos direktrisasining tenglamasi $3 x-y+3=0$ ma’lum bo‘lsa, uning tenglamasini tuzing.

\textbf{548.} $M_1 (1;-2) $ nuqta fokusi $F (-2; 2) $, unga mos direktrisa esa $2x-y-1=0$ tenglama bilan berilgan giperbolaga tegishli. Bu giperbolaning tenglamasi tuzilsin.

\textbf{560.} $\frac{x^2}{16}-\frac{y^2}{64}=1$ giperbolaga $10 x-3 y+9=0$ to‘g‘ri chiziqqa parallel bo‘lgan urinmalarning tenglamalarini tuzing.

\textbf{561.} $\frac{x^2}{16}-\frac{y^2}{8}=-1$ giperbolaga $2 x+4 y-5=0$ to‘g‘ri chiziqqa parallel urinmalar o‘tkazing va ular orasidagi $d$ masofani hisoblang.

\textbf{563.} Ushbu $x^2-y^2=16$ giperbolaga $A (-1;-7)$ nuqtadan o‘tkazilgan urinmalar tenglamasi tuzilsin.



\section{Parabolaning kanonik tenglamasi}



\textbf{600.} Agar parabolaning fokusi $F (7; 2) $ va direktrisa $x-5=0$ tenglamasi berilgan bo'lsa uning tenglamasini tuzing.

\textbf{601.} Agar parabolaning fokusi $F (4;3) $ va direktrisa $y+1=0$ tenglamasi berilgan bo'lsa uning tenglamasini tuzing.

\textbf{602.} Agar parabolaning fokusi $F(2;-1) $ va direktrisa $x-y-1=0$ tenglamasi berilgan bo'lsa uning tenglamasini tuzing.

\textbf{603.} Berilgan parabola uchi $A(6;-3)$ va uning direktrisasining tenglamasi $3x-5y+1=0$ berilgan. Ushbu parabolaning $F$ fokusini toping.


\textbf{604.} Parabola uchi $A(-2;-1)$ va uning direktrisasining tenglamasi $x+2y-1=0$ berilgan. Ushbu parabolaning tenglamasini tuzing.

\textbf{613.} $y^2=8x$ parabolaning $2x+2y-3=0$ to'g'ri chizig'iga parallel urinmasining tenglamasini tuzing.

\textbf{614.} $x^2=16y$ parabolaning $2x+4y+7=0$ to'g'ri chizig'iga perpendikulyar bo'lgan urinmasining tenglamasini tuzing.

\textbf{619.} $A(5;9)$ nuqtadan $y^2=5x$ parabolaga o'tkazilgan urinmalarning urinish nuqtalarini tutashtiruvchi xordaning tenglamasini tuzing.



\section{ Ikkinchi tartibli chiziqning markazi }



\textbf{666.1)} Ushbu chiziqlar markaziy ekanligini ko'rsating va har bir chiziq uchun markaz koordinatalarini toping: $3x^2+5xy+y^2-8x-11y-7=0$.

\textbf{666.2)} Ushbu chiziqlar markaziy ekanligini ko'rsating va har bir chiziq uchun markaz koordinatalarini toping: $5 x^2+4 x y+2 y^2+20 x+20 y-18=0$;

\textbf{666.3)} Ushbu chiziqlar markaziy ekanligini ko'rsating va har bir chiziq uchun markaz koordinatalarini toping: $9 x^2-4 x y-7 y^2-12=0$;

\textbf{666.4)} Ushbu chiziqlar markaziy ekanligini ko'rsating va har bir chiziq uchun markaz koordinatalarini toping: $2 x^2-6 x y+5 y^2+22 x-36 y+11=0$.

\textbf{668.1)} Ushbu tenglamalar markaziy chiziqlarni ifodalashini ko‘rsating va har bir tenglamani koordinatalar boshini markazga ko‘chirgan holda o‘zgartiring: $3x^2-6xy+2y^2-4x+2y+1=0$.

\textbf{668.2)} Ushbu tenglamalar markaziy chiziqlarni ifodalashini ko‘rsating va har bir tenglamani koordinatalar boshini markazga ko‘chirgan holda o‘zgartiring: $6 x^2+4 x y+y^2+4 x-2 y+2=0$;

\textbf{668.3)} Ushbu tenglamalar markaziy chiziqlarni ifodalashini ko‘rsating va har bir tenglamani koordinatalar boshini markazga ko‘chirgan holda o‘zgartiring: $4 x^2+6 x y+y^2-10 x-10=0$;

\textbf{668.4)} Ushbu tenglamalar markaziy chiziqlarni ifodalashini ko‘rsating va har bir tenglamani koordinatalar boshini markazga ko‘chirgan holda o‘zgartiring: $4 x^2+2 x y+6 y^2+6 x-10 y+9=0$.



\section{Ikkinchi tartibli markaziy chiziq tenglamasini sodda shaklga keltirish}



\textbf{673.1)} Quyidagi tenglamaning tipini aniqlang, koordinata o‘qlarini parallel ko‘chirish orqali sodda shaklga keltiring; qanday geometrik obrazni ifodalashini aniqlang va eski hamda yangi koordinata o‘qlariga nisbatan chizmada tasvirlang: $4 x^2+9 y^2-40 x+36 y+100=0$;

\textbf{673.2)} Quyidagi tenglamaning tipini aniqlang, koordinata o‘qlarini parallel ko‘chirish orqali sodda shaklga keltiring; qanday geometrik obrazni ifodalashini aniqlang va eski hamda yangi koordinata o‘qlariga nisbatan chizmada tasvirlang: $9 x^2-16 y^2-54 x-64 y-127=0$;

\textbf{673.3)} Quyidagi tenglamaning tipini aniqlang, koordinata o‘qlarini parallel ko‘chirish orqali sodda shaklga keltiring; qanday geometrik obrazni ifodalashini aniqlang va eski hamda yangi koordinata o‘qlariga nisbatan chizmada tasvirlang: $9 x^2+4 y^2+18 x-8 y+49=0$;

\textbf{673.4)} Quyidagi tenglamaning tipini aniqlang, koordinata o‘qlarini parallel ko‘chirish orqali sodda shaklga keltiring; qanday geometrik obrazni ifodalashini aniqlang va eski hamda yangi koordinata o‘qlariga nisbatan chizmada tasvirlang: $4 x^2-y^2+8 x-2 y+3=0$;

\textbf{673.5)} Quyidagi tenglamaning tipini aniqlang, koordinata o‘qlarini parallel ko‘chirish orqali sodda shaklga keltiring; qanday geometrik obrazni ifodalashini aniqlang va eski hamda yangi koordinata o‘qlariga nisbatan chizmada tasvirlang: $2 x^2+3 y^2+8 x-6 y+11=0$.

\textbf{674.1)} Berilgan tenglamani sodda shaklga keltiring; tipini aniqlang; qanday geometrik obrazni ifodalashini aniqlang, eski hamda yangi koordinata o‘qlariga nisbatan chizmada tasvirlang: $32x^2+52xy-7y^2+180=0$.: $32 x^2+52 x y-7 y^2+180=0$;

\textbf{674.2)} Berilgan tenglamani sodda shaklga keltiring; tipini aniqlang; qanday geometrik obrazni ifodalashini aniqlang, eski hamda yangi koordinata o‘qlariga nisbatan chizmada tasvirlang: $32x^2+52xy-7y^2+180=0$.: $5 x^2-6 x y+5 y^2-32=0$;

\textbf{674.3)} Berilgan tenglamani sodda shaklga keltiring; tipini aniqlang; qanday geometrik obrazni ifodalashini aniqlang, eski hamda yangi koordinata o‘qlariga nisbatan chizmada tasvirlang: $32x^2+52xy-7y^2+180=0$.: $17 x^2-12 x y+8 y^2=0$;

\textbf{674.4)} Berilgan tenglamani sodda shaklga keltiring; tipini aniqlang; qanday geometrik obrazni ifodalashini aniqlang, eski hamda yangi koordinata o‘qlariga nisbatan chizmada tasvirlang: $32x^2+52xy-7y^2+180=0$.: $5 x^2+24 x y-5 y^2=0$;

\textbf{674.5)} Berilgan tenglamani sodda shaklga keltiring; tipini aniqlang; qanday geometrik obrazni ifodalashini aniqlang, eski hamda yangi koordinata o‘qlariga nisbatan chizmada tasvirlang: $32x^2+52xy-7y^2+180=0$.: $5 x^2-6 x y+5 y^2+8=0$.



\section{Parabolik tenglamani sodda shaklga keltirish}



\textbf{689.1)} Berilgan tenglama parabolik ekanligini ko'rsating; sodda shaklga keltiring; qanday geometrik obrazni ifodalashini aniqlang, eski hamda yangi koordinata o‘qlariga nisbatan chizmada tasvirlang: $9 x^2-24 x y+16 y^2-20 x+110 y-50=0$;

\textbf{689.2)} Berilgan tenglama parabolik ekanligini ko'rsating; sodda shaklga keltiring; qanday geometrik obrazni ifodalashini aniqlang, eski hamda yangi koordinata o‘qlariga nisbatan chizmada tasvirlang: $9 x^2+12 x y+4 y^2-24 x-16 y+3=0$;

\textbf{689.3)} Berilgan tenglama parabolik ekanligini ko'rsating; sodda shaklga keltiring; qanday geometrik obrazni ifodalashini aniqlang, eski hamda yangi koordinata o‘qlariga nisbatan chizmada tasvirlang: $16 x^2-24 x y+9 y^2-160 x+120 y+425=0$.

\textbf{690.1)} Berilgan tenglama parabolik ekanligini ko'rsating; sodda shaklga keltiring; qanday geometrik obrazni ifodalashini aniqlang, eski hamda yangi koordinata o‘qlariga nisbatan chizmada tasvirlang: $9 x^2+24 x y+16 y^2-18 x+226 y+209=0$;

\textbf{690.2)} Berilgan tenglama parabolik ekanligini ko'rsating; sodda shaklga keltiring; qanday geometrik obrazni ifodalashini aniqlang, eski hamda yangi koordinata o‘qlariga nisbatan chizmada tasvirlang: $x^2-2 x y+y^2-12 x+12 y-14=0$

\textbf{690.3)} Berilgan tenglama parabolik ekanligini ko'rsating; sodda shaklga keltiring; qanday geometrik obrazni ifodalashini aniqlang, eski hamda yangi koordinata o‘qlariga nisbatan chizmada tasvirlang: $4 x^2+12 x y+9 y^2-4 x-6 y+1=0$.

\textbf{693.1)} Berilgan tenglamalarning parabolik ekanligini ko‘rsating va ularning har birini $(\alpha x+\beta y)^2+2 D x+2 E y+F=0$ ko‘rinishda yozing: $x^2+4 x y+4 y^2+4 x+y-15=0 ;$

\textbf{693.2)} Berilgan tenglamalarning parabolik ekanligini ko‘rsating va ularning har birini $(\alpha x+\beta y)^2+2 D x+2 E y+F=0$ ko‘rinishda yozing: $9 x^2-6 x y+y^2-x+2 y-14=0$;

\textbf{693.3)} Berilgan tenglamalarning parabolik ekanligini ko‘rsating va ularning har birini $(\alpha x+\beta y)^2+2 D x+2 E y+F=0$ ko‘rinishda yozing: $25 x^2-20 x y+4 y^2+3 x-y+11=0$;

\textbf{693.4)} Berilgan tenglamalarning parabolik ekanligini ko‘rsating va ularning har birini $(\alpha x+\beta y)^2+2 D x+2 E y+F=0$ ko‘rinishda yozing: $16 x^2+16 x y+4 y^2-5 x+7 y=0$;

\textbf{693.5)} Berilgan tenglamalarning parabolik ekanligini ko‘rsating va ularning har birini $(\alpha x+\beta y)^2+2 D x+2 E y+F=0$ ko‘rinishda yozing: $9 x^2-42 x y+49 y^2+3 x-2 y-24=0$.



\section{Ikkinchi tartibli sirtlar}



\textbf{1157.} $\frac{x^2}{12}+\frac{y^2}{4}+\frac{z^2}{3}=1$ ellipsoidi va $2x-3y+4z-11=0$ tekisligining kesishish chizig‘i qanday chiziq ekanligini aniqlang va uning markazini toping.

\textbf{1158.} $\frac{x2}{2}-\frac{z2}{3}=y$ giperbolik paraboloidi va $3x-3y+4z+2=0$ tekisligining kesish chizig‘i qanday chiziq ekanligini aniqlang va uning markazini toping.