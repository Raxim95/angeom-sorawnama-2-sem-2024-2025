\section{Ellipsning kanonik tenglamasi}


\textbf{468.} Ellipsning yarim o‘qlari $a$, $b$ va markazi $C\left(x_0; y_0\right)$ nuqtada bo‘lib, simmetriya o‘qlari koordinata o‘qlariga parallel ekanligi ma’lum bo'lsa uning tenglamasini tuzing.

\textbf{484.} $m$ ning qanday qiymatlarida $y=-x+m$ chiziq: 1) $\frac{x^2}{20}+\frac{y^2}{5}=1$ ellipsni kesib o'tadi; 2) ellipsga urinadi 3) ellipsni kesib o'tmaydi.

\textbf{486.} $\frac{x^2}{a^2}+\frac{y^2}{b^2}=1$ ellipsning $M_1(x_1; y_1)$ nuqtasidagi urinmasining tenglamasini tuzing.

\textbf{487.} $\frac{x^2}{a^2}+\frac{y^2}{b^2}=1$ ellipsning bitta diametrini uchlariga o‘tkazilgan urinmalar parallel bo‘lishini isbotlang (ellipsning diametri deb uning markazidan o‘tuvchi xordagaaytiladi).

\textbf{490.} $\frac{x^2}{30}+\frac{y^2}{24}=1$ ellipsga $4x-2y+23=0$ parallel bo‘lgan urinmalarni o‘tkazing va ular orasidagi masofani hisoblang.

\textbf{492.} $A\left(\frac{10}{3}; \frac{5}{3}\right)$ nuqtadan $\frac{x2}{20}+\frac{y2}{5}=1$ ellipsga urinmalar o‘tkazilgan. Ularning tenglamalarini tuzing.

\textbf{497.} Ellips markazidan uning ixtiyoriy urinmasining fokal o‘q bilan kesishish nuqtasigacha va urinish nuqtasidan fokal o‘qqa tushirilgan perpendikulyar asosigacha bo‘lgan masofalar ko‘paytmasi o‘zgarmas kattalik bo‘lib, ellips katta yarim o‘qining kvadratiga tengligi isbotlansin.

\textbf{498.} Fokuslardan ellipsning istalgan urinmasigacha bo‘lgan masofalar ko‘paytmasi kichik yarim o‘qning kvadratiga tengligini isbotlang.



\section{Giperbolaning kanonik tenglamasi}



\textbf{537.} $\frac{x^2}{a^2}-\frac{y^2}{b^2}=1$ giperbolaning fokusidan asimptotagacha bo‘lgan masofa $b$ ga tengligini isbotlang.

\textbf{538.} $\frac{x^2}{a^2}-\frac{y^2}{b^2}=1$ giperbolaning ixtiyoriy nuqtasidan uning ikkita asimptotasigacha bo‘lgan masofalar ko‘paytmasi $\frac{a^2 b^2}{a^2+b^2}$ ga teng o‘zgarmas kattalik ekanligini isbotlang.

\textbf{539.} $\frac{x^2}{a^2}-\frac{y^2}{b^2}=1$ giperbolaning asimptotalari va uning ixtiyoriy nuqtasidan asimptotalarga parallel qilib o‘tkazilgan to‘g‘ri chiziqlar bilan chegaralangan parallelogrammning yuzi o‘zgarmas son bo‘lib $\frac{a b}{2}$ ga teng bo‘lishini isbotlang.

\textbf{540.} Agar giperbolaning yarim o‘qlari $a$ va $b$, markazi $C\left(x_0; y_0\right) $ va fokuslar quyidagi to‘g‘ri chiziqda joylashgan: 1) $O x$ o‘qiga parallel; 2) $O y$ o‘qiga parallel bo'lsa uning tenglamasini tuzing.

\textbf{555.} $m$ ning qanday qiymatlarida $y=\frac{5}{2} x+m$ to‘g‘ri chiziq $\frac{x^2}{9}-\frac{y^2}{36}=1$ giperbolani 1) kesib o‘tishini; 2) unga urinishini; 3) tashqarisidan o‘tishini aniqlang.

\textbf{556.} $y=k x+m$ to‘g‘ri chiziqning $\frac{x^2}{a^2}-\frac{y^2}{b^2}=1$ giperbolaga urinish shartini keltirib chiqaring.

\textbf{557.} $\frac{x^2}{a^2}-\frac{y^2}{b^2}=1$ giperbolaga uning $M_1\left(x_1; y_1\right) $ nuqtasidagi urinmasining tenglamasini tuzing.

\textbf{558.} Giperbolaning bitta diametr uchlaridan o‘tkazilgan urinmalar parallel bo‘lishini isbotlang.

\textbf{567.} Quyidagi ikki to‘g‘ri chiziqqa urinuvchi giperbolaning tenglamasi tuzilsin: $5x-6y-16=0$, $13x-10y-48=0$, bunda uning o‘qlari koordinata o‘qlari bilan ustma-ust tushadi.

\textbf{569.} $\frac{x^2}{a^2}-\frac{y^2}{b^2}=1$ giperbola va uning biror urinmasi berilgan: $P$-urinmaning $O x$ o‘qi bilan kesishish nuqtasi, $Q$ - urinish nuqtasining o‘sha o‘qdagi proyeksiyasi. $O P \cdot O Q=a^2$ ekanligini isbotlang.



\section{Parabolaning kanonik tenglamasi}



\textbf{609.} Burchak koeffitsiyenti $k$ ning qanday qiymatlarida $y=kx+2$ to‘g‘ri chiziq: 1) $y^2=4x$ parabolani kesib o'tadi; 2) unga urinadi; 3) bu parabola tashqarisidan o‘tadi.

\textbf{610.} $y^2=2 p x$ parabolaga $y=k x+b$ to‘g‘ri chiziq urinish shartini keltirib chiqaring.

\textbf{612.} $y^2=2 p x$ parabolaga uning $M_1\left(x_1; y_1\right) $ nuqtasidagi urinmasining tenglamasini tuzing.

\textbf{626.} Umumiy o‘qqa va uchlari orasida joylashgan umumiy fokusga ega bo‘lgan ikkita parabola to‘g‘ri burchak ostida kesishishini isbotlang.

\textbf{627.} O‘qlari o‘zaro perpendikulyar bo‘lgan ikkita parabola to‘rtta nuqtada kesishsa, bu nuqtalar bitta aylanada yotishini isbotlang.



\section{ Ikkinchi tartibli chiziq markazi }



\textbf{669.} $m$ va $n$ ning qanday qiymatlarida $m x^2+12 x y+9 y^2+4 x+n y-13=0$ tenglama: 1) markaziy chiziqni; 2) markazga ega bo'lmagan chiziq; 3) cheksiz ko‘p markazga ega bo‘lgan chiziqni ifodalaydi.

\textbf{670.} Дано уравнение линии $4 x^2-4 x y+y^2+6 x+1=0$. Определить, при каких значениях углового коэффициента $k$ прямая $y=k x:$ 1) пересекает эту линию в одной точке; 2) касается этой линии; 3) пересекает эту линию в двух точках; 4) не имеет общих точек с этой линией.
$4 x^2-4 x y+y^2+6 x+1=0$ ITECH tenglamasi berilgan. Burchak koeffitsiyenti $k$ ning qanday qiymatlarida $y=kx$ to‘g‘ri chiziq: 1) bu chiziqni bir nuqtada kesib o‘tishi; 2) shu chiziqqa urinadi; 3) bu chiziqni ikki nuqtada kesib o‘tadi; 4) bu to‘g‘ri chiziq bilan umumiy nuqtaga ega emas bólishini aniqlang.



\section{Ikkinchi tartibli markaziy chiziq tenglamasini sodda ko‘rinishga keltirish}



\textbf{676.1)} Berilgan tenglama kanonik ko‘rinishga keltirilsin; tipi aniqlansin; qanday geometrik obrazni ifodalashi aniqlansin; eski va yangi koordinatalar sistemasida geometrik obrazi tasvirlansin: $3 x^2+10 x y+3 y^2-2 x-14 y-13=0$;

\textbf{676.2)} Berilgan tenglama kanonik ko‘rinishga keltirilsin; tipi aniqlansin; qanday geometrik obrazni ifodalashi aniqlansin; eski va yangi koordinatalar sistemasida geometrik obrazi tasvirlansin: $25 x^2-14 x y+25 y^2+64 x-64 y-224=0$;

\textbf{676.3)} Berilgan tenglama kanonik ko‘rinishga keltirilsin; tipi aniqlansin; qanday geometrik obrazni ifodalashi aniqlansin; eski va yangi koordinatalar sistemasida geometrik obrazi tasvirlansin: $4 x y+3 y^2+16 x+12 y-36=0$;

\textbf{676.4)} Berilgan tenglama kanonik ko‘rinishga keltirilsin; tipi aniqlansin; qanday geometrik obrazni ifodalashi aniqlansin; eski va yangi koordinatalar sistemasida geometrik obrazi tasvirlansin: $7 x^2+6 x y-y^2+28 x+12 y+28=0$;

\textbf{676.5)} Berilgan tenglama kanonik ko‘rinishga keltirilsin; tipi aniqlansin; qanday geometrik obrazni ifodalashi aniqlansin; eski va yangi koordinatalar sistemasida geometrik obrazi tasvirlansin: $19 x^2+6 x y+11 y^2+38 x+6 y+29=0$;

\textbf{676.6)} Berilgan tenglama kanonik ko‘rinishga keltirilsin; tipi aniqlansin; qanday geometrik obrazni ifodalashi aniqlansin; eski va yangi koordinatalar sistemasida geometrik obrazi tasvirlansin: $5 x^2-2 x y+5 y^2-4 x+20 y+20=0$.

\textbf{677.1)} Berilgan tenglama kanonik ko‘rinishga keltirilsin; tipi aniqlansin; qanday geometrik obrazni ifodalashi aniqlansin; eski va yangi koordinatalar sistemasida geometrik obrazi tasvirlansin: $14 x^2+24 x y+21 y^2-4 x+18 y-139=0$;

\textbf{677.2)} Berilgan tenglama kanonik ko‘rinishga keltirilsin; tipi aniqlansin; qanday geometrik obrazni ifodalashi aniqlansin; eski va yangi koordinatalar sistemasida geometrik obrazi tasvirlansin: $11 x^2-20 x y-4 y^2-20 x-8 y+1=0$;

\textbf{677.3)} Berilgan tenglama kanonik ko‘rinishga keltirilsin; tipi aniqlansin; qanday geometrik obrazni ifodalashi aniqlansin; eski va yangi koordinatalar sistemasida geometrik obrazi tasvirlansin: $7 x^2+60 x y+32 y^2-14 x-60 y+7=0$;

\textbf{677.4)} Berilgan tenglama kanonik ko‘rinishga keltirilsin; tipi aniqlansin; qanday geometrik obrazni ifodalashi aniqlansin; eski va yangi koordinatalar sistemasida geometrik obrazi tasvirlansin: $50 x^2-8 x y+35 y^2+100 x-8 y+67=0$;

\textbf{677.5)} Berilgan tenglama kanonik ko‘rinishga keltirilsin; tipi aniqlansin; qanday geometrik obrazni ifodalashi aniqlansin; eski va yangi koordinatalar sistemasida geometrik obrazi tasvirlansin: $41 x^2+24 x y+34 y^2+34 x-112 y+129=0$;

\textbf{677.6)} Berilgan tenglama kanonik ko‘rinishga keltirilsin; tipi aniqlansin; qanday geometrik obrazni ifodalashi aniqlansin; eski va yangi koordinatalar sistemasida geometrik obrazi tasvirlansin: $29 x^2-24 x y+36 y^2+82 x-96 y-91=0$;

\textbf{677.7)} Berilgan tenglama kanonik ko‘rinishga keltirilsin; tipi aniqlansin; qanday geometrik obrazni ifodalashi aniqlansin; eski va yangi koordinatalar sistemasida geometrik obrazi tasvirlansin: $4 x^2+24 x y+11 y^2+64 x+42 y+51=0$;

\textbf{677.8)} Berilgan tenglama kanonik ko‘rinishga keltirilsin; tipi aniqlansin; qanday geometrik obrazni ifodalashi aniqlansin; eski va yangi koordinatalar sistemasida geometrik obrazi tasvirlansin: $41 x^2+24 x y+9 y^2+24 x+18 y-36=0$.

\textbf{683.} Har qanday elliptik tenglama uchun $a_{11}$ va $a_{22}$ koeffitsiyentlarning hech biri nolga aylana olmasligini va ular bir xil ishorali sonlar ekanligini isbotlang.

\textbf{684.} Elliptik tipli ($\delta>0$) tenglama $a_{11}$ va $\Delta$ ning turli ishorali sonlar bo‘lgandagina ellipsni aniqlashi isbotlansin.

\textbf{685.} Elliptik tipli ($\delta>0$) tenglama $a_{11}$ va $\Delta$ bir xil ishorali son bo‘lgandagina mavhum ellips tenglamasi bo‘lishini isbotlang.

\textbf{686.} Elliptik tipli ($\delta>0$) tenglama $\Delta=0$ bo‘lgandagina ikkita bir-birini kesuvchi mavhum to‘g‘ri chiziq bo‘lishini isbotlang.


\section{Parabolik tenglamani sodda ko‘rinishga keltirish}



\textbf{691.} Har qanday parabolik tenglama uchun $a_{11}$ va $a_{22}$ koeffitsiyentlar turli ishorali sonlar bo‘la olmasligini va ular bir vaqtda nolga aylana olmasligini isbotlang.

\textbf{692.} Har qanday parabolik tenglama $ (\alpha x+\beta y) ^2+2a_{13}x+2a_{23}y+a_{33}=0$ ko‘rinishda yozilishi mumkinligini isbotlang. Shuningdek, elliptik va giperbolik tenglamalarni bunday ko‘rinishda yozib bo‘lmasligini isbotlang.

\textbf{694.} Agar ikkinchi darajali tenglama parabolik bo‘lib, $ (\alpha x+\beta y) ^2+2a_{13}x+2a_{23}y+a_{33}=0$ ko‘rinishda yozilgan bo‘lsa, uning chap tomonidagi diskriminant $\Delta=- (a_{13} \beta-a_{23} \alpha) ^2$ formula bilan aniqlanishini isbotlang.

\textbf{696.} Parabolik tenglama $\Delta \neq 0$ bo‘lganda va faqat shundagina parabolani aniqlashi isbotlansin. Bu holda parabolaning parametri $p=\sqrt{\frac{-\Delta}{ (a_{11}+a_{33}) ^3}}$ formula bilan aniqlanishini isbotlang.

\textbf{698.} Ikkinchi darajali tenglama faqat va faqat $\Delta=0$ bo‘lgandagina aynigan chiziq tenglamasi bo‘lishini isbotlang.



\section{Ikkinchi tartibli sirtlar}



\textbf{1160.} $x+m z-1=0$ tekislik ushbu $x^2+y^2-z^2=-1$ ikki pallali giperboloidni $m$ ning qanday qiymatlarida a) ellips bo‘yicha, b) giperbola bo‘yicha kesishi aniqlansin.

\textbf{1161.} $m$ ning qanday qiymatlarida $x+m y-2=0$ tekislik $\frac{x^2}{2}+\frac{z^2}{3}=y$ elliptik paraboloidni a) ellips bo‘yicha, b) parabola bo‘yicha kesib o‘tishini aniqlang.

\textbf{1162.} $\frac{x^2}{9}+\frac{z^2}{4}=2 y$ elliptik paraboloid $2 x-2 y-z-10=0$ tekislik bilan bitta umumiy nuqtaga ega ekanligini isbotlang va uning koordinatalarini toping.

\textbf{1163.} Ikki pallali $\frac{x^2}{3}+\frac{y^2}{4}-\frac{z^2}{25}=-1$ giperboloid $5 x+2 z+5=0$ tekislik bilan bitta umumiy nuqtaga ega ekanligini isbotlang va uning koordinatalarini toping.

\textbf{1164.} $\frac{x^2}{81}+\frac{y^2}{36}+\frac{z^2}{9}=1$ ellipsoid $4 x-3 y+12 z-54=0$ tekislik bilan bitta umumiy nuqtaga ega ekanligini isbotlang va uning koordinatalarini toping.

\textbf{1165.} $m$ ning qanday qiymatida $x-2 y-2 z+m=0$ tekislik $\frac{x^2}{144}+\frac{y^2}{36}+\frac{z^2}{9}=1$ ellipsoidga urinishi aniqlansin.