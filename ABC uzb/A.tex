\section{Ellipsning kanonik tenglamasi}


\textbf{444.1)} Fokuslari abssissa o‘qida yotgan va koordinatalar boshiga nisbatan simmetrik, yarim o‘qlari 5 va 2 bo‘lgan ellipsning tenglamasi tuzilsin yarim o‘qlari 5 va 2;

\textbf{444.2)} Fokuslari abssissa o‘qida yotgan va koordinatalar boshiga nisbatan simmetrik bo‘lgan ellipsning tenglamasi tuzilsin, bunda uning katta o‘qi 10 ga, fokuslari orasidagi masofa esa $c = 8$ ga teng;

\textbf{444.3)} Fokuslari abssissa o‘qida yotgan va koordinatalar boshiga nisbatan simmetrik bo‘lgan ellipsning tenglamasi tuzilsin, bunda uning kichik o‘qi 24 ga, fokuslari orasidagi masofa esa $c = 10$ ga teng;

\textbf{444.4)} Fokuslari abssissa o‘qida yotgan va koordinatalar boshiga nisbatan simmetrik bo‘lgan ellipsning tenglamasi tuzilsin, bunda: fokuslari orasidagi masofa $2 c=6$ va ekssentrisiteti $\varepsilon=\frac{3}{5}$;

\textbf{444.5)} Fokuslari abssissa o‘qida yotgan va koordinatalar boshiga nisbatan simmetrik bo‘lgan ellipsning tenglamasi tuzilsin, bunda: katta o'qi 20, ekssentrisiteti $\varepsilon=\frac{3}{5}$;

\textbf{444.6)} Fokuslari abssissa o‘qida yotgan va koordinatalar boshiga nisbatan simmetrik bo‘lgan ellipsning tenglamasi tuzilsin, bunda: kichik o'qi 10 , ekssentrisiteti $\varepsilon=\frac{12}{13}$;

\textbf{444.7)} Fokuslari abssissa o‘qida yotgan va koordinatalar boshiga nisbatan simmetrik bo‘lgan ellipsning tenglamasi tuzilsin, bunda: direktrisalari orasidagi masofa 5 va fokuslari orasidagi masofa $2 c=4$;

\textbf{444.8)} Fokuslari abssissa o‘qida yotgan va koordinatalar boshiga nisbatan simmetrik bo‘lgan ellipsning tenglamasi tuzilsin, bunda: katta o'qi 8, direktrisalari orasidagi masofa 16 ;

\textbf{444.9)} Fokuslari abssissa o‘qida yotgan va koordinatalar boshiga nisbatan simmetrik bo‘lgan ellipsning tenglamasi tuzilsin, bunda: kichik o'qi 6, direktrisalari orasidagi masofa 13 ;

\textbf{444.10)} Fokuslari abssissa o‘qida yotgan va koordinatalar boshiga nisbatan simmetrik bo‘lgan ellipsning tenglamasi tuzilsin, bunda: direktrisalari orasidagi masofa 32 va $\varepsilon=\frac{1}{2}$.

\textbf{445.1)} Fokuslari ordinata o‘qida yotgan va koordinatalar boshiga nisbatan simmetrik bo‘lgan ellipsning tenglamasi tuzilsin, bunda: yarim o'qlari 7 va 2 ;

\textbf{445.2)} Fokuslari ordinata o‘qida yotgan va koordinatalar boshiga nisbatan simmetrik bo‘lgan ellipsning tenglamasi tuzilsin, bunda: katta yarim o'qi 10 , fokuslari orasidagi masofa $2 c=8$;

\textbf{445.3)} Fokuslari ordinata o‘qida yotgan va koordinatalar boshiga nisbatan simmetrik bo‘lgan ellipsning tenglamasi tuzilsin, bunda: fokuslari orasidagi masofa $2 c=24$ ekssentrisiteti $\varepsilon=\frac{12}{13}$;

\textbf{445.4)} Fokuslari ordinata o‘qida yotgan va koordinatalar boshiga nisbatan simmetrik bo‘lgan ellipsning tenglamasi tuzilsin, bunda: kichik o'qi 16 , а ekssentrisiteti $\varepsilon=\frac{3}{5}$;

\textbf{445.5)} Fokuslari ordinata o‘qida yotgan va koordinatalar boshiga nisbatan simmetrik bo‘lgan ellipsning tenglamasi tuzilsin, bunda: fokuslari orasidagi masofa $2 c=6$ direktrisalari orasidagi masofa $16 \frac{2}{3}$;

\textbf{445.6)} Fokuslari ordinata o‘qida yotgan va koordinatalar boshiga nisbatan simmetrik bo‘lgan ellipsning tenglamasi tuzilsin, bunda: direktrisalari orasidagi masofa $10 \frac{2}{3}$ va ekssentrisiteti $\varepsilon=\frac{3}{4}$.

\textbf{450.} Ikki uchi $9 x^2+5 y^2=1$ ellipsning uchlarida, qolgan ikki uchi uning kichik o'qining uchlarida joylashgan to'rtburchakning yuzini hisoblang.

\textbf{462.} $\frac{x^2}{100}+\frac{y^2}{36}=1$ ellipsida joylashgan va o'ng fokusigacha masofasi 14 ga teng nuqtani toping.

\textbf{463.} $\frac{x^2}{16}+\frac{y^2}{7}=1$, ellipsida joylashgan va chap fokusigacha masofasi 2,5 ga teng nuqtani toping.

\textbf{465.1)} Fokuslari abssissa o‘qida yotgan va koordinatalar boshiga nisbatan simmetrik bo‘lgan ellipsning tenglamasi tuzilsin, bunda: $M_1(-2 \sqrt{5} ; 2)$ nuqtasi ellipsga tegishli va kichik yarim o'qi $b=3$;

\textbf{465.2)} Fokuslari abssissa o‘qida yotgan va koordinatalar boshiga nisbatan simmetrik bo‘lgan ellipsning tenglamasi tuzilsin, bunda: $M_1(2 ;-2)$ nuqtasi ellipsga tegishli va katta yarim o'qi $a=4$;

\textbf{465.3)} Fokuslari abssissa o‘qida yotgan va koordinatalar boshiga nisbatan simmetrik bo‘lgan ellipsning tenglamasi tuzilsin, bunda: $M_1(4 ;-\sqrt{3})$ va $M_2(2 \sqrt{2} ; 3)$ nuqtalari ellipsga tegishli;

\textbf{465.4)} Fokuslari abssissa o‘qida yotgan va koordinatalar boshiga nisbatan simmetrik bo‘lgan ellipsning tenglamasi tuzilsin, bunda: $M_1(\sqrt{15} ;-1)$ nuqtasi ellipsga tegishli va fokuslari orasidagi masofa $2 c=8$;

\textbf{465.5)} Fokuslari abssissa o‘qida yotgan va koordinatalar boshiga nisbatan simmetrik bo‘lgan ellipsning tenglamasi tuzilsin, bunda: $M_1\left(2 ;-\frac{5}{3}\right)$ nuqtasi ellipsga tegishli va ekssentrisiteti $\varepsilon=\frac{2}{3}$;

\textbf{465.6)} Fokuslari abssissa o‘qida yotgan va koordinatalar boshiga nisbatan simmetrik bo‘lgan ellipsning tenglamasi tuzilsin, bunda: $M_1(8 ; 12)$ ellipsga tegishli va bu nuqtadan chap fokusigacha masofa $r_1=20$ ga teng;

\textbf{465.7)} Fokuslari abssissa o‘qida yotgan va koordinatalar boshiga nisbatan simmetrik bo‘lgan ellipsning tenglamasi tuzilsin, bunda: $M_1 (-\sqrt{5}; 2) $ nuqtasi ellipsga tegishli va uning direktrisalari orasidagi masofa 10 ga teng.

\textbf{470.} $C (-3; 2) $ ikkala koordinata o‘qiga urinuvchi ellipsning markazi. Bu ellipsning simmetriya o‘qlari koordinata o‘qlariga parallel ekanligini bilgan holda uning tenglamasi tuzilsin.

\textbf{474.} Ekssentrisiteti $\varepsilon=\frac{2}{3}$, fokusi $F(2 ; 1)$ va shu fokus tarafdagi direktrisasi $x-5=0$ bo'lgan ellipsning tenglamasini tuzing.

\textbf{475.} Ekssentrisiteti $\varepsilon=\frac{1}{2}$, fokusi $F(-4 ; 1)$ va shu fokus tarafdagi direktrisasi $y+3=0$ bo'lgan ellipsning tenglamasini tuzing.

\textbf{476.} $A (-3;-5)$ nuqta fokusi $F (-1;-4)$ bo‘lgan ellipsda yotadi va unga mos direktrisa $x-2=0$ tenglama bilan berilgan. Shu ellipsning tenglamasi tuzilsin.



\section{Giperbolaning kanonik tenglamasi}



\textbf{515.1)} Fokuslari abssissa o‘qida joylashgan, koordinatalar boshiga nisbatan simmetrik bo'lgan giperbolaning tenglamasi tuzilsin, bunda: uning o'qlari $2 a=10$ va $2 b=8$;

\textbf{515.2)} Fokuslari abssissa o‘qida joylashgan, koordinatalar boshiga nisbatan simmetrik bo'lgan giperbolaning tenglamasi tuzilsin, bunda: fokuslari orasidagi masofa $2 c=10$ va kichik o'qi $2 b=8$;

\textbf{515.3)} Fokuslari abssissa o‘qida joylashgan, koordinatalar boshiga nisbatan simmetrik bo'lgan giperbolaning tenglamasi tuzilsin, bunda: fokuslari orasidagi masofa $2 c=6$ va ekssentrisiteti $\varepsilon=\frac{3}{2}$;

\textbf{515.4)} Fokuslari abssissa o‘qida joylashgan, koordinatalar boshiga nisbatan simmetrik bo'lgan giperbolaning tenglamasi tuzilsin, bunda: $2 a=16$ va ekssentrisiteti $\varepsilon=\frac{5}{4}$;

\textbf{515.5)} Fokuslari abssissa o‘qida joylashgan, koordinatalar boshiga nisbatan simmetrik bo'lgan giperbolaning tenglamasi tuzilsin, bunda: asimptota tenglamasi $y= \pm \frac{4}{3} x$ va fokuslari orasidagi masofa $2 c=20$;

\textbf{515.6)} Fokuslari abssissa o‘qida joylashgan, koordinatalar boshiga nisbatan simmetrik bo'lgan giperbolaning tenglamasi tuzilsin, bunda: direktrisalari orasidagi masofa $22 \frac{2}{13}$ va fokuslari orasidagi masofa $2 c=26$;

\textbf{515.7)} Fokuslari abssissa o‘qida joylashgan, koordinatalar boshiga nisbatan simmetrik bo'lgan giperbolaning tenglamasi tuzilsin, bunda: direktrisalari orasidagi masofa $\frac{32}{5}$ va ось $2 b=6$;

\textbf{515.8)} Fokuslari abssissa o‘qida joylashgan, koordinatalar boshiga nisbatan simmetrik bo'lgan giperbolaning tenglamasi tuzilsin, bunda: direktrisalari orasidagi masofa $\frac{8}{3}$ va ekssentrisiteti $\varepsilon=\frac{3}{2}$;

\textbf{515.9)} Fokuslari abssissa o‘qida joylashgan, koordinatalar boshiga nisbatan simmetrik bo'lgan giperbolaning tenglamasi tuzilsin, bunda: asimptota tenglamasi $y= \pm \frac{3}{4} x$ va direktrisalari orasidagi masofa $12 \frac{4}{5}$.

\textbf{516.1)} Fokuslari ordinata o‘qida, koordinatalar boshiga nisbatan simmetrik joylashgan giperbolaning tenglamasi tuzilsin, bunda: uning yarim o'qlari $a=6, b=18$;

\textbf{516.2)} Fokuslari ordinata o‘qida, koordinatalar boshiga nisbatan simmetrik joylashgan giperbolaning tenglamasi tuzilsin, bunda: fokuslari orasidagi masofa $2 c=10$ va ekssentrisiteti $\varepsilon=\frac{5}{3}$;

\textbf{516.3)} Fokuslari ordinata o‘qida, koordinatalar boshiga nisbatan simmetrik joylashgan giperbolaning tenglamasi tuzilsin, bunda: asimptota tenglamasi $y= \pm \frac{12}{5} x$ va uchlari orasidagi masofa 48;

\textbf{516.4)} Fokuslari ordinata o‘qida, koordinatalar boshiga nisbatan simmetrik joylashgan giperbolaning tenglamasi tuzilsin, bunda: direktrisalari orasidagi masofa $7 \frac{1}{7}$ va ekssentrisiteti $\varepsilon=\frac{7}{5}$;

\textbf{516.5)} Fokuslari ordinata o‘qida, koordinatalar boshiga nisbatan simmetrik joylashgan giperbolaning tenglamasi tuzilsin, bunda: asimptota tenglamasi $y= \pm \frac{4}{3} x$ va direktrisalari orasidagi masofa $6 \frac{2}{5}$.

\textbf{518.} $16 x^2-9 y^2=144$ giperbola berilgan. Toping: 1) yarim o'qlarini; 2) fokuslarini; 3) ekssentrisitetini; 4) asimptota tenglamasi; 5) direktrisalari tenglamalarini.

\textbf{520.} $\frac{x^2}{4}-\frac{y^2}{9}=1$ giperbolaning asimptotalaridan va $9 x+$ $+2 y-24=0$ to‘g‘ri chiziqdan hosil bo‘lgan uchburchak yuzini hisoblang.

\textbf{521.1)} Berilgan tenglama bilan qaysi chiziq aniqlanishini toping: $y=+\frac{2}{3} \sqrt{x^2-9}$

\textbf{521.2)} Berilgan tenglama bilan qaysi chiziq aniqlanishini toping: $y=-3 \sqrt{x^2+1}$;

\textbf{521.3)} Berilgan tenglama bilan qaysi chiziq aniqlanishini toping: $x=-\frac{4}{3} \sqrt{y^2+9} ;$

\textbf{521.4)} Berilgan tenglama bilan qaysi chiziq aniqlanishini toping: $y=+\frac{2}{5} \sqrt{x^2+25}$

\textbf{524.} Giperbolaning ekssentrisiteti $\varepsilon=2$ ga teng, $M$ nuqtasining bazi bir fokal radiusi 16 ga teng. $M$ nuqtasidan mos direktrisagacha masofani toping.

\textbf{525.} Giperbolaning ekssentrisiteti $\varepsilon=3$, $M$ nuqtasining bazi bir fokal radiusi 4 ga teng. $M$ nuqtadan mos direktrisagacha masofanii toping.

\textbf{532.1)} Fokuslari abssissa o‘qida, koordinatalar boshiga nisbatan simmetrik joylashgan giperbolaning tenglamasi tuzilsin, bunda: $M_1(6 ;-1)$ va $M_2(-8 ; 2 \sqrt{2})$ nuqtalar giperbolaga tegishli;

\textbf{532.2)} Fokuslari abssissa o‘qida, koordinatalar boshiga nisbatan simmetrik joylashgan giperbolaning tenglamasi tuzilsin, bunda: $M_1(-5 ; 3)$ nuqta giperbolaga tegishli va ekssentrisiteti $\varepsilon=\sqrt{2}$;

\textbf{532.3)} Fokuslari abssissa o‘qida, koordinatalar boshiga nisbatan simmetrik joylashgan giperbolaning tenglamasi tuzilsin, bunda: $M_1\left(\frac{9}{2} ;-1\right)$ nuqtasi giperbolaga tegishli va asimptota tenglamalari $y= \pm \frac{2}{3} x$;

\textbf{532.4)} Fokuslari abssissa o‘qida, koordinatalar boshiga nisbatan simmetrik joylashgan giperbolaning tenglamasi tuzilsin, bunda: $M_1\left(-3 ; \frac{5}{2}\right)$ nuqtasi giperbolaga tegishli va direktrisalarining tenglamasi $x= \pm \frac{4}{3}$;

\textbf{532.5)} Fokuslari abssissa o‘qida, koordinatalar boshiga nisbatan simmetrik joylashgan giperbolaning tenglamasi tuzilsin, bunda: asimptota tenglamalari $y= \pm \frac{3}{4} x$ va direktrisalarining tenglamalari $x= \pm \frac{16}{5}$.

\textbf{544.} Ekssentrisiteti $\varepsilon=\frac{5}{4}$, bir fokusi $F(5 ; 0)$ va unga mos direktrisasining tenglamasi $5 x-16=0$ bo'lgan giperbolaning tenglamasini tuzing.

\textbf{545.} Ekssentrisiteti $\varepsilon=\frac{13}{12}$, bir fokusi $F(0 ; 13)$ va unga mos direktrisasining tenglamasi $13 y-144=0$ bo'lgan giperbolaning tenglamasini tuzing.



\section{Parabolaning kanonik tenglamasi}



\textbf{583.1)} Uchi koordinatalar boshida bo‘lgan parabolaning tenglamasini tuzing, bunda: parabola o'ng yarim tekislikda va $Ox$ o'qiga simmetrik joylashgan, va parametri $p=3$;

\textbf{583.2)} Uchi koordinatalar boshida bo‘lgan parabolaning tenglamasini tuzing, bunda: parabola chap yarim tekislikda va $Ox$ o'qiga simmetrik joylashgan, va parametri $p=0,5$;

\textbf{583.3)} Uchi koordinatalar boshida bo‘lgan parabolaning tenglamasini tuzing, bunda: parabola yuqori yarim tekislikda va $Oy$ o'qiga simmetrik joylashgan, va parametri $p=\frac{1}{4}$;

\textbf{583.4)} Uchi koordinatalar boshida bo‘lgan parabolaning tenglamasini tuzing, bunda: parabola pastgi yarim tekislikda va $Oy$ o'qiga simmetrik joylashgan, va parametri $p=3$.

\textbf{585.1)} Uchi koordinatalar boshida bo‘lgan parabolaning tenglamasini tuzing, bunda: parabola $Ox$ o'qiga simmetrik joylashgan va $A(9 ; 6)$ nuqtasidan o'tadi;

\textbf{585.2)} Uchi koordinatalar boshida bo‘lgan parabolaning tenglamasini tuzing, bunda: parabola $Ox$ o'qiga simmetrik joylashgan va $B(-1 ; 3)$ nuqtasidan o'tadi;

\textbf{585.3)} Uchi koordinatalar boshida bo‘lgan parabolaning tenglamasini tuzing, bunda: parabola $Oy$ o'qiga simmetrik joylashgan va $C(1 ; 1)$ nuqtasidan o’tadi.

\textbf{585.4)} Uchi koordinatalar boshida bo‘lgan parabolaning tenglamasini tuzing, bunda: parabola $Oy$ o'qiga simmetrik joylashgan va $D(4 ;-8)$ nuqtasidan o’tadi.

\textbf{586.} Po‘lat tros ikki uchidan osilgan; mahkamlash nuqtalari bir xil balandlikda joylashgan; ular orasidagi masofa 20 m ga teng. Uning mahkamlash nuqtasidan 2 m masofadagi egilish kattaligi, gorizontal bo‘yicha hisoblaganda, 14,4 sm ga teng. Trosni taxminan parabola yoyi shaklida deb hisoblab, mahkamlash nuqtalari orasidagi bu trosning egilish kattaligini aniqlang.

\textbf{589.} $y^2=24 x$ parabolaning $F$ fokusini va direktrisasinining tenglamasini toping.

\textbf{590.} $M$ nuqtasi $y^2=20 x$ parabolaga tegishli, agar uning abssissasi 7 ga teng bo'lsa fokal radiuslarini toping.

\textbf{605.} $x+y-3=0$ to'g'ri chizig'i va $x^2=4 y$ parabolasining kesishish nuqtasini toping.

\textbf{617.} $y^2=36 x$ parabolaning $A(2 ; 9)$ nuqtasidagi urinmasining tenglama tuzing.

\textbf{621.} $\frac{x^2}{100}+\frac{y^2}{225}=1$ ellips va $y^2=24 x$ parabolaning kesishish nuqtalarini aniqlang.

\textbf{622.} $\frac{x^2}{20}-\frac{y^2}{5}=-1$ giperbola va $y^2=3 x$ parabolaning kesishish nuqtalarini aniqlang.


\section{Ikkinchi tartibli chiziq markazi}


\textbf{665.1)} Quyidagi chiziqlardan qaysi biri markaziy (ya’ni yagona markazga ega), qaysi biri markazga ega emas, qaysi biri cheksiz ko‘p markazga ega ekanligini aniqlang: $3 x^2-4 x y-2 y^2+3 x-12 y-7=0$;

\textbf{665.2)} Quyidagi chiziqlardan qaysi biri markaziy (ya’ni yagona markazga ega), qaysi biri markazga ega emas, qaysi biri cheksiz ko‘p markazga ega ekanligini aniqlang: $4 x^2+5 x y+3 y^2-x+9 y-12=0$;

\textbf{665.3)} Quyidagi chiziqlardan qaysi biri markaziy (ya’ni yagona markazga ega), qaysi biri markazga ega emas, qaysi biri cheksiz ko‘p markazga ega ekanligini aniqlang: $4 x^2-4 x y+y^2-6 x+8 y+13=0$;

\textbf{665.4)} Quyidagi chiziqlardan qaysi biri markaziy (ya’ni yagona markazga ega), qaysi biri markazga ega emas, qaysi biri cheksiz ko‘p markazga ega ekanligini aniqlang: $4 x^2-4 x y+y^2-12 x+6 y-11=0$;

\textbf{665.5)} Quyidagi chiziqlardan qaysi biri markaziy (ya’ni yagona markazga ega), qaysi biri markazga ega emas, qaysi biri cheksiz ko‘p markazga ega ekanligini aniqlang: $x^2-2 x y+4 y^2+5 x-7 y+12=0$;

\textbf{665.6)} Quyidagi chiziqlardan qaysi biri markaziy (ya’ni yagona markazga ega), qaysi biri markazga ega emas, qaysi biri cheksiz ko‘p markazga ega ekanligini aniqlang: $x^2-2 x y+y^2-6 x+6 y-3=0$;

\textbf{665.7)} Quyidagi chiziqlardan qaysi biri markaziy (ya’ni yagona markazga ega), qaysi biri markazga ega emas, qaysi biri cheksiz ko‘p markazga ega ekanligini aniqlang: $4 x^2-20 x y+25 y^2-14 x+2 y-15=0$;

\textbf{665.8)} Quyidagi chiziqlardan qaysi biri markaziy (ya’ni yagona markazga ega), qaysi biri markazga ega emas, qaysi biri cheksiz ko‘p markazga ega ekanligini aniqlang: $4 x^2-6 x y-9 y^2+3 x-7 y+12=0$.

\textbf{667.1)} Quyidagi chiziqlarning har biri cheksiz ko‘p markazga ega ekanligi ko'rsatilsin; ularning har biri uchun markazlarning geometrik o‘rni tenglamasi tuzilsin: $x^2-6 x y+9 y^2-12 x+36 y+20=0$;

\textbf{667.2)} Quyidagi chiziqlarning har biri cheksiz ko‘p markazga ega ekanligi ko'rsatilsin; ularning har biri uchun markazlarning geometrik o‘rni tenglamasi tuzilsin: $4 x^2+4 x y+y^2-8 x-4 y-21=0$;

\textbf{667.3)} Quyidagi chiziqlarning har biri cheksiz ko‘p markazga ega ekanligi ko'rsatilsin; ularning har biri uchun markazlarning geometrik o‘rni tenglamasi tuzilsin: $25 x^2-10 x y+y^2+40 x-8 y+7=0$.



\section{Ikkinchi tartibli markaziy chiziq tenglamasini sodda ko‘rinishga keltirish}



\textbf{675.1)} Diskriminantini hisoblash orqali quyidagi tenglamalarning har birining tipini aniqlang: $2 x^2+10 x y+12 y^2-7 x+18 y-15=0$;

\textbf{675.2)} Diskriminantini hisoblash orqali quyidagi tenglamalarning har birining tipini aniqlang: $3 x^2-8 x y+7 y^2+8 x-15 y+20=0$;

\textbf{675.3)} Diskriminantini hisoblash orqali quyidagi tenglamalarning har birining tipini aniqlang: $25 x^2-20 x y+4 y^2-12 x+20 y-17=0$;

\textbf{675.4)} Diskriminantini hisoblash orqali quyidagi tenglamalarning har birining tipini aniqlang: $5 x^2+14 x y+11 y^2+12 x-7 y+19=0$;

\textbf{675.5)} Diskriminantini hisoblash orqali quyidagi tenglamalarning har birining tipini aniqlang: $x^2-4 x y+4 y^2+7 x-12=0$;

\textbf{675.6)} Diskriminantini hisoblash orqali quyidagi tenglamalarning har birining tipini aniqlang: $3 x^2-2 x y-3 y^2+12 y-15=0$.

\textbf{678.1)} Koordinatalar sistemasini almashtirmasdan quyidagi tenglamalarning har biri ellipsni aniqlashini ko'rsating va uning yarim o‘qlarini toping: $41 x^2+24 x y+9 y^2+24 x+18 y-36=0$;

\textbf{678.2)} Koordinatalar sistemasini almashtirmasdan quyidagi tenglamalarning har biri ellipsni aniqlashini ko'rsating va uning yarim o‘qlarini toping: $8 x^2+4 x y+5 y^2+16 x+4 y-28=0$;

\textbf{678.3)} Koordinatalar sistemasini almashtirmasdan quyidagi tenglamalarning har biri ellipsni aniqlashini ko'rsating va uning yarim o‘qlarini toping: $13 x^2+18 x y+37 y^2-26 x-18 y+3=0$;

\textbf{678.4)} Koordinatalar sistemasini almashtirmasdan quyidagi tenglamalarning har biri ellipsni aniqlashini ko'rsating va uning yarim o‘qlarini toping: $13 x^2+10 x y+13 y^2+46 x+62 y+13=0$.

\textbf{679.1)} Koordinatalar sistemasini almashtirmasdan quyidagi tenglamalarning har biri yagona nuqtani (mavhum ellipsni) aniqlashini ko'rsating va uning koordinatalarini toping: $5 x^2-6 x y+2 y^2-2 x+2=0$;

\textbf{679.2)} Koordinatalar sistemasini almashtirmasdan quyidagi tenglamalarning har biri yagona nuqtani (mavhum ellipsni) aniqlashini ko'rsating va uning koordinatalarini toping: $x^2+2 x y+2 y^2+6 y+9=0$;

\textbf{679.3)} Koordinatalar sistemasini almashtirmasdan quyidagi tenglamalarning har biri yagona nuqtani (mavhum ellipsni) aniqlashini ko'rsating va uning koordinatalarini toping: $5 x^2+4 x y+y^2-6 x-2 y+2=0$;

\textbf{679.4)} Koordinatalar sistemasini almashtirmasdan quyidagi tenglamalarning har biri yagona nuqtani (mavhum ellipsni) aniqlashini ko'rsating va uning koordinatalarini toping: $x^2-6 x y+10 y^2+10 x-32 y+26=0$.

\textbf{680.1)} Koordinatalar sistemasini almashtirmasdan quyidagi tenglamalarning har biri giperbolani aniqlashini ko'rsating va uning yarim o‘qlarini toping: $4 x^2+24 x y+11 y^2+64 x+42 y+51=0$;

\textbf{680.2)} Koordinatalar sistemasini almashtirmasdan quyidagi tenglamalarning har biri giperbolani aniqlashini ko'rsating va uning yarim o‘qlarini toping: $12 x^2+26 x y+12 y^2-52 x-48 y+73=0$

\textbf{680.3)} Koordinatalar sistemasini almashtirmasdan quyidagi tenglamalarning har biri giperbolani aniqlashini ko'rsating va uning yarim o‘qlarini toping: $3 x^2+4 x y-12 x+16=0$;

\textbf{680.4)} Koordinatalar sistemasini almashtirmasdan quyidagi tenglamalarning har biri giperbolani aniqlashini ko'rsating va uning yarim o‘qlarini toping: $x^2-6 x y-7 y^2+10 x-30 y+23=0$.

\textbf{681.1)} Koordinatalar sistemasini almashtirmasdan quyidagi tenglamalarning har biri kesishuvchi to'g'ri chiziqlarni (mavhun gierbolani) aniqlashini ko'rsating va tenglamalarini toping: $3 x^2+4 x y+y^2-2 x-1=0$;

\textbf{681.2)} Koordinatalar sistemasini almashtirmasdan quyidagi tenglamalarning har biri kesishuvchi to'g'ri chiziqlarni (mavhun gierbolani) aniqlashini ko'rsating va tenglamalarini toping: $x^2-6 x y+8 y^2-4 y-4=0$;

\textbf{681.3)} Koordinatalar sistemasini almashtirmasdan quyidagi tenglamalarning har biri kesishuvchi to'g'ri chiziqlarni (mavhun gierbolani) aniqlashini ko'rsating va tenglamalarini toping: $x^2-4 x y+3 y^2=0$;

\textbf{681.4)} Koordinatalar sistemasini almashtirmasdan quyidagi tenglamalarning har biri kesishuvchi to'g'ri chiziqlarni (mavhun gierbolani) aniqlashini ko'rsating va tenglamalarini toping: $x^2+4 x y+3 y^2-6 x-12 y+9=0$.

\textbf{682.1)} Koordinatalar sistemasini almashtirmasdan, quyidagi tenglamalar bilan qanday geometrik chakl aniqlanishini toping: $8 x^2-12 x y+17 y^2+16 x-12 y+3=0$;

\textbf{682.2)} Koordinatalar sistemasini almashtirmasdan, quyidagi tenglamalar bilan qanday geometrik chakl aniqlanishini toping: $17 x^2-18 x y-7 y^2+34 x-18 y+7=0$;

\textbf{682.3)} Koordinatalar sistemasini almashtirmasdan, quyidagi tenglamalar bilan qanday geometrik chakl aniqlanishini toping: $2 x^2+3 x y-2 y^2+5 x+10 y=0$;

\textbf{682.4)} Koordinatalar sistemasini almashtirmasdan, quyidagi tenglamalar bilan qanday geometrik chakl aniqlanishini toping: $6 x^2-6 x y+9 y^2-4 x+18 y+14=0$;



\section{Parabolik tenglamani sodda ko‘rinishga keltirish}



\textbf{697.1)} Koordinatalar sistemasini almashtirmasdan, quyidagi tenglamalarning har biri parabolani aniqlashi ko'rsating va parametrini toping: $9 x^2+24 x y+16 y^2-120 x+90 y=0$;

\textbf{697.2)} Koordinatalar sistemasini almashtirmasdan, quyidagi tenglamalarning har biri parabolani aniqlashi ko'rsating va parametrini toping: $9 x^2-24 x y+16 y^2-54 x-178 y+181=0$;

\textbf{697.3)} Koordinatalar sistemasini almashtirmasdan, quyidagi tenglamalarning har biri parabolani aniqlashi ko'rsating va parametrini toping: $x^2-2 x y+y^2+6 x-14 y+29=0$;

\textbf{697.4)} Koordinatalar sistemasini almashtirmasdan, quyidagi tenglamalarning har biri parabolani aniqlashi ko'rsating va parametrini toping: $9 x^2-6 x y+y^2-50 x+50 y-275=0$.



\section{Ikkinchi tartibli sirtlar}



\textbf{1153.} $x-2=0$ tekislik $\frac{x^2}{16}+\frac{y^2}{12}+\frac{z^2}{4}=1$ ellipsoidni ellips bo‘yicha kesib o‘tishini ko'rsating; uning yarim o‘qlari va uchlarini toping.

\textbf{1154.} $z+1=0$ tekislik bir pallali $\frac{x^2}{32}-\frac{y^2}{18}+\frac{z^2}{2}=1$ giperboloidni giperbola bo‘yicha kesib o‘tishini ko'rsating; uning yarim o‘qlari va uchlarini toping.

\textbf{1155.} $y+6=0$ tekislik $\frac{x^2}{5}-\frac{y^2}{4}=6 z$ giperbolik paraboloidni parabola bo‘yicha kesib o‘tishini ko'rsating; parametri va uchini toping.

\textbf{1156.} $y^2+z^2=x$ elliptik paraboloidning $x+2 y-z=0$ tekislik bilan kesimining koordinata tekisliklaridagi proyeksiyalari tenglamalari topilsin.

\textbf{1159.1)} Berilgan tenglama bilan qaysi chiziq aniqlanishini toping: $\left\{\begin{array}{l}\frac{x^2}{3}+\frac{y^2}{6}=2 z, \\ 3 x-y+6 z-14=0\end{array}\right.$

\textbf{1159.2)} Berilgan tenglama bilan qaysi chiziq aniqlanishini toping: $\left\{\begin{array}{l}\frac{x^2}{4}-\frac{y^2}{3}=2 z \\ x-2 y+2=0 ;\end{array}\right.$

\textbf{1159.3)} Berilgan tenglama bilan qaysi chiziq aniqlanishini toping: $\left\{\begin{array}{l}\frac{x^2}{.4}+\frac{y^2}{9}-\frac{z^2}{36}=1, \\ 9 x-6 y+2 z-28=0,\end{array}\right.$