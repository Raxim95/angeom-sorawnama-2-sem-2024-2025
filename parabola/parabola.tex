583.1) Составить уравнение параболы, вершина которой находится в начале координат, зная, что: парабола расположена в правой полуплоскости симметрично относительно оси $O x$, и ее параметр $p=3$;
583.2) Составить уравнение параболы, вершина которой находится в начале координат, зная, что: парабола расположена в певой полуплоскости симметрично относительно оси $O x$, и ее параметр $p=0,5$;
583.3) Составить уравнение параболы, вершина которой находится в начале координат, зная, что: парабола расположена в верхней полуплоскости симметррично относительно оси $O y$, и ее параметр $p=\frac{1}{4}$;
583.4) Составить уравнение параболы, вершина которой находится в начале координат, зная, что: парабола расположена в нижней полуплоскости симметрично относительно оси $O y$, и ее параметр $p=3$.
584. Определить величину параметра и расположение относительно координатных осей следующих парабол: 1) $y^2=6 x$; 2) $x^2=5 y$; 3) $y^2=-4 x$; 4) $x^2=-y$.
585.1) Составить уравнение параболы, вершина которой находится в начале координат, зная, что: парабола расположена симметрично относительно оси $O x$ и проходит через точку $A(9 ; 6)$;
585.2) Составить уравнение параболы, вершина которой находится в начале координат, зная, что: парабола расположена симметрично относительно оси $O x$ и проходит через точку $B(-1 ; 3)$;
585.3) Составить уравнение параболы, вершина которой находится в начале координат, зная, что: парабола расположена симметрично относительно оси $O y$ и проходит через точку $C(1 ; 1)$.
585.4) Составить уравнение параболы, вершина которой находится в начале координат, зная, что: парабола расположена симметрично отнлсительно оси $O y$ и проходит через точку $D(4 ;-8)$.586. Стальной трос подвешен за два конца; точки креп.тения расположены на одинаковой высоте; расстояние между ними равно 20 м. Величина его прогиба на расстолиии 2 m от точки крепления, считая по горизонтали, равна 14,4 см. Определить величину прогиба этого троса в ссредине между точками крепления, приближенно считая, что трос имеет форму дуги параболы.
587. Составить уравнение параболы, которая имеет фокус $E(0 ;-3)$ и проходит через начало координат, зная, что ее осью служит ось $O y$.
588.1) Установить, какие линии определяются следующими уравнениями: $y=+2 \sqrt{x}$;
588.2) Установить, какие линии определяются следующими уравнениями: $y=+\sqrt{-x}$;
588.3) Установить, какие линии определяются следующими уравнениями: $y=-3 \sqrt{-2 x}$;
588.4) Установить, какие линии определяются следующими уравнениями: $y=-2 \sqrt{x}$;
588.5) Установить, какие линии определяются следующими уравнениями: $x=+\sqrt{5 y}$;
588.6) Установить, какие линии определяются следующими уравнениями: $x=-5 \sqrt{-y}$;
588.7) Установить, какие линии определяются следующими уравнениями: $x=-\sqrt{3 y}$;
588.8) Установить, какие линии определяются следующими уравнениями: $x=+4 \sqrt{-y}$.
589. Найти фокус $F$ и уравнение директрисы параболы $y^2=24 x$.
590. Вычислить фокальный радиус точки $M$ параболы $y^2=20 x$, если абсцисса точки $M$ равна 7 .
591. Вычислить фокальный радиус точки $M$ параболы $y^2=12 x$, если ордината точки $M$ равна 6 .
592. На параболе $y^2=16 x$ найти точки, фокэльный радиус которых равен 13.
593. Составить уравнение параболы, если дан фо~ кус $F(-7 ; 0)$ и уравнсние директрисы $x-7=0$.
594.1) Составить уравнение параболы, зная, что .ее вершина совпадает с точкой ( $\alpha ; \beta$ ), параметр равен $p$, ось параллельна оси $O x$ и парабола простирается в бесконечность: в положительном направлении оси $O x$;
594.2) Составить уравнение параболы, зная, что .ее вершина совпадает с точкой ( $\alpha ; \beta$ ), параметр равен $p$, ось параллельна оси $O x$ и парабола простирается в бесконечность: в отрицательном направлении оси $O x$.
594.1) Составить уравнение параболы, зная, что .ее вершина совпадает с точкой ( $\alpha ; \beta$ ), параметр равен $p$, ось параллельна оси $O x$ и парабола простирается в бесконечность: в положительном направлении оси $O x$;
594.2) Составить уравнение параболы, зная, что .ее вершина совпадает с точкой ( $\alpha ; \beta$ ), параметр равен $p$, ось параллельна оси $O x$ и парабола простирается в бесконечность: в отрицательном направлении оси $O x$.
596.1) Установить, что каждое из следующих уравнений определяет параболу, и найти координаты ее вершины $A$, величину параметра $p$ и уравнение директрисы: $y^2=4 x-8$;
596.2) Установить, что каждое из следующих уравнений определяет параболу, и найти координаты ее вершины $A$, величину параметра $p$ и уравнение директрисы: $y^2=4-6 x$;
596.3) Установить, что каждое из следующих уравнений определяет параболу, и найти координаты ее вершины $A$, величину параметра $p$ и уравнение директрисы: $x^2=6 y+2$
596.4) Установить, что каждое из следующих уравнений определяет параболу, и найти координаты ее вершины $A$, величину параметра $p$ и уравнение директрисы: $x^2=2-y$
597.1) Установить, что каждое из следующих уравнений определяет параболу, и найти координаты ее вершины $A$ и величину параметра $p$ : $y=\frac{1}{4} x^2+x+2$; 
597.2) Установить, что каждое из следующих уравнений определяет параболу, и найти координаты ее вершины $A$ и величину параметра $p$ : $y=4 x^2-8 x+7 ;$ 
597.3) Установить, что каждое из следующих уравнений определяет параболу, и найти координаты ее вершины $A$ и величину параметра $p$ : $y=-\frac{1}{6} x^2+2 x-7$
598. Установить, что каждое из следующих уравнений определяет параболу, и найти координаты ее вершина $A$ и величину параметра $p: 1) x=2 y^2-12 y+14$; 2) $x=-\frac{1}{4} y^2+y$; 3) $x=-y^2+2 y-1$.
599.1) Установить, какие линии определяются следующими уравнениями: $y=3-4 \sqrt{x-1}$;
599.2) Установить, какие линии определяются следующими уравнениями: $x=-4+3 \sqrt{y+5}$
599.3) Установить, какие линии определяются следующими уравнениями: $x=2-\sqrt{6-2 y}$;
599.4) Установить, какие линии определяются следующими уравнениями: $y=-5+\sqrt{-3 x-21}$.
600. Составить уравнение параболы, если даны ее фокус $F(7 ; 2)$ и директриса $x-5=0$
601. Составить уравнение параболы, если даны ее фокус $F(4 ; 3)$ и директриса $y+1=0$.
602. Составить уравнение параболы, если даны ее фокус $F(2 ;-1)$ и директриса $x-y-1=0$.
603. Даны вершина параболы $A(6 ;-3)$ и уравнение ее директрисы $3 x-5 y+1=0$. Найти фокус $F$ этой параболы.
604. Даны вершина параболы $A(-2 ;-1)$ и урав нение ее директрисы $x+2 y-1=0$. Составить уравнение этой параболы.
605. Определить точки пересечения прямой $x+y$ -$-3=0$ и параболы $x^2=4 y$.
606. Определить точки пересечения прямой $3 x+$; $1+4 y-12=0$ и параболы $y^2=-9 x$.
607. Определить точки пересечения прямой $3 x$ -$-2 y+6=0$ и параболы $y^2=6 x$.
608. В следующих случаях определить, как расположена данная прямая относительшо данной парабо-лы-пересекает ли, касается или проходит вне ее: 1) $x-y+2=0, y^2=8 x$; 2) $8 x+3 y-15=0, x^2=$ $=-3 y$; 3) $5 x-y-15=0, y^2=-5 x$.
609. Определить, при каких значениях углового коэффициента $k$ прямая $y=k x+2$ 1) пересекает параболу $y^2=4 x$; 2) касается ее; 3) проходит вне этсй параболы.
610. Вывести условие, при котором прямая $y=$ $=k x+b$ касается параболы $y^2=2 p x$.
611. Доказать, что к параболе $y^2=2 p x$ можно провести одну и только одну касательную с угловым коэффициентом $k \neq 0$.
612. Составить уравнение касагельной к параболе $y^2=2 p x$ в ее точке $M_1\left(x_1 ; y_1\right)$.
613. Составить уравнение прямой, которая касается параболы $y^2=8 x$ и параллельна прямой $2 x+2 y-$ $-3=0$.
614. Составить уравнение прямой, которая касается параболы $x^2=16 y$ и перпендикулярна к прямой $2 x+$. $+4 y+7=0$.
615. Провести касательную к параболе $y^2=12 x$ параллельно прямой $3 x-2 y+30=0$ и вычислить расстояние $d$ между этой касательной и данной прямой.
616. На параболе $y^2=64 x$ найти точку $M_1$, ближайшую к прямой $4 x+3 y-14=0$, и вычислить расстояние $d$ от точки $M_1$ до этой прямой.
617. Составить уравнения касательных к параболе $y^2=36 x$, проведенных из точки $A(2 ; 9)$.
618. K параболе $y^2=2 p x$ проведена касательная. 'Доказать, что вершина этой параболы лежит посредине между точкой пересечения касательной с осью Ox и проекцией точки касания на ось $O x$.
619. Из точки $A(5 ; 9)$ проведены касательные к параболе $y^2=5 x$. Составить уравнение хорды, соединяющей точки касания.
620. Из точки $P(-3 ; 12)$ проведены касательные к параболе $y^2=10 x$. Вычислить расстояние $d$ от точки $P$ до хорды параболы, соединяющей точки касания.
621. Определить точки пересечения эллипса $\frac{x^2}{100}+\frac{y^2}{225}=1$ и параболы $y^2=24 x$
622. Определить точки пересечения гиперболы $\frac{x^2}{20}$ -$-\frac{y^2}{5}=-1$ и параболы $y^2=3 x$
623. Определить точки пересечения двух парабол: $y=x^2-2 x+1, x=y^2-6 y+7$.
624. Доказать, что прямая, касающаяся параболы в некоторой точке $M$, составляет равные углы с фокальным радиусом точки $M$ и с лучом, который, исходя из $M$, идет параллельно оси параболы в ту сторону, куда парабола бесконечно простирается.
625. Из фокуса параболы $y^2=12 x$ под острым углом $\alpha$ к оси $O x$ направлен луч света. Известно, что $\operatorname{tg} \alpha=\frac{3}{4}$. Дойдя до параболы, луч от нее отразился. Составить уравнение прямой, на которой лежит отраженный луч.
626. Доказать, что две параболы, имеющие общую ось и общий фокус, расположенный между их вершинами, пересекаются под прямым углом.
627. Доказать, что если две параболы со взаимно перпендикулярными осями пересекаются в четырех точках, то эти точки лежат на одной окружности.
